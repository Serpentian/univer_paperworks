\introduction % Структурный элемент: ВВЕДЕНИЕ

Мир вычислительных технологий за свою жизнь пережил значительные изменения:
от монолитных приложений прошлых лет до современных микросервисов - подходы к
обработке данных претерпели глубокую трансформацию. Централизованные приложения,
некогда считавшиеся вершиной технологий, уже не отвечали запросам времени, и
цифровая реальность потребовала чего-то более гибкого, масштабируемого и
устойчивого.

Так наступила эпоха распределенных систем. Они разделяют задачи на части,
распределяя их между множеством узлов, работающих в гармонии друг с другом.
Любое высоконагруженное приложение или база данных использует распределенные
системы для обработки миллионов одновременных запросов.

Однако под всей этой кажущейся простотой скрывается фундаментальная проблема:
необходимость согласования действий и состояний множества узлов, разбросанных
по разным частям мира и часто подверженных сбоям. Здесь на помощь и приходят
алгоритмы консенсуса, которые в распределенных системах выступают хранителями
целостности данных, гарантами отказоустойчивости.

\subsection*{Обоснование актуальности темы исследования}

В современных распределённых системах алгоритмы консенсуса (такие как Raft,
Paxos) являются основой для обеспечения согласованности данных и состояний
между узлами. Учитывая растущую зависимость от распределённых
систем в различных отраслях, корректность этих алгоритмов критически важна
для обеспечения надёжности и доступности сервисов.

Формальная верификация предоставляет математически строгие методы проверки
алгоритмов. Использование TLA+ для анализа и проверки алгоритма Raft позволяет
доказать корректность его ключевых свойств. Это снижает вероятность ошибок и
позволяет проверить корректность дизайна алгоритма. Только после уверенности
в алгоритме можно приступать к его имплементации в коде.

Верификация алгоритмов распределённых систем становится всё более востребованной
в сфере информационных технологий. Компании, такие как Amazon, Google, и
Microsoft, активно используют формальные методы для проверки своих систем.
Исследование формальной верификации Raft с применением TLA+ предоставляет
актуальные знания и навыки, востребованные на рынке труда.

Таким образом, анализ и формальная верификация алгоритма Raft с использованием
TLA+ представляют собой актуальную тему, находящуюся на пересечении интересов
академического сообщества и индустрии IT. Результаты исследования будут полезны
как исследователям, изучающим механизмы согласования в распределённых системах,
так и практикующим разработчикам, создающим надёжные и отказоустойчивые системы.

\subsection*{Цели и задачи НИРС}

Целью исследования является изучение алгоритма Raft, его формальный
анализ и верификация. Для достижения этой цели предполагается решение
следующих задач:

\begin{itemize}
    \item изучение и описание принципов работы Raft;
    \item сравнительный анализ Raft и других алгоритмов консенсуса;
    \item формальная спецификация алгоритма Raft на TLA+ и проверка его корректности.
\end{itemize}

В рамках данной работы рассматриваются только невизантийские алгоритмы
консенсуса.

Стоит отметить, что алгоритмы консенсуса в распределенных системах с
византийскими ошибками не рассматриваются в данной работе.
