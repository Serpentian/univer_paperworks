\conclusion

В ходе проведённой работы были проанализированы основные аспекты реализации консенсуса в распределённых системах с учётом заранее определённых ограничений (модель сети, уровни синхронности и типы отказов). Рассмотрены и сопоставлены два известных алгоритма консенсуса — Paxos и Raft: первый отличается сильной теоретической базой, однако достаточно сложен для понимания и практической имплементации; второй (Raft) спроектирован с упором на простоту объяснения и реализует те же гарантии согласованности.

В практической части работы был выбран алгоритм Raft и выполнена его формальная спецификация на языке TLA+. С помощью соответствующей конфигурации модели была осуществлена проверка корректности поведения в рамках заданных условий. Результаты верификации подтвердили соответствие системы основным свойствам безопасности (отсутствие расхождений в закоммиченных данных) и продемонстрировали правильную работу механизма выбора лидера и репликации лога.

Таким образом, проделанное исследование показало, что формальное описание и проверка протокола Raft с помощью TLA+ существенно повышают надёжность разработки и помогают выявлять потенциальные ошибки на ранних этапах проектирования. Полученные спецификации и результаты моделирования могут служить основой для дальнейшего расширения модели (например, учёта большего количества серверов, более сложных типов отказов) или для сравнения с другими алгоритмами консенсуса в рамках дальнейших исследований.
