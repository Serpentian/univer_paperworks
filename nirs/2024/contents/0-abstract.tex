\abstract % Структурный элемент: РЕФЕРАТ

Ключевые слова: распределенные системы, задача двух генералов, невозможность
Фишера-Линча-Патерсона, консенсус, Raft, Paxos, язык спецификаций TLA+.

Основная цель работы — провести анализ алгоритма консенсуса Raft, описать
принципы его работы, сопоставить его с другими невизантийскими алгоритмами и
проверить его корректность с использованием языка спецификаций TLA+.

В процессе работы было проведено исследование отечественной и зарубежной
литературы по заданной теме, изучены подходы к достижению консенсуса в
распределенных системах. Была написана TLA+ спецификация одной из имплементаций
алгоритма Raft, в частности Raft с реконфигурацией реплик.

В результате данного исследования было установлено, что Raft проще для понимания
и реализации, но принципиального отличия между Raft и Paxos нет, а потому выбор
за разработчиками. Было установлено, что дизайн Raft соответствует предъявляемым
требованиям.
