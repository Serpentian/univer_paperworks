\section*{Введение}

\section{Анализ предметной области и постановка задачи}
\begin{itemize}
    \item Современные подходы к построению распределённых вычислительных систем.
    \item Проблемы согласованности и отказоустойчивости.
    \item Роль алгоритмов консенсуса.
    \item Обоснование выбора алгоритма Raft.
    \item Формулировка цели и задач исследования.
\end{itemize}

\section{Теоретические основы алгоритма Raft}
\begin{itemize}
    \item Модель системы и основные допущения (сеть, узлы, сбои).
    \item Принцип выбора лидера.
    \item Репликация и согласование логов.
    \item Обработка сбоев и восстановление.
    \item Доказанные свойства алгоритма (безопасность, живучесть).
\end{itemize}

\section{Проектирование архитектуры системы}
\begin{itemize}
    \item Общая архитектура системы распределённых вычислений.
    \item Модульная структура: серверная часть, клиентская часть, коммуникация.
    \item Описание протокола взаимодействия между узлами.
    \item Механизм выбора лидера и обновления состояния.
    \item Механизм обработки сбоев и переключения на нового лидера.
\end{itemize}

\section{Реализация системы распределённых вычислений}
\begin{itemize}
    \item Выбор технологий и инструментов.
    \item Реализация серверного узла.
    \item Реализация клиента.
    \item Организация логов и репликации.
\end{itemize}

\section{Экспериментальная часть и результаты тестирования}
\begin{itemize}
    \item Методика проведения экспериментов.
    \item Сценарии сбоев (отказ лидера, отказ узла, задержки сети).
    \item Результаты тестирования и их анализ.
    \item Оценка отказоустойчивости и масштабируемости.
\end{itemize}

\section{Перспективы дальнейших исследований}
\begin{itemize}
    \item Возможность интеграции с формальной спецификацией Raft.
    \item План разработки каркаса для проверки покрытия кода спецификацией TLA+.
    \item Расширение функциональности системы (поддержка реконфигурации, оптимизация производительности).
\end{itemize}

\section*{Заключение}

\section*{Список использованных источников}

\section*{Приложения}

--------------------------------------------------------------------------------
--------------------------------------------------------------------------------

\section*{Аннотации к главам}

\subsection*{Глава 1. Анализ предметной области и постановка задачи}

В этой главе рассматриваются современные подходы к построению распределённых
систем и анализируются их ключевые характеристики: масштабируемость,
отказоустойчивость и согласованность. Подробно описываются типичные проблемы,
возникающие при проектировании систем, работающих на множестве узлов, включая
сетевые задержки, отказы оборудования и необходимость достижения единого
состояния. Отдельное внимание уделяется понятию консенсуса и роли алгоритмов
консенсуса в обеспечении корректной работы распределённых систем. В конце главы
формулируются цель и задачи исследования, определяются требования к
разрабатываемой системе и обосновывается выбор алгоритма Raft как основы для
реализации.

\subsection*{Глава 2. Теоретические основы алгоритма Raft}

Глава посвящена подробному разбору алгоритма Raft. Описывается модель системы,
в которой работает алгоритм, и делаются необходимые допущения о типах сбоев и
коммуникационных условиях. Рассматриваются ключевые механизмы Raft: выбор
лидера, репликация логов, обработка клиентских запросов и механизм обеспечения
безопасности данных. Также анализируются доказанные свойства Raft, такие как
безопасность (safety) и живучесть (liveness), которые гарантируют корректность
и продолжение работы системы при сбоях. Эта глава создаёт теоретическую основу
для проектирования архитектуры системы в следующей части работы.

\subsection*{Глава 3. Проектирование архитектуры системы}

В данной главе разрабатывается архитектура распределённой системы, построенной
на алгоритме Raft. Описывается модульная структура, включающая серверные узлы,
их роли (лидер и последователи), а также клиентские компоненты. Приводится
описание протоколов взаимодействия между узлами, форматов сообщений и стратегии
обработки сбоев. Особое внимание уделяется выбору механизмов хранения логов,
синхронизации состояния и переключения лидера при его отказе. Результатом этой
главы является подробное описание архитектуры, готовой к реализации.

\subsection*{Глава 4. Реализация системы распределённых вычислений}

В этой главе приводятся детали реализации спроектированной архитектуры.
Описываются выбранные технологии и языки программирования, обосновывается выбор
библиотек и инструментов. Подробно рассматривается реализация серверного узла,
включая логику выбора лидера, обработку запросов и репликацию логов.
Рассматривается реализация клиентской части, обеспечивающей постановку задач на
выполнение и получение результатов. Глава завершается описанием механизмов
логирования и мониторинга системы для упрощения отладки и анализа работы.

\subsection*{Глава 5. Экспериментальная часть и результаты тестирования}

В этой главе проводится экспериментальная проверка работоспособности
реализованной системы. Описывается методика тестирования, включающая сценарии
отказов узлов, сетевых задержек и восстановления после сбоев. Приводятся
результаты экспериментов, демонстрирующие сохранение согласованности данных и
продолжение выполнения задач при частичных сбоях. Анализируются показатели
отказоустойчивости, времени переключения на нового лидера и масштабируемости
системы при увеличении числа узлов. Результаты сравниваются с ожидаемыми
свойствами алгоритма Raft и формулируются выводы.

\subsection*{Глава 6. Перспективы дальнейших исследований}

Заключительная глава посвящена обсуждению возможных направлений развития
системы и применения полученных результатов. Рассматривается перспектива
интеграции созданной реализации с формальной спецификацией алгоритма Raft,
разработанной ранее, для проверки соответствия кода спецификации. Описывается
концепция каркаса для тестирования покрытия кода формальной моделью, который
планируется разработать в будущих исследованиях. Дополнительно приводятся
возможные пути расширения функциональности системы, такие как поддержка
динамической реконфигурации кластера, улучшение производительности и
оптимизация сетевого взаимодействия. Эта глава демонстрирует, что работа не
только решает поставленные задачи, но и создаёт научно-практический задел для
дальнейших исследований.
