\introduction % Структурный элемент: ВВЕДЕНИЕ

Современные программные системы становятся всё более сложными и
распределёнными. Увеличение количества пользователей, потребность в высокой
доступности сервисов и минимальных задержках вынуждают разработчиков переходить
от монолитных решений к распределённым архитектурам. Подобные системы позволяют
масштабировать вычислительные ресурсы, распределять нагрузку между множеством
узлов и сохранять работоспособность даже при частичных отказах оборудования.

Однако разработка распределённых систем связана с рядом фундаментальных
проблем. Одной из ключевых задач является обеспечение согласованности состояния
между узлами, которые могут работать в разных географических точках, испытывать
сетевые задержки или полностью выходить из строя. Для решения этой задачи
применяются алгоритмы консенсуса, гарантирующие, что все узлы системы приходят
к единому решению, несмотря на наличие сбоев.

Алгоритм Raft является одним из наиболее популярных и практически применимых
решений для построения отказоустойчивых систем. Он предлагает понятный механизм
выбора лидера и репликации логов, что делает его удобным для реализации и
интеграции в реальные проекты. Использование Raft позволяет создавать системы,
способные обеспечивать согласованное выполнение вычислительных задач и
сохранять данные даже при сбоях отдельных узлов.

\subsection*{Обоснование актуальности темы исследования}

С каждым годом увеличивается зависимость бизнеса и общества от информационных
систем, работающих круглосуточно и без сбоев. Даже кратковременная
недоступность может привести к существенным финансовым потерям, снижению
доверия пользователей и репутационным рискам. Поэтому разработка
отказоустойчивых распределённых решений является одной из ключевых задач
современной ИТ-индустрии.

Алгоритм Raft получил широкое распространение благодаря простоте понимания и
доказанной надёжности. Его использование позволяет снизить вероятность
критических ошибок и облегчить построение согласованных кластеров узлов.
Разработка системы, использующей Raft, позволяет на практике изучить принципы
построения отказоустойчивых систем и проверить их работу в реальных сценариях
сбоев.

Дополнительно данная работа закладывает основу для следующего этапа
исследования — разработки инструментария для тестирования покрытия спецификацией
TLA+ реального кода. Созданная в рамках этой работы реализация алгоритма Raft
будет использована в качестве экспериментальной базы для автоматической
проверки соответствия кода его формальной спецификации. Формальная верификация
будет взята из предыдущей научной работы. Такой подход позволит объединить
инженерные и формальные методы верификации, повысив надёжность проектируемых
распределённых систем.

Таким образом, выбранная тема является актуальной, так как направлена на
исследование и практическую реализацию решений, повышающих надёжность и
доступность распределённых вычислительных систем, а также создаёт основу для
дальнейших исследований в области автоматизации верификации и ее
поддерживаемости.

\subsection*{Цели и задачи НИРС}

Целью работы является разработка и исследование отказоустойчивой системы
распределённых вычислений, основанной на алгоритме консенсуса Raft, а также
подготовка практической базы для будущей интеграции с инструментами формальной
верификации.

Для достижения этой цели решаются следующие задачи:

\begin{itemize}
    \item провести обзор принципов построения отказоустойчивых распределённых
          систем;
    \item изучить устройство алгоритма Raft и описать его ключевые механизмы;
    \item спроектировать архитектуру системы распределённых вычислений с
          использованием Raft;
    \item реализовать прототип системы и провести его ручное тестирование при
          сбоях узлов;
\end{itemize}

