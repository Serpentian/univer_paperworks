\conclusion

В рамках проведённого исследования была разработана и реализована
распределённая система вычислений, использующая алгоритм консенсуса Raft для
обеспечения согласованности состояния между узлами. Работа стала логическим
продолжением предыдущего научно-исследовательского проекта, в котором была
построена и формально проверена спецификация алгоритма Raft на языке TLA+. В
настоящей работе сделан следующий шаг — разработана практическая реализация
системы, включающая модуль Raft, реплицированную машину состояний с поддержкой
снимков данных, клиентскую часть для постановки задач и воркеры для их выполнения.
Проведено ручное тестирование кластера, включающее сценарии смены лидера,
репликации команд и обработку пользовательских задач, что подтвердило
корректность реализации и её устойчивость к отказам.

При этом стратегическая цель проекта выходит за рамки одной реализации: создать
фреймворк для C/C++, ориентированный на промышленную интеграцию (в первую
очередь в СУБД \textit{Tarantool}), который обеспечит непрерывную связь между
формальной спецификацией и реальным кодом. Предыдущая НИРС дала нам
спецификацию Raft на TLA+; текущая работа — рабочую систему. Следующий шаг —
соединить эти два мира инженерным инструментом, чтобы спецификация не
«устаревала» по мере эволюции кода и чтобы расхождения детектировались
автоматически на уровне трасс исполнения.

Планируемый фреймворк будет собирать во время тестов трассы состояний реальной
системы (в debug-сборках и при управляемом однонитевом исполнении): выбранные
переменные и внутренние маркеры будут помечаться препроцессором, а их значения
фиксироваться между \emph{yield}/точками квазидетерминизма в лог-файл. Затем
трасса преобразуется в абстракцию, сопоставимую со стейтами TLA+, и
сравнивается с поведением модели; отсутствие соответствующей траектории в
спецификации трактуется как сигнал к обновлению модели или кодификации
недостающих предпосылок. Для тяжёлых эффектов (например, запись на диск)
допускается замещение «сложных» функций упрощёнными переходами состояния — это
позволит отделить функциональную корректность от особенностей подсистем
ввода-вывода и повысить наблюдаемость. Концептуально подход опирается на идеи
трасс/проекций состояний и синхронизации модели с реализацией
\cite{trace_pn,tlacoverage,merz2024}.

Интеграция в \textit{Tarantool} предполагает два направления: во-первых,
инжектируемые C/C++–хуки и лёгкие макросы для маркировки наблюдаемых переменных
и точек переходов в существующем коде ядра; во-вторых, конвейер тестирования
(unit, интеграционные, стохастические тесты), который автоматически собирает и
верифицирует трассы в CI. Такой процесс создаёт контур \emph{continuous
verification}: каждая регрессия и каждое изменение протокола отражаются
одновременно в кодовой базе и в TLA+ спецификации, снижая стоимость расхождений
«модель↔код» и ускоряя обратную связь для разработчиков.

Таким образом, проделанная работа сформировала инженерный фундамент (реализация
Raft) и одновременно задала методологический вектор: переход от разовой
формальной проверки к систематической, автоматизированной и интегрированной в
жизненный цикл разработки. Разработка C/C++ фреймворка для синхронизации
спецификации и кода и его последующая интеграция в \textit{Tarantool} является
естественным и практически значимым продолжением проекта, направленным на
повышение надёжности, воспроизводимости и эволюционной управляемости
распределённых компонентов.

