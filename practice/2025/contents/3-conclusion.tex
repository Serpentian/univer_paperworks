\conclusion

В ходе прохождения практики был проведён комплексный анализ механизма
шардирования в СУБД Tarantool. Основное внимание было уделено изучению
архитектуры модуля \texttt{vshard}, принципов распределения данных и
обеспечения согласованности при выполнении операций в шардированном кластере.

Были рассмотрены и проанализированы существующие подходы к реализации
Map-Reduce запросов по репликам. В результате исследования выявлены ключевые
проблемы, связанные с обеспечением консистентности данных при выполнении
распределённых запросов, и предложена альтернативная реализация.

Практическая значимость работы заключается в:
\begin{itemize}
    \item Систематизации знаний о работе шардированного кластера Tarantool;
    \item Выявлении ограничений существующей реализации модуля \texttt{vshard};
    \item Разработке предложений по расширению функциональности для поддержки
        Map-Reduce операций по репликам.
\end{itemize}

Полученные результаты могут быть использованы для дальнейшего развития модуля
шардирования Tarantool и улучшения его безопасности. Проведённое исследование
демонстрирует важность комплексного подхода к проектированию распределённых
систем и необходимость тщательного анализа требований к согласованности данных.

Результаты работы подтверждают возможность реализации эффективного механизма
выполнения Map-Reduce запросов по репликам в шардированной среде с соблюдением
требований к консистентности данных и производительности системы.
