\structure{ОСНОВНАЯ~ЧАСТЬ}

\section{Характеристика организации}

ВK Цифровые Технологии – подразделение VK, развивающее продукты и сервисы для
цифрового бизнеса. В основе экосистемы решений VK Цифровые технологии лежит
многолетний опыт развития интернет-сервисов и технологии на базе открытого
кода. VK Цифровые Технологии предоставляет готовые сервисы для решения бизнес
задач любой сложности, занимается заказной разработкой и управлением
ИТ-инфраструктурой \cite{VkTech}.

В портфеле VK цифровые Технологии — облачные сервисы VK Cloud Solutions,
платформа in-memory вычислений Tarantool, платформа взаимодействия бизнеса и
государства VK Tax Monitoring, а также линейка программных продуктов для
управления персоналом, автоматизации производства и бизнес-процессов.

Tarantool как продукт появился 4 апреля 2016 года, когда Mail.ru Group (на
данный момент известная как VK) сообщила о создании нового направления бизнеса,
в рамках которого компания начала предоставлять корпоративным клиентам услуги в
области хранения данных.

Изначально Tarantool применялся только в собственных проектах Mail.ru, в том
числе в почтовом сервисе и облачном хранилище «Облако Mail.Ru». Затем компания
превратила эту СУБД в продукт с открытым исходным кодом, который к началу
апреля 2016 года внедрен рядом российских и международных компаний. В
частности, Tarantool начал использоваться сервисом бесплатных объявлений Avito,
социальной сетью знакомств Badoo и разработчиком систем информационной
безопасности Wallarm.

На сегодняшний день Tarantool активно используется в банковской сфере (среди
клиентов Tarantool можно выделить ВТБ, Альфа Банк, Банк Открытие и Газпромбанк)
и для e-commerce (Магнит, Wildberries, Sitilink, X5Group)
\cite{Tarantool}.

Перечень лицензий организации приведен в таблице 1.

\setcounter{table}{0}
\begin{table}
\small % Уменьшаем размер шрифта
\centering
\caption{Перечень лицензий организации}
\begin{tabularx}{\textwidth}{|l|X|>{\raggedright\arraybackslash}p{3cm}|}
\hline
\textbf{№ Лицензии и дата} & \textbf{Вид деятельности} & \textbf{Орган выдачи} \\
\hline
3599 от 07.05.2019 & Деятельность по технической защите конфиденциальной информации & ФСТЭК России \\
\hline
1877 от 07.05.2019 & Разработка и производство средств защиты конфиденциальной информации & ФСТЭК России \\
\hline
Л024-00107-00/00384565 от 07.05.2019 & Деятельность по технической защите конфиденциальной информации & ФСТЭК России \\
\hline
Л050-00107-00/00384566 от 07.05.2019 & Разработка и производство средств защиты конфиденциальной информации & ФСТЭК России \\
\hline
Л051-00105-00/00639121 от 30.01.2023 &
Разработка, производство, распространение шифровальных (криптографических) средств, информационных систем и телекоммуникационных систем, защищенных с использованием шифровальных (криптографических) средств &
ФСБ России \\
\hline
\end{tabularx}
\label{tab:licenses}
\end{table}
