\section{Шардирование СУБД Tarantool}

В данном разделе будет подробно рассмотрен механизм шардирования в СУБД
Tarantool. Будет описан процесс настройки и способ использования шардированого
кластера. Также будут рассмотрены некоторые из внутренних сервисов модуля
шардирования, которые необходимы для понимания предлагаемой реализации
Map-Reduce запроса по репликам.

\subsection{Общая информация}

В Tarantool группа узлов, которые работают с копиями одной и той же базы
данных, объединяется в репликасет. Внутри репликасета каждому узлу назначается
определенная роль: мастер или реплика. Только мастер может обрабатывать DDL,
DML и DCL запросы, реплики же обрабатывают только DQL. Внутри репликасета
между узлами идет процесс репликации, то есть данные копируются.

За шардирование в Tarantool отвечает модуль \textit{vshard}
\cite{VshardGithub}. Он написан на языке Lua, так как Tarantool представляет
собой не только СУБД, но и может служить сервером приложений, логика которого
пишется на Lua.

\subsection{Настройка шардированного кластера СУБД}

\textbf{Шаг 1. Подготовка узлов}

Для настройки репликасета необходимо подготовить несколько узлов, на которых будет запущен Tarantool. Каждый узел должен быть настроен и доступен для связи с другими узлами в кластере. Необходимо убедиться, что на каждом узле установлена одна и та же версия Tarantool. В примере настройка репликасета производится локально, на одном компьютере. \\

\textbf{Шаг 2. Конфигурация мастера}

Первым шагом является настройка мастера, который будет основным источником данных для репликации. Его необходимо сконфигурировать со следующими параметрами \cite{TarantoolDoc}:

\begin{itemize}
    \item \textit{listen} - URI, на котором узел принимает входящие подключения.
    \item \textit{replication} - список URI, с которых узел реплицирует данные.
\end{itemize}

Пример конфигурирования мастера приведен на рисунке~\ref{fig:fig01}.

\begin{figure}
  \centering
  \includegraphics[scale=0.35]{inc/master.png}
  \caption{Конфигурация мастера}
  \label{fig:fig01}
\end{figure}

\textbf{Шаг 3. Конфигурация реплики}

Реплики настраиваются аналогичным образом, но с указанием опции \textit{read\_only}, устанавливающей режим только для чтения, чтобы запретить выполнение операций записи на реплике. Реплику можно сделать анонимной, указав параметр \textit{replication\_anon}.

\textbf{Шаг 4. Проверка состояния репликасета}

После настройки и запуска всех узлов необходимо убедиться, что репликасет работает корректно. Сделать это можно с помощью \textit{box.info.replication}, как показано на рисунке~\ref{fig:fig03}. Эта команда включает информацию о подключении к узлу, статусе синхронизации (\textit{vclock}) и задержке (\textit{lag}) \cite{TarantoolDoc}.

\begin{figure}
  \centering
  \includegraphics[scale=0.35]{inc/master-info.png}
  \caption{Состояние подключения узла}
  \label{fig:fig03}
\end{figure}


\subsection{Подключение узла к репликасету}

В данной части описывается протокол подключения реплики, так как это необходимо для понимания решений, предлагаемых к поставленным во введении задачам.

\begin{enumerate}
    \item В ходе конфигурации реплика генерирует UUID, являющийся уникальным идентификатором этого узла в репликасете. Для каждого URI в \textit{box.cfg.replication} создается сущность, называемая applier, задача которой состоит в применении данных, получаемых от мастера. Applier инициирует подключение к мастеру.
    \item Applier посылает сообщение IPROTO\_JOIN, при получении которого мастер создает relay, нужный для пересылки изменений БД на реплику. IPROTO\_JOIN представляет собой запрос на добавление узла к кластеру и необходим для получение начального состояния с мастера. Процесс отсылки начального состояния делится на две фазы.
    \item Initial JOIN. Мастер создает read-view (снимок) текущего состояние БД и посылает его реплике. По окончании initial JOIN мастер добавляет узел в репликасет путем вставки UUID реплики с соответсвующим ID в спейс \_cluster.
    \item Final JOIN. С момента создания read-view до окончания его пересылки может пройти много времени и состоянии БД наверняка изменится. Потому в фазе final JOIN мастер посылает все изменения, появившиеся со времени начала пересылки read-view.
    \item По окончании JOIN, реплика посылает запрос IPROTO\_SUBSCRIBE. Мастер отвечает своим текущим vclock-ом. С этого момента реплика переходит в стадию FOLLOW, она применяет все обновления WAL, исходящие от мастера.
\end{enumerate}

В Tarantool также есть возможность создания анонимных реплик, которые не являются участником репликасета, не могут становиться мастером, не учавствуют в кворуме синхронной репликации. Однако они получают и применяют поток репликационных данных. Вместо IPROTO\_JOIN они посылают IPROTO\_FETCH\_SNAPSHOT, который выполняет только первую фазу подключения: initial JOIN. Анонимные реплики не добавляются в спейс \_cluster. В любой момент анонимная реплика может стать обычной, послав запрос IPROTO\_REGISTER.
