\conclusion

В процессе выполнения работы были изучены и реализованы ключевые подходы к решению задачи точечной маршрутизации интернет-трафика на основе доменов с использованием возможностей операционной системы OpenWrt. Были подробно рассмотрены принципы работы DNS-серверов, механизмы маршрутизации и перенаправления трафика, а также инструменты, обеспечивающие защиту от возможного вмешательства со стороны провайдеров, такие как DNSCrypt-proxy2 и Stubby.

На основании проведённой работы удалось продемонстрировать эффективное решение задачи маршрутизации трафика через туннель для заданных доменов, что позволяет обеспечить стабильную работу даже в условиях ограничений со стороны интернет-провайдеров. Практическая ценность данной работы заключается в возможности автоматизации описанных процессов, что снижает порог сложности для внедрения подобных решений на устройствах конечных пользователей.

Результаты работы подтверждают актуальность применения OpenWrt и сопутствующих инструментов для реализации сложных сценариев маршрутизации и управления трафиком. Полученные выводы и предложенные методы могут быть полезны как для дальнейших исследований в области сетевых технологий, так и для практического использования в целях обеспечения конфиденциальности, обхода блокировок и повышения гибкости настроек сетевой инфраструктуры.
