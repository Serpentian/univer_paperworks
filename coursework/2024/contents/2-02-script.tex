\section{Автоматическая настройка}

Для автоматизации настроек точечной маршрутизации на роутере с OpenWrt, был создан скрипт, который выполняет шаги, описанные в предыдущей части. Полный код скрипта представлен в Приложении 1. Скрипт предполагает свежую установку OpenWrt или минимальные изменения в текущей конфигурации. Если на роутере уже есть другие туннели или маршруты, выполнение скрипта может привести к некорректной настройке, поэтому рекомендуется предварительно проверить текущую конфигурацию.

\begin{enumerate}
    \item Выбор туннеля: Скрипт предложит выбрать туннель. Для WireGuard (WG) настройка будет выполнена автоматически. Для OpenVPN, sing-box или tun2socks потребуется вручную настроить соответствующие клиенты. Для sing-box скрипт создаст шаблон конфигурации.
    \item Подмена DNS-запросов: Если провайдер подменяет DNS-запросы, скрипт предложит установить резолверы, такие как DNSCrypt-proxy2 или Stubby. DNSCrypt-proxy2 более функционален, но занимает больше памяти, в то время как Stubby более легковесен.
    \item Выбор доменов: Скрипт позволяет легко выбрать список доменов, для которых будет настроена точечная маршрутизация. Эти списки загружаются автоматически и добавляются в конфигурацию DNS-сервера.
    \item Смена туннелей: Скрипт также позволяет легко переключать туннели. Например, при переходе с OpenVPN на sing-box он автоматически обновит файрвол, но для предотвращения конфликтов интерфейсов потребуется вручную остановить старую службу или изменить интерфейс.
\end{enumerate}

Этот скрипт значительно упрощает настройку, но в сложных случаях, когда требуются изменения в существующих туннелях или маршрутах, может потребоваться ручное вмешательство.

