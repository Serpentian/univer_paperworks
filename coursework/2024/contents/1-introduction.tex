\introduction % Структурный элемент: ВВЕДЕНИЕ

Современные сети требуют гибкого и эффективного управления трафиком, особенно в условиях стремительного роста числа подключенных устройств и использования сложных сетевых топологий. Одним из ключевых инструментов для таких задач является маршрутизация, позволяющая направлять трафик по различным маршрутам в зависимости от его назначения. В частности, точечная маршрутизация по доменам открывает новые возможности для управления потоками данных, позволяя выделять трафик определённых ресурсов или групп доменов и перенаправлять его через заранее заданные маршруты.

В данной курсовой работе рассматривается реализация точечной маршрутизации по доменам на роутерах с прошивкой OpenWrt. OpenWrt, как одна из наиболее популярных и гибких прошивок для роутеров, предоставляет мощные инструменты для настройки сетевых параметров, включая работу с VPN-туннелями, списками доменов и кастомными правилами маршрутизации. Такие решения актуальны как для повышения конфиденциальности (через использование VPN), так и для обеспечения избирательного доступа к ресурсам или обхода сетевых ограничений.

Целью работы является анализ методов настройки маршрутизации по доменам на маршрутизаторах с OpenWrt и разработка оптимального решения, позволяющего пользователям гибко управлять трафиком. В ходе исследования планируется рассмотреть теоретические аспекты маршрутизации и провести практическую реализацию на примере конкретной конфигурации.
