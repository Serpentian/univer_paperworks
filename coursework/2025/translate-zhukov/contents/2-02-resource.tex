\section{Оценка квантовых ресурсов}

Ниже приведены квантовые ресурсы, необходимые для того чтобы поставить под
угрозу некоторые реальные числа RSA, основываясь на алгоритме SQIF, который
описан в данной работе. Основные упоминаемые ресурсы — число кубитов и глубина
квантовой схемы одного слоя QAOA. Как правило, квантовые схемы нельзя запускать
на квантовых устройствах напрямую, поскольку их проект не учитывает топологию
связей реальных физических систем. Исполнение часто требует дополнительных
ресурсов — вспомогательных кубитов и увеличения глубины схем. Мы рассмотрели
ресурсы для трёх типовых топологий: полностью связанной системы ($K_n$),
двумерной решётки (2DSL --- 2D-lattice system) и одномерной линейной цепочки
(LNN --- 1D-chain system). Мы показываем на конкретных схемах, что встраивание
не требует лишних кубитов, а глубина одного слоя QAOA остается $O(n)$ во всех
трёх случаях. Следовательно, для факторизации целых чисел нашим алгоритмом
необходимы сублинейные квантовые ресурсы. Возьмём RSA‑2048 в качестве примера,
в этом случае требуемое число кубитов $n = {2 \cdot 2048} / {\log 2048} \approx
372$. Глубина одного слоя QAOA составляет $1118$ для топологии $K_n$, $1139$
для 2DSL и $1490$ для самой простой LNN, что достижимо для устройств NISQ в
ближайшем будущем. Квантовые ресурсы для разных длин чисел RSA приведены в
таблице \ref{tab:tab1}. Подробный анализ дан в \cite{cite_31}.

\begin{table}[h]
    \centering
    \caption{
        Оценка ресурсов для чисел RSA. Основные квантовые ресурсы включают
        число кубитов и глубину квантовой схемы QAOA с одной итерацией в трёх
        типовых топологиях: полностью связанной системе ($K_n$), двумерной
        решётке (2DSL) и линейной цепочке (LNN). Результаты получены без учёта
        нативной компиляции базового модуля ZZ (или базового модуля ZZ‑SWAP) в
        конкретной физической системе.
    }
    \begin{tabular}{lrrrr}
        \hline\hline
        \text{RSA number} & \text{Qubits} & \text{$K_n$‑depth} &
        \text{2DSL‑depth} & \text{LNN‑depth} \\
        \hline
        RSA‑128  &  37 &  113 &  121 &  150 \\
        RSA‑256  &  64 &  194 &  204 &  258 \\
        RSA‑512  & 114 &  344 &  357 &  458 \\
        RSA‑1024 & 205 &  617 &  633 &  822 \\
        RSA‑2048 & 372 & 1118 & 1139 & 1490 \\
        \hline\hline
    \end{tabular}
    \label{tab:tab1}
\end{table}
