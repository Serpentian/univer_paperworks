\appendixsection{Предварительная обработка: детали факторизации}

\subsection*{Построение решётки и целевого вектора}

В качестве примера возьмём факторизацию числа \(N = 48567227\) на 5 кубитах,
чтобы показать вычислительные шаги до квантовой части. К ним относятся
построение решётки и целевого вектора, LLL‐редукция и выполнение алгоритма
ближайшей плоскости Бабая. Случаи с 3 и 10 кубитами будут приведены сразу.
Здесь применяется сублинейная схема размерности решётки. Размерность,
необходимая для факторизации \(N = 48567227\), равна \(\log N / \log\log N =
26/5 \approx 5\). База простых чисел состоит из первых пяти простых: \(\{-1, 2,
3, 5, 7, 11\}\).

Чтобы получить достаточно случайных целочисленных решёток, мы слегка изменяем
построение решётки из уравнения Б.7. Во‑первых, вместо диагонали \(\sqrt{\ln
p_i}\) используем \(\lceil i/2 \rfloor\), где \(\lceil\rfloor\) — округление к
ближайшему целому. Во‑вторых, чтобы получить различные fac‐отношения, применяем
случайную перестановку \(f\) диагональных элементов. Кроме того, вместо
«весового» элемента \(N^{c}\) используем \(10^{c}\). Таким образом, если \(c\)
— целое число, решётку легко сделать целочисленной, а \(c\) прямо задаёт
точность. Конкретная структура решётки задаётся уравнениями Г.33 и Г.34:

\begin{equation}
B_{n,c}=
\begin{pmatrix}
f(1) & 0 & \dots & 0 \\
0 & f(2) & \dots & 0 \\
\vdots & \vdots & \ddots & \vdots \\
0 & 0 & \dots & f(n) \\
\bigl[10^{c}\ln 2\bigr] & \bigl[10^{c}\ln 3\bigr] & \dots & \bigl[10^{c}\ln 11\bigr]
\end{pmatrix},
\end{equation}

\begin{equation}
t_{n}=
\begin{pmatrix}
0 \\ \vdots \\ 0 \\ \bigl[10^{c}\ln N\bigr]
\end{pmatrix}.
\end{equation}

Здесь \(B_{n,c}\) — матричная форма решётки, каждый столбец является вектором
базиса. Подстрочные индексы обозначают размерность \(n\) и параметр точности
\(c\). В случае с 5 кубитами размерность равна 5, а параметр точности — 4.
Элементы \(f(i)\) на диагонали — случайная перестановка множества \(\{\lceil
1/2 \rfloor, \dots, \lceil 5/2 \rfloor\} = \{1, 1, 2, 2, 3\}\). Соответственно
решётка и целевой вектор, относящиеся к sr‑паре, даны в уравнениях Г.35 и Г.36:

\begin{equation}
B_{5,4}=
\begin{pmatrix}
2     & 0     & 0     & 0     & 0     \\
0     & 1     & 0     & 0     & 0     \\
0     & 0     & 3     & 0     & 0     \\
0     & 0     & 0     & 2     & 0     \\
0     & 0     & 0     & 0     & 1     \\
6931  & 10986 & 16094 & 19459 & 23979
\end{pmatrix},
\end{equation}

\begin{equation}
t_{5}=
\begin{pmatrix}
0\\
0\\
0\\
0\\
0\\
176985
\end{pmatrix}.
\end{equation}

Аналогично для случая с 3 кубитами выполняются условия \(n = 3\) и \(c = 1.5\).
Соответствующие решётка и целевой вектор, соответсвующий sr-паре:
\begin{equation}
B_{3,1.5} =
\begin{pmatrix}
1 & 0 & 0 \\
0 & 1 & 0 \\
0 & 0 & 2 \\
22 & 35 & 51
\end{pmatrix},
\qquad
t_{3} =
\begin{pmatrix}
0 \\ 0 \\ 0 \\ 240
\end{pmatrix}.
\end{equation}

В случае с 10 кубитами выполняются условия \(n = 10\) и \(c = 4\). Решётка и
целевой вектор для sr‑пары приведены в уравнениях Г.38 (пропущено в изначальной
работе) и Г.39:

\begin{equation}
B_{10,4}=
\begin{pmatrix}
 3    & 0    & 0    & 0    & 0    & 0    & 0    & 0    & 0    & 0    \\
 0    & 2    & 0    & 0    & 0    & 0    & 0    & 0    & 0    & 0    \\
 0    & 0    & 3    & 0    & 0    & 0    & 0    & 0    & 0    & 0    \\
 0    & 0    & 0    & 1    & 0    & 0    & 0    & 0    & 0    & 0    \\
 0    & 0    & 0    & 0    & 1    & 0    & 0    & 0    & 0    & 0    \\
 0    & 0    & 0    & 0    & 0    & 3    & 0    & 0    & 0    & 0    \\
 0    & 0    & 0    & 0    & 0    & 0    & 1    & 0    & 0    & 0    \\
 0    & 0    & 0    & 0    & 0    & 0    & 0    & 1    & 0    & 0    \\
 0    & 0    & 0    & 0    & 0    & 0    & 0    & 0    & 2    & 0    \\
 0    & 0    & 0    & 0    & 0    & 0    & 0    & 0    & 0    & 2    \\
 6931 & 10986& 16094& 19459& 23979& 25649& 28332& 29444& 31355& 33673
\end{pmatrix}
\end{equation}

\begin{equation}
t_{10} = (0\; 0\; 0\; 0\; 0\; 0\; 0\; 0\; 0\; 0\; 331993)^{T}.
\end{equation}

\subsection*{Решение CVP с помощью алгоритма Бабая}

Гладкую пару отношений можно получить, решив CVP на построенной выше решётке.
До применения квантового метода приближённое оптимальное решение CVP извлекают
классическим алгоритмом редукции решётки (алгоритм Бабая). Сначала на базис
решётки выполняется LLL‑редукция с параметром \(\delta=3/4\).
LLL‑редуцированные базисы для трёх случаев факторизации обозначаются,
соответственно, \(D_{3,1.5}\) (Г.40), \(D_{5,4}\) (Г.41) и \(D_{10,4}\) (Г.42):

\begin{equation}
D_{3,1.5} =
\begin{pmatrix}
 1 & -4 & -3 \\
-2 &  1 &  2 \\
 2 &  2 &  0 \\
 3 & -2 &  4
\end{pmatrix}
\end{equation}

\begin{equation}
D_{5,4} =
\begin{pmatrix}
  6 & -8 &  2 & -4 & -4 \\
 -4 & -3 & 11 & -5 & -3 \\
  6 &  6 &  3 &  0 & -3 \\
  4 & -2 &  0 & 12 &  4 \\
 -2 &  2 & -6 & -2 &  1 \\
 -3 &  5 & -3 &  4 & -17
\end{pmatrix}
\end{equation}

\begin{equation}
D_{10,4} =
\begin{pmatrix}
 0 &  0 &  3 &  0 &  0 &  0 &  3 &  0 & -3 & -3 \\
 0 &  2 &  0 &  4 & -4 &  0 &  4 & -2 &  4 &  0 \\
-3 &  0 &  0 &  0 &  0 & -3 &  0 &  0 &  0 &  0 \\
 1 &  2 &  1 &  4 & -2 & -2 &  0 & -1 &  0 &  0 \\
 2 &  0 &  0 & -2 &  0 &  1 & -1 &  0 &  4 &  0 \\
 0 &  0 & -3 & -3 &  0 &  0 &  1 &  0 & -3 &  3 \\
-3 &  3 &  1 &  0 &  1 &  2 &  1 &  2 & -2 & -1 \\
 0 & -2 &  0 &  1 &  2 & -1 &  1 & -3 &  3 & -3 \\
 2 & -2 &  0 & -2 &  0 &  1 &  2 &  0 &  2 &  2 \\
 2 & -2 &  2 &  0 &  2 & -2 &  2 &  2 &  0 &  0 \\
 0 & -2 & -2 &  0 &  1 &  3 &  1 & -2 & -2 & -1
\end{pmatrix}
\end{equation}

Затем выполняется процедура уменьшения размера. Берётся самый «длинный» вектор
LLL‑базиса (правый столбец матрицы \(D_{5,4}\)) и пошагово вычитается из
целевого вектора \(t_{5}\), используя округлённые коэффициенты Грама–Шмидта
\(\mu_{i}\;(i=1,\dots,5)\), пока не будет обработан самый «короткий» базисный
вектор (левый столбец). В конце получают вектор расстояния \(\bar t_{5}\) и
приближённый ближайший вектор \(b_{\mathrm{op}}\). Поскольку длина \(\bar
t_{5}\) характеризует качество решения CVP, её также называют «коротким
вектором» в контексте CVP. Классические оптимальные решения, найденные
алгоритмом Бабая, приведены в Г.43–Г.46.
\begin{equation}
b_{\text{op}} =
(2\;4\;9\;8\;0\;176993)^{T},
\qquad
\bar t_{5}=b_{\text{op}}-t_{5}=
(2\;4\;9\;8\;0\;8)^{T}.
\end{equation}
\begin{equation}
b_{\text{op}} =
(0\;4\;4\;242)^{T},
\qquad
\bar t_{3}=b_{\text{op}}-t_{3}=
(0\;4\;4\;2)^{T}.
\end{equation}
\begin{equation}
b_{\text{op}} =
(3\;4\;0\;1\;2\;3\;2\;3\;2\;2\;331993)^{T},
\end{equation}
\begin{equation}
\bar t_{10}=b_{\text{op}}-t_{10}=
(3\;4\;0\;1\;2\;3\;2\;3\;2\;2\;0)^{T}.
\end{equation}

Приближённый ближайший вектор оказывается довольно далёк от целевого \(t_{5}\):
\(\lVert\bar t_{5}\rVert^{2}=229\). Во всех трёх случаях факторизации можно
получить вектор, более близкий (то есть более короткий), чем у алгоритма Бабая,
посредством квантовой оптимизации.

\subsection*{Гамильтониан задачи}

В основном тексте было описано, как построить гамильтониан задачи, отображая
двоичные переменные $x_i,\; i = 1,\dots,n$ на элементы Паули‑$Z$,

\begin{equation}
H_{c}
  = \bigl\lVert t - \sum_{i=1}^{n} \hat x_{i} d_{i} - b_{\mathrm{op}}\bigr\rVert^{2}
  = \sum_{j=1}^{n+1} \Bigl|\, t_{j} - \sum_{i=1}^{n} \hat x_{i} d_{i,j}
           - b^{\,j}_{\mathrm{op}} \Bigr|^{2}
\end{equation}

Мы кодируем плавающие переменные $x_i\in\{-1,0,1\}$ (определённые
промежуточными вычислениями алгоритма Бабая) одним кубитом. Квантовый оператор
$\hat x_i$ отображается в базис Паули‑$Z$ по следующим правилам:

\begin{equation}
\hat x_{i} =
\begin{cases}
\dfrac{I - \sigma^{\,i}_{z}}{2}, & \text{если } c_{i} \le \mu_{i},\\[6pt]
\dfrac{\sigma^{\,i}_{z} - I}{2}, & \text{если } c_{i} > \mu_{i}.
\end{cases}
\end{equation}

Если коэффициент был округлён вниз, т.е.\ $c_i\le\mu_i$, то его значение
увеличивается на 1 или остаётся неизменным; в этом случае плавающее значение
$x_i\in\{0,1\}$ соответствует собственным значениям оператора
$\dfrac{I-\sigma_z^{(i)}}{2}$ и наоборот. Тем самым информация об округлении
$c_i$ в алгоритме Бабая определяет кодирование $x_i$. Нетрудно видеть, что
более низкое энергетическое состояние гамильтониана даёт приближённое решение
ближайшего вектора в решётке $\Lambda$, поскольку гамильтониан соответствует
функции потерь.

Для случая с пятью кубитами гамильтониан можно записать как $H_{c5}=
\sum_{j=1}^{6} \hat h_j$, где
\begin{equation}
\begin{cases}
\hat h_{1} = (\, 6\hat x_{1} - 8\hat x_{2} + 2\hat x_{3}
              - 4\hat x_{4} - 4\hat x_{5} + 2\,)^{2},\\[2pt]
\hat h_{2} = (\, -4\hat x_{1} - 3\hat x_{2} + 11\hat x_{3}
              - 5\hat x_{4} - 3\hat x_{5} + 4\,)^{2},\\[2pt]
\hat h_{3} = (\, 6\hat x_{1} + 6\hat x_{2} + 3\hat x_{3}
              - 0\hat x_{4} - 3\hat x_{5} + 9\,)^{2},\\[2pt]
\hat h_{4} = (\, 4\hat x_{1} - 2\hat x_{2} + 0\hat x_{3}
              + 12\hat x_{4} + 4\hat x_{5} + 8\,)^{2},\\[2pt]
\hat h_{5} = (\, -2\hat x_{1} + 2\hat x_{2} - 6\hat x_{3}
              - 2\hat x_{4} + \hat x_{5}\,)^{2},\\[2pt]
\hat h_{6} = (\, -3\hat x_{1} + 5\hat x_{2} - 3\hat x_{3}
              + 4\hat x_{4} - 17\hat x_{5} + 8\,)^{2}.
\end{cases}
\end{equation}

Конкретный способ кодирования каждой переменной $x_i,\; i=1,\dots,5$
определяется по промежуточным данным алгоритма Бабая и приведён в таблице Г.1.

Соответственно, 5‑кубитный гамильтониан сводится к равнению Г.50. Кодирование
кубитов и гамильтониан для 3‑кубитного случая даны в таблице Г.2 равнении Г.51.
Кодирование кубитов и гамильтониан для 10‑кубитного случая даны в таблице Г.3 и
уравнении Г.52.
%––––– 5‑кубитный гамильтониан –––––
\begin{equation}
\begin{aligned}
H_{c5}=\, &781\,I
- 14\sigma_{z}^{1}
- 640\sigma_{z}^{2}
- 810\sigma_{z}^{3}
- 213\sigma_{z}^{4}
- 4.5\sigma_{z}^{5}
- 13.5\sigma_{z}^{1}\sigma_{z}^{2}
+ 3.5\sigma_{z}^{1}\sigma_{z}^{3} \\[2pt]
&{}+ 18\sigma_{z}^{1}\sigma_{z}^{4}
+ 17.5\sigma_{z}^{1}\sigma_{z}^{5}
- 29\sigma_{z}^{2}\sigma_{z}^{3}
+ 19.5\sigma_{z}^{2}\sigma_{z}^{4}
- 34\sigma_{z}^{2}\sigma_{z}^{5}
- 31.5\sigma_{z}^{3}\sigma_{z}^{4} \\[2pt]
&{}- 2.5\sigma_{z}^{3}\sigma_{z}^{5}
+ 4.5\sigma_{z}^{4}\sigma_{z}^{5}.
\end{aligned}
\end{equation}

%––––– 3‑кубитный гамильтониан –––––
\begin{equation}
H_{c3}=43.5\,I
- 4\,\sigma_{z}^{1}\sigma_{z}^{2}
+ 2.5\,\sigma_{z}^{1}\sigma_{z}^{3}
- 1.5\,\sigma_{z}^{1}
+ 3\,\sigma_{z}^{2}\sigma_{z}^{3}
- 3.5\,\sigma_{z}^{2}
- 4\,\sigma_{z}^{3}.
\end{equation}

%––––– 10‑кубитный гамильтониан (деление на 4 вынесено наружу) –––––
\begin{equation}
\begin{aligned}
H_{c10}= \frac14\Bigl(&
 708\,I
+ 22\sigma_{z}^{1}\sigma_{z}^{2}
+ 16\sigma_{z}^{1}\sigma_{z}^{3}
+  8\sigma_{z}^{1}\sigma_{z}^{4}
- 14\sigma_{z}^{1}\sigma_{z}^{5}
+  8\sigma_{z}^{1}\sigma_{z}^{6}
+  4\sigma_{z}^{1}\sigma_{z}^{7}
-  8\sigma_{z}^{1}\sigma_{z}^{8} \\[2pt]
&{}- 10\sigma_{z}^{1}\sigma_{z}^{9}
- 22\sigma_{z}^{1}\sigma_{z}^{10}
- 46\sigma_{z}^{1}
- 14\sigma_{z}^{2}\sigma_{z}^{3}
+ 20\sigma_{z}^{2}\sigma_{z}^{4}
+ 14\sigma_{z}^{2}\sigma_{z}^{5}
- 12\sigma_{z}^{2}\sigma_{z}^{6} \\[2pt]
&{}+ 20\sigma_{z}^{2}\sigma_{z}^{7}
- 24\sigma_{z}^{2}\sigma_{z}^{8}
- 28\sigma_{z}^{2}\sigma_{z}^{9}
+  2\sigma_{z}^{2}\sigma_{z}^{10}
- 16\sigma_{z}^{2}
- 18\sigma_{z}^{3}\sigma_{z}^{4}
+ 10\sigma_{z}^{3}\sigma_{z}^{5} \\[2pt]
&{}+ 36\sigma_{z}^{3}\sigma_{z}^{6}
+ 16\sigma_{z}^{3}\sigma_{z}^{8}
+  6\sigma_{z}^{3}\sigma_{z}^{9}
- 30\sigma_{z}^{3}\sigma_{z}^{10}
- 78\sigma_{z}^{3}
+ 28\sigma_{z}^{4}\sigma_{z}^{5}
- 26\sigma_{z}^{4}\sigma_{z}^{6} \\[2pt]
&{}+ 10\sigma_{z}^{4}\sigma_{z}^{8}
+ 16\sigma_{z}^{4}\sigma_{z}^{9}
-  4\sigma_{z}^{4}\sigma_{z}^{10}
- 72\sigma_{z}^{4}
+ 10\sigma_{z}^{5}\sigma_{z}^{6}
+ 24\sigma_{z}^{5}\sigma_{z}^{7}
+ 20\sigma_{z}^{5}\sigma_{z}^{8} \\[2pt]
&{}+ 12\sigma_{z}^{5}\sigma_{z}^{9}
-  8\sigma_{z}^{5}\sigma_{z}^{10}
- 116\sigma_{z}^{5}
-  5\sigma_{z}^{6}\sigma_{z}^{7}
+ 22\sigma_{z}^{6}\sigma_{z}^{8}
-  6\sigma_{z}^{6}\sigma_{z}^{9}
- 36\sigma_{z}^{6}\sigma_{z}^{10} \\[2pt]
&{}- 120\sigma_{z}^{6}
- 16\sigma_{z}^{7}\sigma_{z}^{8}
+ 16\sigma_{z}^{7}\sigma_{z}^{9}
+ 20\sigma_{z}^{7}\sigma_{z}^{10}
-  84\sigma_{z}^{7}
+ 34\sigma_{z}^{8}\sigma_{z}^{9} \\[2pt]
&{}- 42\sigma_{z}^{8}\sigma_{z}^{10}
- 36\sigma_{z}^{8}
+ 18\sigma_{z}^{9}\sigma_{z}^{10}
- 74\sigma_{z}^{9}
- 24\sigma_{z}^{10}
\Bigr).
\end{aligned}
\end{equation}

\begin{table}[h]
    \centering
    \caption{
        Кодирование кубитов для 5‑кубитного случая. Подскрипт~«$j$» убывает слева
        направо.
    }
\begin{tabular}{@{}lccccc@{}}
\hline\hline
\textbf{шаги} &
\(\;1\;(x_{5})\;\) &
\(2\;(x_{4})\) &
\(3\;(x_{3})\) &
\(4\;(x_{2})\) &
\(5\;(x_{1})\) \\ \hline
\(\mu_{j}\)      & $-8731.5607$ &  $3882.5019$ & $-1837.4760$ &  $-354.467$ & $-3092.4957$ \\
\(c_{j}\)        & $-8732$      &      $3883$  &     $-1837$  &     $-354$  &     $-3092$  \\
\(\mu_{j}-c_{j}\)&   $0.4393$   &   $-0.4981$  &   $-0.4760$  &  $-0.4669$  &   $-0.4957$  \\
кодирование     &  $(0,1)$     &   $(0,-1)$   &   $(0,-1)$   &   $(0,-1)$  &   $(0,-1)$   \\
\hline\hline
\end{tabular}
    \label{tab:tab3}
\end{table}

\begin{table}[h]
    \centering
    \caption{
        Кодирование кубитов для 3‑кубитного случая. Подскрипт~«$j$» убывает
        слева направо.
    }
\begin{tabular}{@{}lccc@{}}
\hline\hline
\textbf{шаги} &
\(\;1\;(x_{3})\;\) &
\(2\;(x_{2})\) &
\(3\;(x_{1})\) \\ \hline
\(\mu_{j}\)      & $ 33.5812$ & $-20.4974$ & $21.6667$ \\
\(c_{j}\)        & $     34$ &     $-20$  &     $ 22$ \\
\(\mu_{j}-c_{j}\)& $-0.4188$ & $-0.4974$ & $-0.3333$ \\
кодирование     & $(0,-1)$ & $(0,-1)$ & $(0,-1)$ \\
\hline\hline
\end{tabular}
    \label{tab:tab4}
\end{table}

\begin{table}[h]
    \centering
    \caption{
        Кодирование кубитов для 10‑кубитного случая. Подскрипт~«$j$» убывает
        слева направо.
    }
\resizebox{\textwidth}{!}{%
\begin{tabular}{@{}lcccccccccc@{}}
\hline\hline
\textbf{шаги} & \(1\;(x_{10})\) & \(2\;(x_{9})\) & \(3\;(x_{8})\) &
\(4\;(x_{7})\) & \(5\;(x_{6})\) & \(6\;(x_{5})\) &
\(7\;(x_{4})\) & \(8\;(x_{3})\) & \(9\;(x_{2})\) & \(10\;(x_{1})\) \\ \hline
\(\mu_{j}\)      & 21514.149 & -45688.541 & -29225.450 & -5953.325  & 29891.446 & 23868.721 & 42395.337 & -18221.276 & -29823.805 &  5952.889 \\
\(c_{j}\)        &     21514 &    -45689 &    -29225  &    -5953   &     29891 &     23869 &     42395 &    -18221  &    -29824  &      5953 \\
\(\mu_{j}-c_{j}\)&      0.149 &      0.459 &     -0.450 &     -0.325 &      0.446 &     -0.279 &      0.337 &     -0.276 &      0.195 &    -0.111 \\
кодирование     &   $(0,1)$  &   $(0,1)$  &  $(0,-1)$  &  $(0,-1)$  &   $(0,1)$  &  $(0,-1)$  &   $(0,1)$  &  $(0,-1)$  &   $(0,1)$  &  $(0,-1)$ \\
\hline\hline
\end{tabular}
}
    \label{tab:tab5}
\end{table}


\subsection*{Энергетический спектр и целевое состояние}

Численно просматриваем энергетический спектр гамильтониана задачи. Ниже
приводим лишь десять низших уровней энергии и~соответствующие квантовые
состояния; при наличии — также соответствующие sr‑пары. Следует различать
предел гладкости $B_{2}$ для~$|u-vN|$ и~предел $B_{1}$ для самой пары $(u,v)$.
В~5‑кубитном случае берём $B_{2}=p_{50}=229$, а~размерность соответствующей
системы линейных уравнений (обозначим eq‑dim) равна~51. Данные для 3‑ и
10‑кубитного случаев даны в таблице Г.4. Размерность системы $\sim 2n^{2}$ —
полиномиальна от $n$, поэтому ослабить предел $B_{2}$ разумно. С одной стороны,
при методе решета Шнорра размерность базы простых (то есть размерность решётки)
мала, и решение системы требует небольших ресурсов. С другой стороны, алгоритм
предъявляет высокие требования к качеству коротких векторов, что резко
увеличивает совокупную трудоёмкость. Умеренное ослабление $B_{2}$ снижает
требования к вектору, увеличивая число уравнений, но в целом повышая
эффективность алгоритма.

\begin{table}[h]
    \centering
    \caption{
Два предела гладкости для трёх случаев факторизации.
    }
\begin{tabular}{@{}lccccc@{}}
\hline\hline
\textbf{случай} & \textbf{B1‑dim} & \textbf{B1} &
\textbf{B2‑dim} & \textbf{B2} & \textbf{eq‑dim} \\ \hline
3 кубита  & 3  &  5 &  15 &   47 &  16 \\
5 кубит   & 5  & 11 &  50 &  229 &  51 \\
10 кубит  & 10 & 29 & 200 & 1223 & 201 \\
\hline\hline
\end{tabular}
    \label{tab:tab6}
\end{table}

В таблице Г.5 первый столбец — десять низших уровней энергии гамильтониана
из уравнения Г.50; второй — сами энергии, т.\,е.\ квадраты норм коротких
векторов. Третий столбец содержит собственные состояния. Строка 1 — основное
состояние, его энергия — 186. Длина соответствующего вектора минимальна, однако
sr‑пара не~получена: связь между коротким вектором и sr‑парой носят
вероятностный характер. Энергия четвёртого возбуждённого состояния равна 215,
и~$|u-vN|=12097706= 2\!\cdot\!41\!\cdot\!43\!\cdot\!47\!\cdot\!73$ гладко
относительно $B_{2}$; из состояния $(00111)$ получается sr‑пара. Седьмое
возбуждённое состояние $(00000)$ даёт энергию 229 — это оптимум алгоритма
Бабая. Таким образом, квантовый метод выдаёт более короткий вектор и sr‑пару,
что делает состояние $(00111)$ целевым.

\begin{table}[h]
    \centering
    \caption{
    Первые десять наименьших уровней энергии и соответствующие квантовые
    состояния. Четвёртое возбуждённое состояние порождает гладкую пару
    отношений, и соответствующее значение \( |u-vN| = 12097706 =
    2\cdot41\cdot43\cdot47\cdot73 \) является гладким относительно границы
    \(B_{2}\), что делает это состояние целевым для 5‑кубитного случая.
    }
\begin{tabular}{@{}ccccccc@{}}
\hline\hline
\textbf{уровень} & \textbf{энергия} & \textbf{состояние} &
\(u\) & \(v\) & \(|u-vN|\) & гладкое \\ \hline
0 & 186 & 0 1 0 1 1 0 & 21435888100 &   441 & $89{\cdot}199337$ & нет \\
1 & 189 & 0 1 1 1 0 0 &  3401399712 &    7 & $53{\cdot}3191$   & нет \\
2 & 193 & 1 1 1 0 0 0 & 1215290846 &   25 & $3^{2}{\cdot}370057$ & нет \\
3 & 198 & 1 0 0 0 1 1 &  776562633 &   16 &  512999           & нет \\
4 & 215 & 0 0 1 1 1 1 & 11789738455 & 243 & $2^{4}{\cdot}43{\cdot}47{\cdot}73$ & да \\
5 & 218 & 1 1 0 0 0 0 &  243045684  &    5 &  205949           & нет \\
6 & 222 & 1 1 1 1 0 0 & 4167418.11  & 8575 & 249693139         & нет \\
7 & 229 & 0 0 0 0 0 0 &   48620250 &    1 & $17{\cdot}3119$    & нет \\
8 & 230 & 1 0 0 0 0 0 & 194500845  &    4 & $41^{2}{\cdot}5657$ & нет \\
9 & 232 & 1 0 1 1 1 2 & 2.85312E+11 & 5880 & $37^{2}{\cdot}7124977$ & нет \\
\hline\hline
\end{tabular}
    \label{tab:tab7}
\end{table}

Выбирать в качестве целевого не обязательно основное, а~лишь достаточно низкое
состояние: QAOA трудно сходится к~глобальному минимуму, но при малой энергии
подготовленного состояния вероятность измерить нужное собственное состояние
высока. Эксперименты, описанные ниже, это подтверждают.

Аналогично приведены четыре низших собственных состояния для 3‑кубитного случая
(таблица Г.6). Sr‑пары получаются из первых трёх уровней, включая основное.
Второе возбуждённое состояние $(000)$ эквивалентно оптимуму Бабая; после
квантовой оптимизации получены ещё более короткие векторы длиной 33 и 35, также
дающие новые sr‑пары. Целевым выбираем основное состояние $(001)$, которое
готовится в эксперименте. Для 10‑кубитного случая (таблица Г.7) основное
состояние $(0100010010)$ ведёт к sr‑паре и становится целевым.

\begin{table}[h]
    \centering
    \caption{
    Четыре низших уровня энергии и соответствующие квантовые состояния
    (3‑кубитный случай).
    }
\begin{tabular}{@{}ccccccc@{}}
\hline\hline
\textbf{уровни} & \textbf{энергия} & \textbf{состояние} &
    \(u\) & \(v\) & \(|u-vN|\) & гладкое \\ \hline
    0 & 33 & 0 0 1 &  1800 & 1 & $7{\times}23$ & да \\
    1 & 35 & 1 1 0 &  1944 & 1 & $17$ & да \\
    2 & 36 & 0 0 0 &  2025 & 1 & $2^{6}$ & да \\
    3 & 42 & 1 0 0 &  3645 & 2 &  277 & да \\
\hline\hline
\end{tabular}
    \label{tab:tab8}
\end{table}

\begin{table}[h]
    \centering
    \caption{
    Первые десять наименьших уровней энергии и соответствующие квантовые
    состояния для 10‑кубитного случая. Основное состояние \((0100010010)\)
    порождает гладкую пару отношений, а соответствующее значение  $|u-vN| \;=\;
    2\cdot31\cdot97\cdot109\cdot163\cdot433$ гладко по границе \(B_{2}\), что
    делает данное состояние целевым для 10‑кубитного случая.
    }
    \resizebox{\textwidth}{!}{%
\begin{tabular}{@{}ccccccc@{}}
\hline\hline
\textbf{уровень} & \textbf{энергия} & \textbf{состояние} &
    \(u\) & \(v\) & \(|u-vN|\) & гладкое \\ \hline
0 & 51 & 0 1 0 0 0 1 0 0 1 0 &
  785989264048241 & 3 &
    $2{\times}31{\times}97{\times}109{\times}163{\times}433$ & да \\
1 & 57 & 0 1 0 0 0 0 0 0 1 0 &
  261933899831373 & 1 &
    $2^{3}{\times}29{\times}203014633$ & нет \\
2 & 60 & 0 0 0 0 0 0 0 0 0 0 &
  262049748526566 & 1 &
    $47{\times}139{\times}10523389$ & нет \\
3 & 60 & 0 0 0 1 0 1 0 0 0 0 &
  262123789565918 & 1 &
    $3803{\times}3754673$ & нет \\
4 & 61 & 0 1 0 0 0 0 1 1 0 0 &
  262027921960805 & 1 &
    $2^{4}{\times}457{\times}2243{\times}2861$ & нет \\
5 & 65 & 0 1 0 0 0 0 0 1 0 0 &
  7598979238585630 & 29 &
    $3^{2}{\times}211{\times}1531{\times}835897$ & нет \\
6 & 66 & 0 0 0 0 0 1 1 0 0 0 &
  4455399847833940 & 17 &
    $2^{3}{\times}24407{\times}11764801$ & нет \\
7 & 68 & 0 0 0 0 0 0 0 0 1 0 &
  261988302332823 & 1 &
    $2{\times}7^{2}{\times}22717{\times}22963$ & нет \\
8 & 70 & 0 0 0 0 0 1 0 0 0 0 &
  262012871275155 & 1 &
    $2{\times}1693{\times}9412891$ & нет \\
9 & 70 & 0 0 0 0 1 0 0 0 0 0 &
  262002304109546 & 1 &
    21304883317 & нет \\
\hline\hline
\end{tabular}
    }
    \label{tab:tab9}
\end{table}
