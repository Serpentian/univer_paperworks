\structure{РЕФЕРАТ}

Алгоритм Шора серьёзно поставил под вопрос безопасность информации, основанную
на криптосистемах с открытым ключом. Однако для взлома широко используемой
схемы RSA-2048 требуются миллионы физических кубитов, что значительно превышает
текущие технические возможности. Здесь мы сообщаем об универсальном квантовом
алгоритме факторизации целых чисел, объединяющем классическую редукцию базиса
решетки с квантовым алгоритмом приближённой оптимизации (QAOA). Количество
требуемых кубитов равно $O(\log N / \log\log N)$, что сублинейно относительно
битовой длины целого числа $N$, делая этот алгоритм самым экономичным по числу
кубитов алгоритмом факторизации на сегодняшний день. Мы экспериментально
демонстрируем алгоритм, факторизуя целые числа размером до 48 бит с помощью 10
сверхпроводящих кубитов, что является наибольшим целым числом, факторизованным
на квантовом устройстве. Мы оцениваем, что квантовая схема с 372 физическими
кубитами и глубиной в тысячи операций необходима для того, чтобы бросить вызов
RSA-2048 при помощи нашего алгоритма. Наше исследование демонстрирует
значительные перспективы для ускорения применения текущих шумных квантовых
компьютеров и прокладывает путь к факторизации больших целых чисел, имеющих
реальное криптографическое значение.
