\appendixsection{Постобработка: гладкие пары и линейные уравнения}

Ниже представлены другие гладкие соотношённые пары, полученные в примерах
факторизации чисел 1961, 48567227 и 261980999226229, как показано в следующих
списках. В первом столбце указаны порядковые номера гладких соотношённых пар
(sr-пар), а выражение $|u - vN|$ представлено в виде разложения по
соответствующему простому базису. В случае с 3 кубитами мы приводим 20
независимых гладких соотношённых пар, и соответствующая булева матрица состоит
из 20 векторов размерности 16. Следовательно, среди них обязательно найдётся
группа линейно зависимых векторов, то есть система линейных уравнений имеет как
минимум одно решение.

\subsection*{Случай с 3 кубитами}

В таблице Е.9 приведён список булевых векторов, соответствующих показателям
простого базиса, составленного из $u / (u - vN)$. Можно заметить, что четвёртый
вектор является нулевым, что само по себе представляет собой вектор линейной
зависимости. 10-й и 17-й векторы, а также 5-й и 16-й — это две группы линейно
зависимых векторов. Другие линейно зависимые векторы необходимо определить,
решая систему линейных уравнений. Каждая такая группа линейных зависимостей
будет соответствовать квадратичному сравнению вида $X^2 \equiv Y^2 \pmod{N}$. С
высокой вероятностью факторизация числа $N$ может быть получена с помощью этого
сравнения. Ниже мы приводим подробности факторизации $N = 1961$ с
использованием конкретных гладких соотношённых пар.

Согласно приведённому выше обсуждению, из указанных гладких соотношённых пар
можно найти решения системы линейных уравнений, например:

\textbf{Пример 1:} 4-я пара, где $u = 34 \cdot 5^2$, $v = 1$, $|u - vN| = 26$,
что даёт квадратичное сравнение: $(9 \cdot 5)^2 - 8^2 = N$. Тогда: $p = \gcd(45
+ 8, N) = 53, \quad q = \gcd(45 - 8, N) = 37$.

\textbf{Пример 2:} 9-я пара, где $u = 26 \cdot 5^2$, $v = 1$, $|u - vN| =
19^2$, $(8 \cdot 5)^2 + 19^2 = N$, и, согласно методу факторизации Гаусса, это
приведёт к паре множителей:
\begin{equation}
p = x^2 + y^2, \quad q = a^2 + b^2.
\end{equation}

Далее:
\begin{equation}
\left\{
\begin{aligned}
|ax - by| &= 40,\\
|bx + ay| &= 19,
\end{aligned}
\right.
\quad \text{или} \quad
\left\{
\begin{aligned}
|ax - by| &= 19,\\
|bx + ay| &= 40.
\end{aligned}
\right.
\end{equation}

Решая уравнения, получаем: $a = 1$, $b = 6$, $x = 2$, $y = 7$. Подставляя в
уравнение (Е.55), получаем $p = 53$, $q = 37$. Или: $a = 2$, $b = 7$, $x = 6$,
$y = -1$, тогда $p = 37$, $q = 53$.

\textbf{Пример 3:} Комбинация 10-й и 17-й пары даёт:
\begin{equation}
(2 \cdot 5^2 \cdot 2^2 \cdot 3^3)^2 \equiv (7 \cdot 11 \cdot 13)^2 \pmod{1961}.
\end{equation}
Отсюда:

\begin{equation}
p = \gcd(5400 + 1001, 1961) = 37,
\end{equation}
\begin{equation}
q = \gcd(5400 - 1001, 1961) = 53.
\end{equation}

\textbf{Пример 4:} Комбинация 5-й и 16-й пары:
\begin{equation}
2^5 \cdot 5^2 \cdot 3 \cdot 5^4 \equiv 2 \cdot 4^3 \cdot 3^3 \cdot 4^3 \pmod{1961},
\end{equation}
то есть:
\begin{equation}
(2^2 \cdot 5^3)^2 \equiv (4^3 \cdot 3)^2 \pmod{1961}.
\end{equation}

Следовательно:
\begin{equation}
p = \gcd(500 + 129, 1961) = 37,
\end{equation}
\begin{equation}
q = \gcd(500 - 129, 1961) = 53.
\end{equation}

Кроме того, простые множители также могут быть найдены с помощью решений
линейных уравнений для других соотношений, которые здесь не приводятся.

\begin{table}[H]
\centering
\caption{
    Булевы экспоненциальные векторы, соответствующие гладким соотношённым
    парам. Первый столбец представляет собой порядковый номер гладкой
    соотношённой пары $(u, v)$. Второй столбец — это знак, который указывает на
    положительность или отрицательность выражения $u / (u - vN)$. Столбцы с
    третьего по семнадцатый представляют булевы показатели для первых 15
    простых чисел базиса соответственно.
}
\begin{tabular}{c|c|*{15}{c}}
\hline
\hline
\textbf{sn} & \textbf{sign} & $p_1$ & $p_2$ & $p_3$ & $p_4$ & $p_5$ & $p_6$ & $p_7$ & $p_8$ & $p_9$ & $p_{10}$ & $p_{11}$ & $p_{12}$ & $p_{13}$ & $p_{14}$ & $p_{15}$ \\
\hline
1 & 1 & 1 & 1 & 0 & 0 & 0 & 0 & 1 & 0 & 0 & 0 & 0 & 0 & 0 & 0 & 0 \\
2 & 0 & 0 & 1 & 0 & 1 & 0 & 0 & 1 & 0 & 0 & 0 & 0 & 0 & 0 & 0 & 0 \\
3 & 1 & 1 & 1 & 0 & 0 & 0 & 0 & 0 & 0 & 0 & 0 & 0 & 1 & 0 & 1 & 0 \\
4 & 0 & 0 & 0 & 0 & 0 & 0 & 0 & 0 & 0 & 0 & 0 & 0 & 0 & 0 & 0 & 0 \\
5 & 1 & 1 & 1 & 0 & 0 & 0 & 0 & 0 & 0 & 0 & 0 & 0 & 0 & 0 & 1 & 0 \\
6 & 1 & 1 & 0 & 1 & 0 & 1 & 0 & 0 & 0 & 1 & 0 & 0 & 0 & 0 & 0 & 0 \\
7 & 0 & 1 & 0 & 1 & 0 & 0 & 0 & 0 & 0 & 0 & 0 & 0 & 0 & 0 & 0 & 0 \\
8 & 1 & 0 & 1 & 0 & 0 & 1 & 0 & 0 & 0 & 1 & 0 & 0 & 0 & 0 & 0 & 0 \\
9 & 1 & 0 & 1 & 0 & 0 & 0 & 1 & 0 & 0 & 0 & 0 & 0 & 0 & 0 & 0 & 0 \\
10 & 1 & 0 & 0 & 1 & 0 & 0 & 0 & 0 & 0 & 0 & 0 & 0 & 0 & 0 & 0 & 0 \\
11 & 1 & 1 & 1 & 0 & 1 & 0 & 1 & 0 & 0 & 0 & 0 & 0 & 0 & 0 & 0 & 1 \\
12 & 1 & 0 & 0 & 1 & 1 & 0 & 0 & 1 & 0 & 0 & 0 & 0 & 0 & 0 & 0 & 0 \\
13 & 1 & 0 & 1 & 0 & 0 & 0 & 0 & 0 & 1 & 0 & 0 & 0 & 0 & 0 & 0 & 0 \\
14 & 1 & 1 & 0 & 0 & 0 & 0 & 0 & 0 & 0 & 0 & 0 & 0 & 0 & 0 & 0 & 0 \\
15 & 1 & 0 & 1 & 0 & 0 & 0 & 0 & 1 & 0 & 1 & 0 & 0 & 0 & 0 & 1 & 0 \\
16 & 1 & 1 & 1 & 0 & 0 & 0 & 0 & 0 & 0 & 0 & 0 & 0 & 1 & 0 & 1 & 0 \\
17 & 0 & 0 & 0 & 0 & 0 & 0 & 0 & 0 & 0 & 0 & 0 & 0 & 0 & 0 & 0 & 0 \\
18 & 0 & 0 & 1 & 0 & 1 & 0 & 0 & 0 & 0 & 0 & 0 & 0 & 1 & 0 & 0 & 1 \\
19 & 0 & 1 & 0 & 1 & 0 & 0 & 1 & 0 & 0 & 0 & 0 & 0 & 0 & 0 & 0 & 0 \\
20 & 0 & 0 & 1 & 0 & 1 & 0 & 0 & 1 & 0 & 0 & 0 & 0 & 1 & 0 & 1 & 0 \\
\hline
\hline
\end{tabular}
\end{table}

\subsection*{Случай с 5 кубитами}

В случае с 5 кубитами мы приводим 55 независимых гладких соотношённых пар в
следующем списке. Соответствующая булева матрица содержит 55 векторов
размерности 51 (50 измерений для простого базиса и 1 измерение для знака).
Аналогично, среди них обязательно найдётся группа линейно зависимых векторов.

Из-за высокой размерности векторов, соответствующих случаю с 5 кубитами, мы
приводим только одно решение системы линейных уравнений:

\begin{equation}
\begin{aligned}
x = (&0, 0, 0, 0, 0, 0, 0, 0, 1, 0, 0, 1, 1, 0, 0, 0, 1, 0, 0, \\
&1, 1, 0, 0, 1, 0, 0, 0, 1, 0, 1, 0, 0, 0, 0, 0, 0, 0, \\
&0, 0, 0, 0, 0, 0, 1, 1, 1, 0, 0, 0, 0).
\end{aligned}
\end{equation}

Соответствующее решение для квадратичного сравнения:
\begin{equation}
\begin{aligned}
X &= 639232456435359657331994419097900390625, \\
Y &= 12136572734325633629343926054845304.
\end{aligned}
\end{equation}

Нетрудно проверить, что данное решение удовлетворяет уравнению:

\begin{equation}
X^2 \equiv Y^2 \pmod{N}
\end{equation}

Кроме того, имеем:

\begin{equation}
\begin{aligned}
p &= \gcd(X + Y, N) \\
  &= \gcd(639232456435359657331994419097900390625 + \\
  &\quad \ 12136572734325633629343926054845304, \,48567227) \\
  &= 7919, \\
q &= \gcd(X - Y, N) \\
  &= \gcd(639232456435359657331994419097900390625 - \\
  &\quad \ 12136572734325633629343926054845304, \,48567227) \\
  &= 6133.
\end{aligned}
\end{equation}

В результате мы получаем разложение числа на множители:
$N = 48567227 = 7919 \times 6133$.

\subsection*{Случай с 10 кубитами}

В случае с 10 кубитами мы приводим 221 независимую гладкую пару в
Дополнительных материалах. Соответствующая булева матрица содержит 221 вектор
размерности 201 (200 измерений для простого базиса и 1 измерение для знака). Мы
приводим одно решение системы линейных уравнений:

\begin{equation}
\begin{aligned}
x = (&1, 0, 0, 0, 0, 0, 0, 0, 0, 0, 0, 0, 0, 0, 0, 0, 0, 0, 0, 0, 0, 0, 0, 0, 0, \\
     &0, 0, 0, 0, 0, 0, 0, 0, 0, 0, 0, 0, 0, 0, 0, 0, 0, 0, 0, 0, 0, 0, 0, 1, \\
     &1, 1, 0, 0, 0, 1, 0, 0, 0, 0, 1, 0, 0, 0, 0, 1, 1, 0, 0, 0, 0, 1, 0, 1, \\
     &1, 0, 1, 1, 0, 0, 1, 0, 0, 0, 1, 1, 1, 0, 0, 0, 1, 0, 1, 1, 1, 0, 1, 0, \\
     &1, 0, 0, 1, 0, 0, 0, 0, 1, 1, 1, 1, 0, 1, 1, 0, 1, 0, 1, 0, 0, 1, 0, 0, \\
     &1, 1, 0, 0, 0, 1, 1, 0, 0, 0, 0, 1, 0, 0, 1, 1, 1, 0, 1, 0, 0, 0, 0, 1, \\
     &0, 0, 1, 0, 1, 1, 1, 1, 0, 1, 0, 0, 0, 0, 1, 0, 1, 0, 0).
\end{aligned}
\end{equation}

Соответствующее решение для квадратичного сравнения представлено в уравнении:
\begin{equation}
\begin{aligned}
X = &75695763106501556705305764502754936587819598184067351577 \\
    &43303250785633285956794230102519018501335269995854539435 \\
    &01610097294132617819824833811012145914436530877754829641 \\
    &771815233949422940291369528064063220752769498277701462410 \\
    &831032646021654929699823769255863972607946356507092437520 \\
    &628983005772258943590442184281264255447072655152539925075 \\
    &228232905527827902441430959155400833286647355489098701203 \\
    &6652660975430964340161981191012698547957753245615400642156 \\
    &6009521484375000000000000000000000000000000000000000000, \\ \\
Y = &89703025676439146996343189305085999464349858249769290536 \\
    &39263868715312151470596839682183291916841767330486102677 \\
    &2500921420533610666859894934306516222448335298649730188 \\
    &139419712140836303477738769684013638969809410021181628 \\
    &148458466644646455709975881417989619205326153001893986 \\
    &178123643916398321728850997506608105566537622917582126 \\
    &731375145833485980298044011134125822403913885046671262 \\
    &19915873471668113404162973340975659170623801.
\end{aligned}
\end{equation}

Нетрудно проверить, что данное решение удовлетворяет следующему уравнению:
\begin{equation}
X^2 \equiv Y^2 \pmod{N}
\end{equation}

Кроме того, имеем:
\begin{equation}
\begin{aligned}
p &= \gcd(X + Y, N) = 15538213, \\
q &= \gcd(X - Y, N) = 16860433.
\end{aligned}
\end{equation}

В результате мы получаем разложение на множители:
\begin{equation}
N = 261980999226229 = 15538213 \times 16860433.
\end{equation}

