\appendixsection{Основы теории решёток}

В последние годы решётки служат алгоритмическим инструментом для решения
широкого круга задач в информатике, математике и криптографии, особенно в
квантовоустойчивых криптографических протоколах. Ниже приведены базовые понятия
и известные алгоритмы, тесно связанные с нашей работой.

\subsection*{Базовые понятия}

Пусть $\lVert\cdot\rVert$ — евклидова норма векторов из $R^{m}$. Векторы
отмечаем полужирным (в переводе этого нет), матрицы пишем построчно; элементы
матрицы $M$ обозначаем $m_{i,j}$. Верхний индекс~$T$ — транспонирование.

\begin{itemize}
\item \textbf{Решётка.} Пусть $b_{1},\ldots,b_{n}\in R^{m}$ — линейно
        независимые столбцы, тогда множество всех линейных комбинаций их
        целочисленных коэффициентов — решётка, определяемая как
        \begin{equation} \Lambda(B) =\{\,Bx \mid x\in Z^{n}\,\} =\{\,b =
            x_{1}b_{1}+\dots+x_{n}b_{n} \mid x_{1},\dots,x_{n}\in Z\,\}.
        \end{equation} где $B=[b_{1},\ldots,b_{n}]\in R^{m\times n}$ — матрица
        базиса, которая также может быть использована чтобы представлять
        решетку для простоты. $\{b_1, \ldots, b_n\}$ - группа базиса решетки.
        Размерность решётки $n$. Её детерминант $\det\Lambda=(\det
        B^{T}B)^{1/2}$, здесь $B^T$ - транспонированная матрица $B$. При
        квадратной $B$ имеем $\det\Lambda=\det B$. Детерминант представляет
        объём решётки в геометрическом представлении, обозначается как
        $\operatorname{vol}(\Lambda)$. Длина точки $b\in R^{m}$ определяется
        как $\lVert b\rVert=(b^{T}b)^{1/2}$.

\item \textbf{Последовательные минимумы.} Для $n$‑мерной решётки $\Lambda$
        положительные числа $\lambda_{1}(\Lambda)\le\lambda_{2}(\Lambda)
        \le\dots\le\lambda_{n}(\Lambda)$ называются последовательными
        минимумами, где $\lambda_{k}(\Lambda)$ — наименьший радиус шара с
        центром в нуле, содержащего $k$ линейно независимых векторов из
        $\Lambda$. Обозначим $\lambda_{1}=\lambda_{1}(\Lambda)$ как длину
        кратчайшего ненулевого вектора из $\Lambda$.

\item \textbf{Постоянная Эрмита.} Эрмитовым инвариантом решётки $\Lambda$
        называется
        \begin{equation}
          \gamma(\Lambda)
          = \frac{\lambda_{1}^{2}(\Lambda)}{\operatorname{vol}(\Lambda)^{2/n}}
          = \frac{\lambda_{1}^{2}(\Lambda)}{\det(\Lambda)^{2/n}}.
        \end{equation}

        Постоянная Эрмита $\gamma_{n}$ — максимальное значение $\gamma(\Lambda)$
        для всех $n$‑мерных решёток, или минимальное константное $\gamma$,
        удовлетворяющее $\lambda_{1}(\Lambda)^{2}\le\gamma(\det\Lambda)^{2/n}$
        для всех решёток размености $n$ соответственно.

\item \textbf{QR‑разложение.} У решёточной матрицы базиса $B$ есть единственное
        разложение $B=QR\in R^{m\times n}$, где $Q\in R^{m\times n}$,
        $R=[r_{i,j}]_{1\le i,j\le n} \in R^{n\times n}$, здесь $Q\in R^{m\times
        n}$ — изомет­рическая (столбцы ортогональны и единичной длины), а $R
        \in R^{n\times n}$ — верхнетреугольная матрица с положительными
        диагональными элементами $r_{i,i}$. Коэффициенты Грама–Шмидта
        $\mu_{j,i}=r_{i,j}/r_{i,i}$ легко вычисляются из QR‑разложения. Для
        целочисленной $B$ коэффициенты $\mu_{j,i}$ обычно рациональны.

\item \textbf{Задача кратчайшего вектора (SVP).}
      Дана группа базиса $B$ решётки $\Lambda$.

      \begin{itemize}
          \item Задача кратчайшего вектора (SVP): Требуется найти вектор
              $v\in\Lambda$, такой что $\lVert v\rVert=\lambda_{1}(\Lambda)$.
          \item Приближённая задача кратчайшего вектора ($\alpha$‑SVP): Найти
              ненулевой вектор $v\in\Lambda$, удовлетворяющий
              $\lVert v\rVert\le\alpha\,\lambda_{1}(\Lambda)$.
          \item Эрмитова задача кратчайшего вектора ($r$‑Hermite SVP):
              Найти ненулевой вектор $v\in\Lambda$, такой что
              $\lVert v\rVert \;\le\; r\,\det(\Lambda)^{1/n}$.
      \end{itemize}

      Параметр $\alpha\ge1$ в $\alpha$‑SVP называется фактором аппроксимации.
      Обычнро задача упрощается при увеличении $\alpha$. При $\alpha=1$ задачи
      $\alpha$‑SVP и SVP совпадают. Истинное значение $\lambda_{1}$ в
      $\alpha$‑SVP трудно вычислить из-за сложности SVP, поэтому решение
      $\alpha$‑SVP не всегда легко проверить. Задача $r$‑Hermite SVP является
      вычислимой (относительно просто вычислимой), она определяется величиной
      $\det(\Lambda)^{1/n}$ вместо $\Lambda_1$. В результате, мы можем лешко
      проверить решение, но не можем сравнить его с кратчайшим вектором.

\item \textbf{Задача ближайшего вектора (CVP)}
      Дана группа базиса $B$ решётки $\Lambda$ и целевой вектор
      $t\in\operatorname{span}(B)$.

      \begin{itemize}
          \item Задача ближайшего вектора (CVP): Найти вектор $v\in\Lambda$,
              такой что расстояние $\lVert v - t\rVert$ может быть минимизировано,
              т.е. $\lVert v - t\rVert = \operatorname{dist}(\Lambda,t)$.
          \item $\alpha$-приближенная задача ближайшего вектора ($\alpha$‑CVP):
              Найти вектор $v\in\Lambda$, такой что расстояние
              $\lVert v - t\rVert \le \alpha \times \operatorname{dist}(\Lambda,t)$
          \item $r$-приближенная задача ближайшего вектора (($r$‑AbsCVP):  Найти
              $v\in\Lambda$ такое, что $\lVert v - t\rVert \le r$.
      \end{itemize}

      Определения аналогичны случаям для SVP; параметр $\alpha\ge1$ в
      $\alpha$‑CVP играет ту же роль, что и в $\alpha$‑SVP. В $r$‑AbsCVP
      параметр $r$ может быть любым разумным значением, соизмеримым с
      $\operatorname{dist}(\Lambda,t)$, например $\det(\Lambda)^{1/n}$ в
      $r$‑Hermite SVP.
\end{itemize}

\subsection*{Алгоритм LLL}
\begin{algorithm}[htp!]
    \SetAlgoLined

    \KwData{базис решётки $\text{b}_1, \dots, \text{b}_n \in \text{Z}^m$, параметр $\delta$}
    \KwResult{$\delta$-LLL-редуцированный базис}

    \textbf{Шаг 1:} Орторгонализация Грама-Шмидта \\
    Выполнить орторгонализацию Грама-Шмидта для базиса $\text{b}_1, \dots, \text{b}_n$, обозначим результат как $\tilde{\text{b}}_1, \dots, \tilde{\text{b}}_n \in \text{R}^m$

    \textbf{Шаг 2:} Редукция

    \For{$i = 2$; $i < n$; $i = i + 1$}{
        \For{$j = i-1$; $i > 1$; $ i = i - 1$}{
            $c_{i,j} = \left\lfloor \dfrac{\langle \text{b}_i, \tilde{\text{b}}_j \rangle}{\langle \tilde{\text{b}}_j, \tilde{\text{b}}_j \rangle} \right\rceil$; \\
            $\text{b}_i \gets \text{b}_i - c_{i,j} \cdot \text{b}_j$;
        }
    }

    \textbf{Шаг 3:} Обмен

    \If{$\exists \, i$ такое, что $\delta \cdot \| \tilde{\text{b}}_i \|^2 > \| \mu_{i+1,i} \tilde{\text{b}}_i + \tilde{\text{b}}_{i+1} \|^2$}{
        $\text{b}_i \leftrightarrow \text{b}_{i+1}$; \\
        перейти к шагу 1;
    }

    \textbf{Шаг 4:} Вывод базиса $\text{b}_1, \dots, \text{b}_n$

    \caption{Алгоритм LLL-редукции}
    \label{alg:lll}
\end{algorithm}

Алгоритм LLL — один из самых известных алгоритмов редукции решётки; он был
предложен А. К. Ленстрой, Х. В. Ленстрой (мл.) и Л. Ловашем в 1982 г.
\cite{cite_35}. Для $n$‑мерной решётки этот алгоритм позволяет решать $\alpha
\;=\; \left(\frac{2}{\sqrt{3}}\right)^{n}$ за полиномиальное время. Ниже
приведены связанные понятия и алгоритмы.

\newcounter{tmp-enum-lll}

\begin{itemize}
    \item \textbf{LLL базиc}: Базис $B = QR$ называется LLL‑редукции или
        LLL‑базисом при параметре редукции $\delta\in(1/4,1]$, если выполняются
        условия:
        \begin{enumerate}
            \item $\frac{\lvert r_{i,j}\rvert}{r_{i,i}} \;\le\; \frac12, \quad \text{for all } j > i;$
            \item $\delta\, r_{i,i}^{2}\;\le\;r_{i,i+1}^{2} + r_{i+1,i+1}^{2},\quad \text{for } i = 1,\ldots,n-1.$
            \setcounter{tmp-enum-lll}{\value{enumi}}
        \end{enumerate}

        Очевидно, LLL‑базис также удовлетворяет
        \( r_{i,i}^{2} \le \alpha\, r_{i+1,i+1}^{2} \),
        для \(\alpha = \frac{1}{(\delta - 1 / 4)}\).

        Параметры, рассматриваемые в оригинальной литературе по алгоритму LLL,
        равны $\delta = \tfrac34$, $\alpha = 2$. Известный результат о LLL‑базисе
        показывает, что для любого $\delta < 1$ LLL‑базис может быть получен за
        полиномиальное время и хорошо аппроксимирует последовательные минимумы\:

        \begin{enumerate}
            \setcounter{enumi}{\value{tmp-enum-lll}}
            \item $\alpha^{-\,i+1} \;\le\; \lVert b_{i}\rVert^{2}\,\lambda_{i}^{-2}
                \;\le\; \alpha^{\,n-1}, \quad \text{for } i = 1,\ldots,n;$
            \item $\lVert b_{1}\rVert^{2} \;\le\; \alpha^{\frac{n-1}{2}}\,
                \bigl(\det\Lambda\bigr)^{2/n}.$
        \end{enumerate}

    \item \textbf{Алгоритм LLL}: Для заданного набора базиса
        $B = [b_{1},\ldots,b_{n}] \in Z^{m\times n}$
        алгоритм может привести его к LLL‑редуцированному виду
        или преобразовать в LLL‑базис. Алгоритм состоит из трёх
        основных шагов: ортогонализация Грама–Шмидта, редукции и
        обмен. Конкретные шаги приведены в Алгоритме \ref{alg:lll}.
\end{itemize}

\subsection*{Алгоритм ближайшей плоскости Бабая}

Алгоритм ближайшей плоскости Бабая \cite{cite_32} (далее — алгоритм Бабая)
применяется для решения CVP. Для $n$‑мерной решётки он может получать фактор
аппроксимации $\alpha \;=\; 2\!\left(\frac{2}{\sqrt{3}}\right)^{n}$ для
$\alpha$‑CVP. Алгоритм состоит из двух этапов, первый из которых заключается в
редукции решетки с помощью алгоритма LLL. Второй является процедурой уменьшения
размера, который в основном вычисляет линейную комбинацию целочисленных
коэффициентов, ближайших к целевому вектору $t$ в LLL-базисе. Этот шаг по сути
совпадает со вторым шагом в LLL-редукции. Подробные действия приведены в
Алгоритме \ref{alg:babai}.

\begin{algorithm}[htp!]
    \SetAlgoLined

    \KwData{базис решётки $\text{b}_1, \dots, \text{b}_n \in \text{Z}^m$, параметр $\delta = 3/4$, целевой вектор $\text{t} \in \text{Z}^m$}
    \KwResult{вектор $\text{x} \in \Lambda(B)$, такой что $\|\text{x} - \text{t}\| \leq 2^{n/2} \, \text{dist}(\text{t}, \Lambda(B))$}

    \textbf{Шаг 1:} LLL-редукция \\
    Применить LLL-редукцию к базису $B$ с параметром $\delta$ \\
    Обозначим результат: $\tilde{\text{b}}_1, \dots, \tilde{\text{b}}_n \in \text{R}^m$

    \textbf{Шаг 2:} Уменьшение размера \\
    $\text{b} \gets \text{t}$

    \For{$j = n$; $j > 1$ $j = j-1$}{
        $c_j = \left\lfloor \dfrac{\langle \text{b}, \tilde{\text{b}}_j \rangle}{\langle \tilde{\text{b}}_j, \tilde{\text{b}}_j \rangle} \right\rceil$ \\
        $\text{b} \gets \text{b} - c_j \cdot \text{b}_j$
    }

    \textbf{Шаг 3:} Вернуть $\text{t} - \text{b}$

    \caption{Алгоритм Бабая}
    \label{alg:babai}
\end{algorithm}
