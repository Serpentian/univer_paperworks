\conclusion

Проблема факторизации целых чисел является краеугольным камнем безопасности
широко применяемой сегодня криптографии с открытым ключом RSA. В этой работе мы
предложили общий квантовый алгоритм факторизации целых чисел, основанный на
классическом методе редукции базиса решётки. Для факторизации $m$‑битного
целого числа $N$ алгоритму требуется $O(m/\log m)$ кубитов, что составляет
сублинейную величину относительно битовой длины $N$. Этот квантовый алгоритм
факторизации использует меньше всего кубитов по сравнению с предыдущими
методами, включая алгоритм Шора. Мы продемонстрировали принцип факторизации на
сверхпроводящем квантовом процессоре. 48‑битное число $261980999226229$,
факторизованное в нашей работе, является к настоящему времени наибольшим целым
числом, разложенным универсальным методом на реальной квантовой системе. Мы
проанализировали квантовые ресурсы, необходимые для факторизации RSA-2048 в
квантовых системах с тремя типовыми топологиями. Обнаружено, что квантовая
схема с 372 физическими кубитами и глубиной в несколько тысяч операций
необходима, чтобы бросить вызов RSA‑2048 даже в самой простой линейной цепочке.
Такой масштаб квантовых ресурсов, вероятно, будет достигнут на устройствах NISQ
в ближайшем будущем. Следует отметить, что квантовое ускорение алгоритма
остаётся неясным из‑за неоднозначной сходимости QAOA. Тем не менее идея
оптимизации процедуры «уменьшения размера» в алгоритме Бабаи при помощи QAOA
может использоваться как подпрограмма в широком классе распространённых
алгоритмов редукции базиса решётки и, далее, способствовать анализу
криптографических задач, устойчивых к квантовому взлому, основанных на
решётках.
