\structure{ВВЕДЕНИЕ}

Квантовые вычисления вступили в эпоху шумных квантовых устройств промежуточного
масштаба (NISQ) \cite{cite_1, cite_2}. Важной задачей эпохи NISQ является
демонстрация того, что устройства NISQ могут превзойти классические компьютеры
при решении задач с практической значимостью, то есть достижение практического
квантового преимущества. Алгоритмы, требующие минимальных ресурсов и
использующие ограниченное число доступных кубитов и глубину схем для решения
задач, сложных для классических вычислений, имеют особую важность. Вариационные
квантовые алгоритмы, использующие гибридную схему вычислений
«классика+квантовые вычисления», обладают значительным потенциалом для
получения значимого квантового преимущества в эпоху NISQ \cite{cite_3, cite_4,
cite_2, cite_5, cite_6}. Одним из таких алгоритмов является квантовый алгоритм
приближённой оптимизации (QAOA) \cite{cite_5}, первоначально предложенный для
решения задач на собственные значения, который впоследствии широко применялся в
различных областях, таких как химическое моделирование \cite{cite_7, cite_8},
машинное обучение \cite{cite_9}, а также инженерные приложения \cite{cite_10,
cite_11}.

Факторизация целых чисел является одной из важнейших основ современной
информационной безопасности \cite{cite_12}. Экспоненциальное ускорение
факторизации алгоритмом Шора \cite{cite_13} является выдающимся примером
превосходства квантовых вычислений. Однако выполнение алгоритма Шора на
отказоустойчивом квантовом компьютере требует значительных ресурсов
\cite{cite_14, cite_15}. На сегодняшний день наибольшее целое число,
факторизованное алгоритмом Шора на существующих квантовых системах, это число
21 \cite{cite_16, cite_17, cite_18}. Альтернативно, факторизация целых чисел
может быть сведена к задаче оптимизации, решаемой посредством адиабатических
квантовых вычислений (AQC) \cite{cite_19, cite_20, cite_21, cite_22} или QAOA
\cite{cite_23}. Более крупные числа были факторизованы этими методами на
различных физических системах \cite{cite_24, cite_25, cite_26, cite_27}.
Максимальные числа, факторизованные на данный момент, включают 291311 (19 бит)
в системе NMR \cite{cite_26}, 249919 (18 бит) на квантовом отжигателе D-Wave
\cite{cite_25}, 1099551473989 (41 бит) на сверхпроводящем устройстве
\cite{cite_27}. Однако следует отметить, что некоторые из факторизованных чисел
были специально подобраны с особыми структурами \cite{cite_28}, поэтому
наибольшее число, факторизованное универсальным методом на реальной физической
системе, на сегодняшний день составляет 249919 (18 бит).

В данной работе мы предлагаем универсальный квантовый алгоритм факторизации
целых чисел, требующий лишь сублинейные квантовые ресурсы. Алгоритм основан на
классическом алгоритме Шнорра \cite{cite_29, cite_30}, использующем редукцию
базиса решётки для факторизации целых чисел. Мы используем QAOA для оптимизации
наиболее трудоёмкой части алгоритма Шнорра, что ускоряет общее время
факторизации. Для целого числа $N$, имеющего $m$ бит, количество требуемых
кубитов в нашем алгоритме составляет $O(m / \log m)$, что сублинейно
относительно битовой длины числа $N$. Это делает наш алгоритм наиболее
экономным по числу кубитов по сравнению с существующими алгоритмами, включая
алгоритм Шора. С использованием данного алгоритма нами успешно факторизованы
числа 1961 (11 бит), 48567227 (26 бит) и 261980999226229 (48 бит) с
использованием, соответственно, 3, 5 и 10 кубитов на сверхпроводящем квантовом
процессоре. Число в 48 бит (261980999226229) также является наибольшим целым
числом, факторизованным универсальным методом на реальном квантовом устройстве.
Далее мы оцениваем квантовые ресурсы, необходимые для факторизации RSA-2048.
Согласно нашим расчётам, квантовая схема с 372 физическими кубитами и глубиной
порядка тысяч операций необходима для факторизации RSA-2048 даже в самой
простой одномерной системе. Подобный масштаб квантовых ресурсов, вероятно,
станет достижимым на устройствах NISQ в ближайшем будущем.

