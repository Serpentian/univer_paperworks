\structure{ВВЕДЕНИЕ}

Эта глава будет посвящена проблеме определения безопасности для симметричной
криптографии и соответствующих примитивов, таких как блочные шифры и генераторы
потока ключей, а также вопросу, как рассуждать о безопасности различных
конструкций, рассмотренных в Главе~1.

Общий подход к анализу криптографических конструкций заключается в том, чтобы
построить редукцию, которая использует противника против одной конструкции для
выполнения другой задачи. Затем проводится анализ редукции, чтобы определить её
вероятность успеха в этой задаче относительно вероятности успеха исходного
противника.

Это может выглядеть как замена одной неизвестной величины другой неизвестной
величиной. Но это поверхностное впечатление. Основная проблема заключается в
том, что, хотя криптографические конструкции могут быть довольно сложными,
контекст, в котором требуется их анализировать, будет ещё более сложным. Цель
состоит в том, чтобы свести анализ сложных криптографических конструкций в
сложных контекстах к анализу более простых криптографических конструкций или
даже естественных математических задач в более простых контекстах. В конечном
итоге можно использовать методы, набросанные в Главах~1--5, чтобы оценить,
насколько трудно атаковать примитивы, и посредством редукций установить оценку
сложности атаки криптографической конструкции.

Основная задача этой главы — проанализировать широкий спектр целей
безопасности, что является удивительно сложной темой. Затем обсуждаются
конструкции и базовые примитивы. Одно важное общее утверждение позволяет
рассматривать конфиденциальность и целостность отдельно.

Вторая задача — дать вводное обсуждение каналов. Это в некотором смысле
изначальная криптографическая тема: как двум сторонам защитить разговор? К
этому вопросу будет возвращение позже.

Хеш-функции были впервые представлены в Разделе~4.2. Строго говоря, хеш-функции
не связаны с шифрованием и дешифрованием, но данный концепт обсуждается в этой
главе, поскольку проектирование распространённых хеш-функций в некотором смысле
связано с проектированием симметричных примитивов. Также рассматриваются
идеальные модели для хеш-функций и других примитивов.

\textbf{О языке}. Современный стиль в криптографии состоит в том, чтобы точно
определять, что представляет собой противник для схемы и как измерять,
насколько хорошо этот противник действует. Таким образом, точно определяется
лишь понятие уровня небезопасности, и доказываются соотношения между уровнями
небезопасности.

Можно было бы определить уровни безопасности как отрицание уровня
небезопасности и формулировать теоремы в терминах уровней безопасности. Однако
поскольку фактическое доказательство связывает именно уровни небезопасности,
такой подход быстро становится запутанным и неудобным.

Вместо этого принято соглашение, что утверждение о том, что нечто является
безопасным, носит неформальный характер, что уместно, поскольку почти все
положительные утверждения о уровнях безопасности являются предположениями.
Соотношения между уровнями небезопасности затем превращаются в противоположные
неформальные утверждения о неформальных заявлениях безопасности. Читателю
предлагается сравнить неформальные утверждения в ранних главах с
соответствующими теоремами в более поздних главах.
