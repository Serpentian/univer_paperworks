\conclusion

Проведенное исследование позволило систематизировать и углубить понимание
безопасности симметричных криптосистем. Основные достижения работы включают:

\begin{enumerate}
\item \textbf{Формализацию понятий безопасности}. Были строго определены и
сопоставлены различные подходы к конфиденциальности (семантическая
безопасность, неотличимость, real-or-random), доказана их эквивалентность. Это
предоставляет гибкий инструментарий для анализа криптосистем.

\item \textbf{Анализ целостности}. Показано, что целостность шифртекста
является более сильным и полезным свойством, чем целостность открытого текста,
особенно в контексте доказательств безопасности при атаках с выбранным
шифртекстом.

\item \textbf{Исследование криптографических примитивов}. Изучены свойства
поточных шифров, PRF и PRP, установлены связи между ними. Лемма о переключении
между PRP и PRF позволяет использовать блочные шифры в конструкциях, требующих
псевдослучайных функций.

\item \textbf{Практические конструкции}. Режимы CTR и CBC проанализированы с
точки зрения безопасности, получены оценки преимущества противника в
зависимости от параметров системы.

\item \textbf{Методологический вклад}. Работа демонстрирует мощь формальных
методов в криптографии, включая редукционные доказательства, гибридные
аргументы и симуляционные техники.
\end{enumerate}

Проведенная работа закладывает фундамент для дальнейшего изучения и разработки
безопасных симметричных криптосистем, сочетая теоретическую строгость с
практической применимостью.
