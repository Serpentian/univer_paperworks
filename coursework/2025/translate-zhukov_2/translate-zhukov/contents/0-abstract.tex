\structure{РЕФЕРАТ}

В данной курсовой работе исследуются основы симметричных криптосистем с
ассоциированными данными и формальные определения безопасности. Работа
начинается с расширения функциональности симметричных криптосистем путем
введения ассоциированных данных, которые кодируют контекст появления
шифртекста. Это позволяет связывать сообщение с его контекстом без включения
последнего в само сообщение.

Основное внимание уделяется трем ключевым аспектам безопасности:
конфиденциальности, целостности и аутентификации. Подробно анализируются
различные подходы к определению конфиденциальности, включая семантическую
безопасность, неотличимость (left-or-right безопасность) и безопасность
real-or-random. Доказывается эквивалентность этих понятий, что позволяет
выбирать наиболее удобное из них для анализа конкретных криптосистем.

Рассматриваются атаки с выбранным открытым текстом (CPA) и выбранным
шифртекстом (CCA), а также формализуются соответствующие игры безопасности.
Особое внимание уделяется целостности шифртекста и открытого текста,
демонстрируется их неравносильность и важность целостности шифртекста для
доказательств безопасности.

В работе также представлены базовые криптографические примитивы, такие как
поточные шифры, псевдослучайные функции (PRF) и псевдослучайные подстановки
(PRP). Исследуются их свойства и взаимосвязи, включая лемму о переключении
между PRP и PRF. Приводятся практические конструкции, такие как режим счётчика
(CTR) и режим сцепления блоков (CBC), и анализируется их безопасность.
