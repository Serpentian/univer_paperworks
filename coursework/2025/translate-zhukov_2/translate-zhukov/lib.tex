\captionsetup[figure]{
    width=1.0\linewidth,
    justification=justified,
    font=small,
    name=Рисунок,
    labelsep=endash,
    position=below,
}

\captionsetup[table]{
    name=Таблица,
    labelsep=endash, % Тире
    position=above,
    justification=justified,
    singlelinecheck=false,
    font=small,
}

\usepackage{amssymb}

\usepackage[most]{tcolorbox}
\usepackage{enumitem}
\usepackage{amsmath,amssymb}

% стиль бокса, похожий на книгу
\tcbset{
  experiment/.style={
    colback=gray!5,
    colframe=gray!40,
    boxrule=0.4pt,
    arc=1pt,
    left=6pt,right=6pt,top=6pt,bottom=6pt,
    enhanced,
  }
}

\usepackage[most]{tcolorbox}
\usepackage{amsmath,amssymb}
\usepackage{xcolor}
\usepackage{enumitem}
\usepackage{varwidth}

\definecolor{defblue}{RGB}{92,141,178}

\newtcbtheorem[auto counter, number within=section]{mydefinition}{Определение}{%
  enhanced,
  colback=white,
  colframe=defblue,
  boxrule=0pt,
  borderline west={1pt}{0pt}{defblue},
  fonttitle=\bfseries,
  coltitle=black,
  attach boxed title to top left={yshift=-1.5mm,xshift=4mm},
  boxed title style={empty,boxrule=0pt},
  overlay unbroken={%
    \node[fill=defblue,text=white,circle,inner sep=1.2pt]
      at ([xshift=-6pt]frame.west|-title.north) {\bfseries D};
  },
}{def}


\usepackage[most]{tcolorbox}
\usepackage{xcolor}

\definecolor{exblue}{RGB}{160,170,210} % мягкий сиренево-голубой как на скрине

\newtcbtheorem[number within=section]{myexercise}{Упражнение}{%
  enhanced,
  colback=white,
  colframe=exblue,
  boxrule=0pt,
  borderline west={2pt}{0pt}{exblue}, % вертикальная линия
  sharp corners,
  fonttitle=\bfseries\itshape,        % "Упражнение 7.1." курсив, как в книге
  coltitle=black,
  attach boxed title to top left={yshift=-1.5mm, xshift=4mm},
  boxed title style={empty, boxrule=0pt},
  overlay unbroken={%
    \node[
      fill=exblue,
      text=white,
      circle,
      inner sep=1.2pt
    ] at ([xshift=-6pt]frame.west|-title.north) {\bfseries E};
  },
}{ex}


\definecolor{propurple}{RGB}{170,130,170} % цвет, как на скрине (фиолетовый)

\newtcbtheorem[number within=section]{myproposition}{Утверждение}{%
  enhanced,
  colback=white,
  colframe=propurple,
  boxrule=0pt,
  borderline west={2pt}{0pt}{propurple}, % вертикальная линия слева
  sharp corners,
  fonttitle=\bfseries,        % жирный заголовок
  coltitle=black,
  attach boxed title to top left={yshift=-1.5mm,xshift=4mm},
  boxed title style={empty, boxrule=0pt},
  overlay unbroken={%
    \node[
      fill=propurple,
      text=white,
      circle,
      inner sep=1.2pt
    ] at ([xshift=-6pt]frame.west|-title.north) {\bfseries P};
  },
}{prop}

\definecolor{propurple}{RGB}{170,130,170} % цвет, как на скрине (фиолетовый)

\newtcbtheorem[number within=section]{mylemma}{Лемма}{%
  enhanced,
  colback=white,
  colframe=propurple,
  boxrule=0pt,
  borderline west={2pt}{0pt}{propurple}, % вертикальная линия слева
  sharp corners,
  fonttitle=\bfseries,        % жирный заголовок
  coltitle=black,
  attach boxed title to top left={yshift=-1.5mm,xshift=4mm},
  boxed title style={empty, boxrule=0pt},
  overlay unbroken={%
    \node[
      fill=propurple,
      text=white,
      circle,
      inner sep=1.2pt
    ] at ([xshift=-6pt]frame.west|-title.north) {\bfseries L};
  },
}{prop}

\newtcbtheorem[number within=section]{mytheorem}{Теорема}{%
  enhanced,
  colback=white,
  colframe=propurple,
  boxrule=0pt,
  borderline west={2pt}{0pt}{propurple}, % вертикальная линия слева
  sharp corners,
  fonttitle=\bfseries,        % жирный заголовок
  coltitle=black,
  attach boxed title to top left={yshift=-1.5mm,xshift=4mm},
  boxed title style={empty, boxrule=0pt},
  overlay unbroken={%
    \node[
      fill=propurple,
      text=white,
      circle,
      inner sep=1.2pt
    ] at ([xshift=-6pt]frame.west|-title.north) {\bfseries T};
  },
}{prop}

\newenvironment{mylistingtext}[1]{
    \begin{figure}[h!]
    \centering
    \begin{minipage}{0.95\linewidth}
    \small               % уменьшенный шрифт
    \textbf{Листинг #1}~–\;
}{
    \end{minipage}
    \end{figure}
}
