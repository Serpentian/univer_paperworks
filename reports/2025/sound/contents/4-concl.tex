\conclusion

В ходе данного практического задания с использованием современных программных
средств были изучены:
\begin{enumerate}
    \item основные процедуры записи и обработки речевых сигналов;
    \item методы измерения параметров и анализа характеристик речевого сигнала,
          включая частотные характеристики гласных и согласных звуков.
\end{enumerate}

Определенные в рамках практической работы параметры звуков «А», «Б», «В»
приведены в таблице 4.

\begin{table}[ht]
\centering
\caption{Характеристики звуков}
\begin{tabular}{|c|c|cccc|}
\hline
\textbf{Звук} & \textbf{Слово-источник} & \multicolumn{4}{c|}{\textbf{Характеристики}} \\
\hline
             &                         & \textbf{Первая г.} & \textbf{Вторая г.} & \textbf{Длит.} & \textbf{Полоса частот} \\
\hline
А & А     & 127\,Гц   & 207\,Гц  & 0,255\,с   & 100--5223\,Гц \\
\hline
В & Двадцать & 88\,Гц   & 170\,Гц  & 0,065\,с  & 50--5175\,Гц \\
\hline
Г & Года & 114\,Гц   & 230\,Гц   & 0,102\,с  & 50--5175\,Гц \\
\hline
\end{tabular}
\end{table}

Рассмотрены возможности программных продуктов для работы со звуковыми сигналами
Praat, SoundEdit и WaveView. WaveView плохо работает на современных Windows и
обладает крайне плохим пользовательским интерфейсом, SoundEdit больше не
открывается на Windows 11. Хорошей альтернативой, по моему мнению, является
приложение с открытым исходным кодом Sonic Visualizer, поддерживающий систему
плагинов и обладающий большим комьюнити. Если какой-то возможности не хватает,
то лучше написать плагин в большую и хорошо работающую программу, чем
изобретать новую.
