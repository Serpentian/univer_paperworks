\structure{СПИСОК ЛИТЕРАТУРЫ}

    1. Горшков Ю.Г. Анализ и засекречивание речевого сигнала: Учебное пособие. М.: Издательство МГТУ им. Н.Э. Баумана. 2007. 34 с.

    2. Горшков Ю.Г. Криминалистическое исследование фонограмм / Методические указания к лабораторным работам. МГТУ им. Н.Э. Баумана. 2017. 32

    3. Горшков Ю.Г. Обработка речевых и акустических биомедицинских сигналов на основе вейвлетов / Научное издание. М.: Радиотехника. 2017. 240 с.

    4. P. Boersma and D. Weenink. Praat: Doing phonetics by computer (version 6.2.14) [computer program], 2022. (http://www.praat.org/)

    5. B. Hayes. Spectrogram reading practice, 2021. (https://linguistics. ucla.edu/people/hayes/103/SpectrogramReading/index.htm)

    6. Горшков Ю.Г. Визуализация многоуровневого вейвлет-анализа фонограмм //   Электронный журнал «Научная визуализация». Национальный Исследовательский   Ядерный Университет «МИФИ» № 2, том 7, квартал 2, 2015. C. 96-111.
