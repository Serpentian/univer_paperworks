\conclusion

Проведённый сравнительный анализ управленческих циклов (рис. \ref{fig:compare})
показывает, что большинство распространённых моделей управления — PDCA, DMAIC,
OODA — представляют собой частные, функционально ограниченные реализации
управленческого процесса. Они описывают лишь отдельные его стадии, не охватывая
всей полноты функций, обеспечивающих целостность и самовоспроизводимость
управления.

Циклы PMBOK и SCRUM выделяются на этом фоне как наиболее развитые формы
управленческой практики, приближающиеся к Полной функции управления (ПФУ) по
структуре и содержанию.

Стандарт PMBOK воспроизводит практически полный контур управления — от
инициации и планирования до анализа, совершенствования и завершения,
обеспечивая тем самым преемственность управленческого цикла.

Методология SCRUM, в свою очередь, демонстрирует воплощение ПФУ в динамическом,
итеративном формате. Её короткие циклы (спринты) включают осмысление вызова,
постановку цели, планирование, реализацию, анализ и корректировку, а функция
синхронизации встраивается в процесс через постоянные коммуникации команды.

Тем самым PMBOK и SCRUM можно рассматривать как практические формы реализации
ПФУ на проектном и командном уровнях, отражающие её ключевые принципы —
непрерывность, адаптивность и рефлексию.

Полная функция управления с включённой функцией синхронизации при этом
выступает как метамодель, интегрирующая все ранее созданные циклы и дающая
системное понимание управленческой деятельности.

Она объединяет в едином процессе выявление, целеполагание, реализацию,
контроль, совершенствование и завершение, обеспечивая устойчивое развитие
системы за счёт замкнутого контура обратных связей и согласованных действий
субъектов и объектов управления.

Тем не менее остаётся открытым вопрос: может ли Полная функция управления в её
универсальной форме одинаково эффективно применяться на всех уровнях — от
стратегического и организационного до операционного и индивидуального
управления? Ответ на этот вопрос требует дальнейших исследований и эмпирической
проверки применимости ПФУ в различных контекстах управленческой практики.
