\structure{ОСНОВНАЯ ЧАСТЬ}

Цикличность (повторяемость) - одно из важнейших свойств, качеств управления.
Различные объекты управления испытывают на себе воздействие некоторого набора
действий, задач, или точнее управленческих функций. В данной работе
производится попытка воссстановить полный управленческий цикл путем
исследования наиболее распространненных управленческих циклов.

Для описания функционирования систем управления используется понятие
управленческого цикла, то есть модели, описывающей процесс управления как
последовательное повторение типовых этапов \cite{Novikov2011}.

В учебниках по менеджменту принято за основу рассматривать цикл функций Анри
Файоля \cite{Management2023}: планировать (\textit{prévoir}), организовывать
(\textit{organiser}), распоряжаться (\textit{commander}), координировать
(\textit{coordonner}) и контролировать (\textit{contróler}). Мы также
рассмотрим наиболее популярный цикл Шухарта-Деминга (PDCA), который взят за
основу международных стандартов ISO, цикл DMAIC, активно используемый для
статистического управления проектами «6 Сигма», циклы управления проектами
«Норд», РМ ВОК и методологию SCRUM.

\section{Общепризнанные управленческие циклы}

\subsection{Цикл А.Файоля}

Впервые функции менеджмента были сформулированы в книге А.Файоля «Общий и
промышленный менеджмент» в 1916 году, где менеджмент рассматривается как
последовательный ряд операций или функций:
\begin{itemize}
    \item планирование (\textit{prévoir});
    \item организование (\textit{organiser});
    \item распоряжение (\textit{commander});
    \item координинация (\textit{coordonner});
    \item контроль (\textit{contróler}).
\end{itemize}

Согласно данному подходу, четыре первичные функции менеджмента (планирование,
организация, мотивация и контроль) объединены связующими процессами
коммуникации и принятия решения \cite{Verg2001}. Взаимосвязь между функциями
менеджмента на рис. \ref{fig:fayol}. представлена схемой, показывающей
содержание любого процесса менеджмента, вне зависимости от особенностей
экономического объекта.

\begin{figure}
  \centering
  \includegraphics[scale=0.6]{inc/fayol.png}
  \caption{Цикл управленческой деятельности (А. Файоль) \cite{Novikov2011}}
  \label{fig:fayol}
\end{figure}

Для реализации менеджмента в центре находится функция координации,
обеспечивающая согласование и взаимодействие всех остальных функций. Сам цикл
состоит лишь из планирования, организации, стимулирования, контроля,
координация же есть во всех четырер функциях управления.

Планирование - функция управления, определяющая цели деятельности, необходимые
для этого средства, а также разработка методов, наиболее эффективных в
конкретных условиях. Планирование включает в себя и составление прогнозов
возможного направления будущего развития организации в тесном взаимодействии с
окружающей ее средой.

Организация - формирование структуры объекта и обеспечение всем необходимым для
его нормальной работы - персоналом, материалами, оборудованием, зданиями,
денежными средствами и др. Любой план предусматривает стадию организации, т.е.
создания реальных условий для достижения запланированных целей.

Руководство (мотивация) - активизация работающих и побуждение их к эффективному
труду для выполнения целей, сформулированных в планах, с помощью экономического
и морального стимулирования и создания условий для проявления творческого
потенциала работников и их саморазвития.

Контроль - количественная и качественная оценка и учет результатов работы.
Контроль является элементом обратной связи, так как на основании его данных
производится корректировка ранее принятых решений, планов, норм и нормативов

Координация - достижение согласованности в работе всех звеньев системы путем
установления рациональных связей (коммуникаций) между ними.

\subsection{Цикл PDCA Шухарта-Деминга}

Цикл PDCA Шухарта-Деминга -- один из самых популярных и широко используемых
алгоритмов. Модель непрерывно вращающегося колеса
«планируй-делай-проверяй-воздействуй» (рис. \ref{fig:pdca}) показала свою
универсальность и применимость в любых ситуациях \cite{Partin2015}. Эта
модель смежна с циклом А. Файоля.

\begin{figure}
  \centering
  \includegraphics[scale=0.6]{inc/pdca.png}
  \caption{PDCA цикл}
  \label{fig:pdca}
\end{figure}

"Планируй". Цикл начинается с изучения текущей ситуации и сбора статистических
данных, которые используются для разработки плана совершенствования. Обычно на
этот этап уходит до 25\% времени.

"Делай" означает саму реализацию и создание условий для точной реализации
плана.

"Проверяй" связан с выяснением, удалось ли добиться задуманных результатов
(улучшений). Здесь есть разночтения: кто-то использует английское слово
«control», а кто-то «check». Последнее предполагает, что критерии оценки уже
определены на этапе планирования, и остается лишь сравнить – удалось или не
удалось достичь планируемого эффекта и перейти к следующему этапу. Есть также
версия, что Деминг в ранних версиях использовал вместо «С» буквц «S» - study
имея ввиду изучение результатов второго этапа.

«Воздействуй». Закрепление усовершенствований в качестве новой практики, что
тоже, в свою очередь будет усовершенствовано. Если спланированный опыт удался,
то возможна стандартизация использования новых методов, если же опыт не удался
– то следует извлечь уроки и планировать следующую итерацию.

Практика использования концепции PDCA западными компаниями часто базировалась
на разделении труда между мастерами, контролерами и рабочими, что закрывало
возможность рефлексивно реализовать функцию «совершенствование». Сегодняшнее
видение – либо целостное исполнение одной задачи одним лицом, либо работа в
команде, с обратной связью и совместной оценкой результатов.

\subsection{Цикл DMAIC}

DMAIC - подход, используемый в методологии «6 сигм», разработанный в компании
Motorola. Цикл DMAIC - аббревиатура наименования стадий (англ. define, measure,
analyze, improve, control), представляет собой детализацию цикла PDCA для
реализации проектов по совершенствованию процессов \cite{Partin2015}.

На первом этапе необходимо определить клиентов (внешних или внутренних) и их
приоритеты; проект, который бы удовлетворил клиента, улучшив при этом цели,
время и бюджет; а также выявить CTQ – характеристики, оказывающие решающее
влияние на качество.

На этапе «измерение» определяются основные метрики проекта, надёжность
источника данных, выявляются внутренние процессы, влияющие на CTQ, и
измеряется актуальное состояние.

Этап «анализ» - исследуются возможности процесса, причины проблем, возможных
рисков. Выявляются наиболее частые причины дефектов и параметры, которые
обуславливают отклонения в процессах.

На этапе «совершенствование» осуществляется модификация процессов и систем,
структурная декомпозиция работ. Определяются способы устранения причин,
подтверждается роль важнейших параметров и диапазоны их приемлемости.
Предлагается модификация процесса, который не будет выходить за приемлемые
рамки.

Последний этап «контроль» предназначен для поддержания введенных изменений
процесса, подготовка отчётов и закрытия проекта \cite{Pyzdek2003}.

Заметим, что DMAIC выглядит детальнее цикла PDCA, т.к. в явном виде включает
«измерение» и «анализ». Однако, в силу своей проектной ориентации
(одноразовости), DMAIC не включает важные шаги по принятию решения: отсутствует
этап выработки стандартного или эвристического решения (например, мозговой
штурм) и определение критериев принятия решения.  Однако, если в концепции
бережливого производства циклы Шухарта–Деминга рассматриваются как
непрекращающееся совершенствование, инициированное людьми и связанное с
корпоративной культурой, в концепции «шесть сигм» мы имеем дело с независимыми
циклами.

\subsection{Цикл НОРД}

Цикл НОРД - (англ. OODA, Observe–Orient–Decide Act «наблюдение, ориентация,
решение, действие») — концепция, разработанная Джоном Бойдом в 1995 году, также
известная как «петля Бойда». НОРД — это кибернетический самовоспроизводящийся и
саморегулирующийся цикл, имеющий в своей структуре 4 процесса \cite{Ivlev2024}.
Цикл этих последовательных четырех процессов предполагает многократное
повторение петли действий: происходит реализация принципа обратной связи (см.
рис.\ref{fig:boyd}).

\begin{figure}
  \centering
  \includegraphics[scale=0.4]{inc/boyd.png}
  \caption{Цикл НОРД - "Петля Бойда"}
  \label{fig:boyd}
\end{figure}

Концепция Бойда была разработана для военных целей и основывалась на боевом
опыте автора в качестве пилота. В современном мире технология распространена в
IT-компаниях для быстрой разработки продуктов в высококонкурентной среде. Если
сравнивать ее с основными и модифицированными, усовершенствованными вариациями
классического цикла PDCA, то очевидно, что в «петле Бойда» отсутствует этап
анализа и устранения отклонений, рефлексии, совершенствования процесса. В
быстроизменяющейся стрессовой внешней среде лицу, принимающему решение,
необходимо мгновенно оценить ситуацию и действовать самостоятельно, на свой
страх и риск. Каждый цикл начинается из новой точки и разрешает конкретную
сложившуюся ситуацию. Единственное, что здесь очень хорошо тренируется и
совершенствуется – опыт управленца наблюдать, ориентироваться, решать и
действовать.

\subsection{Цикл управления проектами РМ ВОК}

В современной литературе накоплен огромный пласт знаний и практического опыта
управления проектами. Эти практики Описаны в американском стандарте PM BOK,
японском P2M, и английском Prince-2, подкреплены стандартом ISO 21500,
исходящих все и того же «классического» толкования всех стандартов управления.

Практические консультанты на основе опыта применения стандарта предложили
циклический способ представления схемы цикла управления проектами PM BOK
\cite{PMBOK2017}, показав нелинейные варианты чередования процессов (рис.
\ref{fig:pmbok}). В смене процессов нет однозначной линейности – «процессы
исполнения» и «процессы анализа» образуют внутренний замкнутый друг на друга
цикл, который имеет два выхода – и обратно на «процессы планирования», и, в
случае достижения необходимого состояния системы – к «процесам завершения».

\begin{figure}
  \centering
  \includegraphics[scale=0.35]{inc/pmbok.png}
  \caption{Схема стандарта управления проектами РМ ВОК}
  \label{fig:pmbok}
\end{figure}

В этом алгоритме четко выделено то, что начало работы – «процессы инициации» -
это не какая-то конкретная точка, монолитная ступень, а отдельный процесс,
путь, который нужно пройти менеджеру. Здесь необходимо осмыслить ситуацию и
проблемы, вызовы, которые требуют запуска управленческого воздействия в виде
проекта по изменению ситуации, процесса или системы в целом. Заметим, что
процессы инициации могут как учитывать необходимость и гармонизацию внешней
среды, так и быть направленными в интересах разработчика. Это вопрос этики.
Однако японский стандарт проекта Р2М требует от менеджера проекта
ориентироваться на миссию программы и организации в целом.

Немаловажен «процесс завершения», или, как говорят в некоторых случаях
«утилизации» проекта. Часто запустив какие то проекты менеджеры увлечены их
развитием, получением быстрых результатов, и, когда проект принес свои эффекты,
о нем забывают, а технология проекта в лучше случае становится едедневной
рутиной, а в худшем – неработающей бюрократической надстройкой, которая
осталась на бумаге, и на которую приходится тратить основной ресурс – время.
Обе ситуации необходимо осмыслить, извлечь уровки и проинформировать среду об
этом решении. Кроме того, навести порядок в документации – утвердить или
отменить действие созданных под проект и в ходе проекта структур, процессов,
регламентов и методов

\subsection{Методология SCRUM}

Продолжая тему управления проектами и завершая обзор общепризнанных
управленческих циклов, остановимся на методике SCRUM. Концепция была
разработана в 80-х годах Хиротака Такэути и Икудзиро Нонака
\cite{Takeuchi1986}, где за основу были приняты лин-методология и цикл OODA
(петля Бойда). Они отметили, что проекты, над которыми работают небольшие
команды из специалистов различного профиля, систематически производят лучшие
результаты, и объяснили это как «подход регби». Они стали катализаторами
последующих разработок Кена Швабера и Джеффа Сазерленда сначала в сфере
программирования а затем и для других процессов.

Scrum2 — минимально необходимый набор мероприятий, артефактов, ролей, на
которых строится процесс Scrum-разработки продукта, позволяющий за
фиксированные небольшие промежутки времени (спринты), предоставлять конечному
пользователю работающий продукт с новыми бизнес-возможностями (инкремент),
имеющими наибольший приоритет

Свое широкое применение Scrum получил уже в 21 веке в IT- отрасли, где
разрабатывается большое количество инновационных цифровых продуктов, не имеющих
целостной заранее описанной технологии и даже с не заданными четко свойствами
конечного результата. Часто в процессе выполнения одной задачи под заказчика,
разработчики видят новые, ранее не запланированные и не предполагаемые
возможности и свойства цифровых систем. Поэтому разработка идет «короткими
схватками» или «перебежками», когда продукт представляется на пользователям
после каждой итерации и можно быстро протестировать его свойства, чтобы
скорректировать и улучшить его на следующем этапе разработки.

Методика Scrum предполагает циклические ответы на следующие вопросы: Куда
стремимся? Каково текущее состояние? Каково следующее целевое состояние? Какова
суть эксперимента. Каковы полученные уроки? Каков следующий шаг?

Показательно разделение привычного этапа постановки целей на три шага –
прояснение вызова и задание вектора к идеальному состоянию («куда стремимся?),
что былла всегда не характерно для краткосрочного планирования, а для стратегий
надо было потратить большое количество времени на формулировку видения и
миссии. Здесь же компактная команда постоянно тренирует это видение на короткой
дистанции, не теряя его из виду. Точка «А» («текущее состояние») задает
понимание имеющейся проблем и ресурсов, а точка «В» (следуюее целевое
состояние») определение направления быстрого движения. Далее команде позволено
творить в режиме экспертимента, изобретая качества и технологию продукта на
ходу, причем время на эту работу в технологии Scrum четко ограничено
продолжительностью короткого спринта (от 1 до 3 недель). По истечении спринта
проводится анализ полученного результата, опыта и извлечение уроков,
определяется текущая точка «А» и все повторяется заново.

\section{Полная функция управления}

Очевидно, что функции планирования, выполнения, контроля и улучшения есть во
всех представленых циклах. Следовательно цикл Шухарта – Деминга PDCA сохраняет
за собой первенство самого распространного, сквозного способа организации
деятельности управленцев. Но, претендует ли он на универсальность? Можно ли
сказать, что он дает необходимые и достаточные элементы для полноценного
творения и созидания систем и их непрерывного улучшения?

Прежде всего, недостаточность цикла PDCA в его исходном виде привела к его
достройке – добавлению этапов и до этапа планирования в разных вариациях циклов
(инициация, определение проблемы, наблюдение, анализ, исследование, измерение и
т.п.), и после этапа совершенствования (извлечение уроков, завершение,
определение следующих шагов и т.п.).

В описаниях практик применения многих исследуемых циклов говорится о том, что
организации внедряют у себя несколько известных подходов, так как они закрывают
отдельные функциональные области деятельности организации или применяются
отдельными командами, а организация в целом, ее процессы и системы не
управляются целостно.

Наиболее полным циклом управленческой деятельности можно назвать "Полную
функцию управления", представленную в "Достаточно общей теории управления"
(\cite{OsnovySociologii}), и определяему как последовательность
разнокачественных действий, в которой реализуется процесс управления во всей
полноте выявленных возможностей и детальности. Иначе говоря, полная функция
управления вбирает в себя всю алгоритмику управления объектом.

ПФУ «описывает преемственные этапы циркуляции и преобразования информации в
процессе управления, начиная с момента выявления субъектом фактора среды,
вызывающего у него субъективную потребность в управлении и формирования
субъектом-управленцем вектора целей управления и далее до осуществления
намеченных целей в процессе управления». Кратко этапы ПФУ в их исходной
авторской последовательности представлены в таблице \ref{tab:control_stages}:

\begin{table}[h]
\centering
\caption{Этапы «Полной функции управления»}
\label{tab:control_stages}
\begin{tabular}{|p{0.45\linewidth}|p{0.45\linewidth}|}
\hline
\textbf{Наименование этапа в интерпретации авторов статьи} & \textbf{Этапы в описании авторов «Достаточно общей теории управления»} \\
\hline
1. Прозрение и выявление & Выявление фактора среды, который «давит на психику», чем и вызывает субъективную потребность в управлении. \\
\hline
2. Идентификация & Формирование навыка (стереотипа) распознавания фактора среды на будущее и распространение его в культуре общества. \\
\hline
3. Анализ и целеполагание & Целеполагание в отношении выявленного фактора.  \\
\hline
4. Концепция управления & Формирование генеральной концепции управления и частных концепций управления в отношении каждой из целей в составе вектора целей\ldots \\
\hline
5. Реализация/Внедрение & Внедрение генеральной концепции управления в жизнь — организация новых или реорганизация существующих управляющих структур, несущих целевые функции управления. \\
\hline
6. Выявление отклонений & Контроль (наблюдение) за деятельностью структур в процессе управления, осуществляемого ими, и координация взаимодействия разных структур. \\
\hline
7. Совершенствование & Совершенствование действующей концепции в случае необходимости. \\
\hline
8. Адаптация/Ликвидация & Ликвидация существующих структур и высвобождение используемых ресурсов в случае ненадобности либо поддержание их в работоспособном состоянии до следующего использования. \\
\hline
\end{tabular}
\end{table}

В своей работе Э.В. Кондратьев и А.Г. Макарова \cite{Kondratiev2024} дополняют
ПФУ новой стадией -- синхронизация. Она реализуется через коммуникации субъекта
и объекта управления, где вся полнота внутреннего вызова и идей (как
представление об идеальном состоянии) предиктора передается корректору,
несущего их в своей объективности и через конкретные действия достигает
целевого состояния реальности. Это представлено на рис. \ref{fig:pfu}.

\begin{figure}
  \centering
  \includegraphics[scale=0.6]{inc/pfu.png}
  \caption{Полная функция управления \cite{Kondratiev2024}}
  \label{fig:pfu}
\end{figure}

\section{Сравнительный анализ циклов управления}

Представим все рассмотренные циклы управления в сравнении между собой и Полной
функцией управления с включенной для полной картины анализа функцией
синхронизации (рис. \ref{fig:compare})

\begin{figure}
  \centering
  \includegraphics[scale=0.23]{inc/compare.png}
  \caption{Сравнительная матрица циклов управления}
  \label{fig:compare}
\end{figure}

В сводной матрице видно, что по сравнению с ПФУ, наиболее полно этапы процесса
управления представлены в технологии SCRUM, в которой есть этапы пересмысления
идеального и анализа текущего состояния системы, и после полученных уроков
требуется этап принятия решения о том, продолжать ли следовать выбранному пути
или стоит его завершить и начать новый. Так как механика SCRUM предполагает
работу в командах на всех этапах, то синхронизация обеспечивается постоянной
коммуникацией участников, хоть и не выделена как отдельная функция.

В остальных циклах управления видны пробелы (или разрывы) функций, не
позволяющие реализовать целостную концепцию управления. Именно поэтому
организации зачастую ищут выход в компиляции управленческих инструментов, и
чаще всего, эти лакуны интуитивно заполняются отдельными задачами со стороны
управленцев, раздаваемыми в ручном режиме, а в более зрелых организациях –
локально регламентированными процессами.
