\appendixsection{Цикл СПРУКАР}

Цикл СПРУКАР подробно рассматривался на лекциях и семинарах, поэтому его
подробный анализ не включен в доклад, чтобы не повторяться. Циклы Фойеля и
PDCA пришлось включить, так как они исключительно общеприняты, на их основе
строятся все остальные рассмотренные циклы.

Представление цикла менеджмента в виде формальных схем получило свое развитие в
работах В.В.Кондратьева \cite{Kondratiev2007} и авторского коллектива под
руководством Д.А.Новикова \cite{Burkov2009}. Этапы или функции управленческого
цикла они рассматривают, в виде циклически повторяющейся последовательности,
представленной на рис. \ref{fig:sprukar}.

\begin{figure}
  \centering
  \includegraphics[scale=0.4]{inc/sprukar.png}
  \caption{Структура системы управления в виде цикла СПРУКАР \cite{Novikov2011}}
  \label{fig:sprukar}
\end{figure}

Цикл СПРУКАР включает в себя:
\begin{itemize}
    \item \textbf{С} -- сбор и анализ информации для принятия управленческого
        решения;
    \item \textbf{П} -- планирование действий -- разработка и принятие
        управленческого решения;
    \item \textbf{Р} -- реализация -- организация и исполнение управленческого
        решения (в т.ч. мотивация исполнителей);
    \item \textbf{У} -- учет фактически полученных результатов реализации
        управленческого решения;
    \item \textbf{К} -- контроль -- сравнение фактического и запланированного
        результата;
    \item \textbf{А} -- анализ -- выявление и изучение причин отклонения
        фактических от запланированных результатов;
    \item \textbf{Р} -- регулирование хода исполнения управленческого решения,
        коррекция (при необходимости) ранее принятых решений (в т.ч. применение
        к исполнителям предусмотренных мотивационных воздействий с целью
        усиления выявленных положительных тенденций и нейтрализации
        отрицательных).
\end{itemize}
