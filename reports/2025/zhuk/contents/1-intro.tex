\structure{ВВЕДЕНИЕ}

В современных условиях управление всё чаще сталкивается с проблемой потери
целостности. Организации применяют множество методик и инструментов — от
классических циклов Файоля и Деминга до современных подходов, таких как PMBOK и
SCRUM.

Однако на практике часто оказывается, что эти модели охватывают лишь часть
управленческого процесса: где-то отсутствует анализ и рефлексия, где-то —
завершение или согласование действий участников. В результате управление теряет
замкнутость и перестаёт быть саморегулирующимся циклом.

Цель доклада — показать, как различные управленческие циклы соотносятся между
собой и с Полной функцией управления (ПФУ), которая рассматривается как
наиболее целостная и универсальная модель управленческой деятельности.

Мы сравним классические и современные циклы — Файоля, PDCA, DMAIC, OODA, PMBOK
и SCRUM, — выделим их сильные стороны, общие этапы и функциональные разрывы.

Особое внимание уделяется функции синхронизации, обеспечивающей согласование
действий субъекта и объекта управления, а также вопросу целостности
управленческого цикла как основы эффективности управления в любой сфере.

В завершение доклада рассматривается Полная функция управления с включённой
функцией синхронизации.
