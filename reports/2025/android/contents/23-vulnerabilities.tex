\section{Угрозы и уязвимости}

Несмотря на многоуровневую архитектуру безопасности и активное применение
механизмов шифрования, устройства Android остаются уязвимыми перед рядом угроз.
Эти угрозы могут быть как программными, так и аппаратными, и направлены на
обход систем защиты, извлечение данных или нарушение целостности информации. В
этом разделе рассмотрим основные типы уязвимостей и приведём конкретные примеры
их эксплуатации.

\subsection{Программные угрозы}

\textbf{1. Вредоносные приложения (malware).}
Одной из самых распространённых угроз являются вредоносные приложения, которые
могут маскироваться под легитимное программное обеспечение. Они способны
похищать пользовательские данные, перехватывать нажатия клавиш, записывать
экран или аудио, а также отправлять информацию на удалённые серверы. Особенно
опасны трояны, получающие доступ к разрешениям на чтение памяти устройства.

\textit{Пример:} вирус \textbf{Joker}, обнаруженный в 2020 году, заражал
десятки приложений в Google Play и тайно подписывал пользователей на платные
услуги, похищая СМС и данные о контактах.

\textbf{2. Эскалация привилегий (Privilege escalation).}
Многие уязвимости связаны с возможностью выполнения кода с повышенными правами.
Злоумышленник может использовать эксплойты, чтобы обойти ограничения песочницы
и получить root-доступ, что даёт полный контроль над системой, включая доступ к
зашифрованным данным (при наличии ключей в оперативной памяти).

\textit{Пример:} уязвимость \textbf{Dirty COW} (CVE-2016-5195) позволяла
изменять содержимое защищённых файлов, используя гонку условий в ядре Linux.
Она применялась для получения root-доступа на Android-устройствах.

\textbf{3. Уязвимости в драйверах и прошивке.}
Драйверы оборудования и компоненты прошивки часто являются слабым звеном в
системе безопасности. Уязвимости в этих компонентах могут позволить выполнять
произвольный код до запуска системы, в обход механизмов шифрования и контроля
целостности.

\textit{Пример:} атаки на загрузчик (bootloader) позволяют заменить прошивку
или ядро, нарушив цепочку доверия Verified Boot и расшифровав пользовательские
данные.

\subsection{Аппаратные угрозы}

\textbf{1. Атаки с физическим доступом.}
Если злоумышленник получает физический доступ к устройству, он может попытаться
извлечь данные с использованием специализированного оборудования.

\textit{Пример:} \textbf{Cold Boot Attack} — техника, при которой данные из
оперативной памяти сохраняются при кратковременном отключении питания и могут
быть извлечены, включая криптографические ключи.

\textbf{2. Брутфорс PIN-кодов и паролей.}
Хотя Android ограничивает число попыток ввода пароля, при наличии аппаратных
уязвимостей можно обойти эти ограничения.

\textit{Пример:} в чипах Qualcomm Snapdragon используется собственная
реализация аппаратно изолированного окружения — QSEE (Qualcomm secure execution
environment). В нем запускаются доверенные обработчики (trustlets), включая
модуль обработки ключей (KeyMaster). Как показал этим летом Гэл Беньямини (Gal
Beniamini), в QSEE по факту нет полной аппаратной изоляции. Атакующий может
запустить свой код в пространстве QSEE. При этом он станет доверенным и
автоматически повысит привилегии, после чего сможет считать через KeyMaster как
зашифрованный мастер-ключ, так и захардкоженный ключ HBK.

Беньямини опубликовал скрипт для извлечения ключей с устройств на базе Qualcomm
Snapdragon и дальнейшие инструкции по подбору пользовательского пароля
перебором. Брутфорс не представляет сложности, так как у основной массы
пользователей короткие пароли. Поскольку атака перебором выполняется не на
смартфоне, а на любом компьютере с помощью скрипта, встроенные средства защиты
от брутфорса оказываются бессильны. При внешнем брутфорсе не возникает ни
проблем с принудительными задержками, ни риска стирания данных после N
неудачных попыток.

\textbf{3. Уязвимости в биометрических системах.}
Системы распознавания отпечатков пальцев и лица могут быть обмануты с помощью
подделок или изображений. Это особенно актуально для устройств с менее
надёжными датчиками.

\textit{Пример:} исследователи демонстрировали возможность обхода сканеров
отпечатков с помощью отпечатков, напечатанных на 3D-принтере.

\subsection{Обход шифрования}

Даже если данные физически зашифрованы, злоумышленник может попытаться получить
к ним доступ путём:

\begin{itemize}
    \item Извлечения ключей из оперативной памяти, если устройство было разблокировано.
    \item Использования уязвимостей в реализации FBE/FDE.
    \item Замены системных библиотек с целью перехвата паролей или PIN-кодов.
\end{itemize}

Кроме того, вредоносное ПО может просто получить доступ к уже расшифрованным
данным, если пользователь сам открыл устройство и дал необходимые разрешения
(например, доступ к хранилищу, камере, микрофону).

\subsection{Недостатки в реализации безопасности}

Иногда даже правильно реализованные механизмы могут страдать от ошибок
конфигурации или недостаточной строгости. Например:

\begin{itemize}
    \item Разработчики приложений могут неправильно использовать Android
        Keystore, храня ключи в открытом виде.
    \item Некоторые производители устройств отключают или модифицируют защитные
        функции (например, Verified Boot) для упрощения прошивки.
    \item Пользователи могут сами ослабить защиту, разблокировав загрузчик или
        установив root-доступ.
\end{itemize}

