\conclusion

В ходе данного исследования была подробно рассмотрена архитектура безопасности
Android и роль шифрования в защите пользовательских данных. Android как
операционная система предоставляет мощный набор механизмов для обеспечения
конфиденциальности, целостности и недоступности информации без соответствующего
доступа. Шифрование в Android является неотъемлемой частью комплексной модели
защиты, начиная с уровня ядра и заканчивая пользовательскими приложениями.

Были изучены такие ключевые технологии, как File-Based Encryption (FBE),
Android Keystore и аппаратные средства защиты (TrustZone, Secure Element, Titan
M). Особое внимание было уделено архитектуре Keystore как важнейшему компоненту
безопасного хранения ключей, и практическим последствиям разблокировки
загрузчика, что особенно актуально в контексте защиты данных при физическом
доступе к устройству.

В результате анализа угроз и уязвимостей стало очевидно, что, несмотря на
высокий уровень защиты, Android-устройства остаются подверженными рискам,
особенно при наличии root-доступа, устаревшего ПО или разблокированного
загрузчика. Практический эксперимент показал, что хотя разблокировка загрузчика
уничтожает ключи и делает невозможным восстановление ранее зашифрованных
данных, она резко снижает безопасность устройства в будущем и даёт
потенциальным злоумышленникам инструменты для модификации системы.

Таким образом, можно сделать следующие выводы:

\begin{itemize}
    \item Современные Android-механизмы шифрования являются надёжными при
        условии соблюдения стандартов безопасности.
    \item Обеспечение безопасности данных требует как технологических решений
        (шифрование, аппаратная защита), так и внимательного отношения со
        стороны пользователя (обновления, контроль разрешений, отказ от
        root-доступа).
\end{itemize}

В условиях постоянного роста цифровых угроз роль эффективного шифрования и
продуманной архитектуры безопасности будет только возрастать. Android
продолжает развиваться, внедряя всё более совершенные механизмы защиты, однако
ответственность за безопасность остаётся также и на стороне пользователя и
разработчиков приложений.
