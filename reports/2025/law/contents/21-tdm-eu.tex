\section{Авторские и смежные права при применении TDM в ЕС}

Согласно исследованию, проведённому в рамках проекта \textit{Future for Text
and Data Mining}, обрабатываемая в ТДМ информация условно делится на два типа:
насыщенные по содержанию данные (статьи, книги, фильмы и др.), подлежащие
авторско-правовой защите, и низкоуровневые данные, такие как числовые
измерения, имена, координаты и т.п., чаще всего не охраняемые правом
\cite{futuretdm}.

Поскольку среди обрабатываемой информации могут находиться охраняемые
материалы, возникает вопрос: подпадает ли ТДМ под действия, ограниченные
исключительным правом. По общему правилу, использование охраняемых произведений
без согласия правообладателя запрещено. Однако закон предусматривает
исключения, позволяющие свободное использование при определённых условиях.

Некоторые исследователи полагают, что ТДМ не нарушает исключительных прав,
поскольку не направлено на использование произведения как культурного объекта —
анализу подвергается его структура как массив данных \cite{muto2016,
geiger2020tdm}. Поскольку авторское право охраняет только форму выражения, а не
идеи и факты, оперирование фактологическим содержанием не требует согласия
автора. Аналогичный подход применим и к базам данных, если обработке
подвергаются не структура, а лишь содержимое \cite{geiger2020tdm}.

Кроме того, копии, возникающие при ТДМ, зачастую являются технически
необходимыми и не предназначены для распространения. Учёные подчёркивают, что
запрет на такие временные копии может фактически блокировать доступ к
информации и мешать исследованиям, что нарушает право на получение знаний
\cite{flynn2020}.

Тем не менее определённые действия, совершаемые в рамках ТДМ, потенциально
могут затрагивать интеллектуальные права, особенно если обрабатываются
сторонние материалы без соответствующего разрешения \cite{geiger2020tdm}. В
каждом конкретном случае правомерность использования зависит от источника
данных, технических методов их извлечения и объёма задействованной информации.

\subsection*{Копирование и воспроизведение при TDM}

Применение TDM часто включает создание копий исходных материалов, особенно при
их преобразовании в машиночитаемый формат или при загрузке на платформу
анализа. Эти действия могут затронуть право на воспроизведение, особенно если
используются защищённые авторским правом произведения \cite{geiger2020tdm}.

Суд Европейского Союза в решении по делу \textit{Infopaq} отметил, что даже
извлечение короткого фрагмента текста (в частности, 11 слов) может считаться
нарушением, если такой отрывок отражает творческое выражение автора
\cite{infopaq2009}. Аналогичный подход содержится и в российском
законодательстве, согласно которому использование части произведения без
согласия возможно только в случае, если она не имеет самостоятельной творческой
ценности.

Существует мнение, что некоторые формы извлечения, например \textit{crawling} —
автоматический просмотр и индексирование контента, не предполагают
значительного копирования и не нарушают авторские права. Однако если такие
действия затрагивают оригинальные элементы произведения, они могут
квалифицироваться как воспроизведение \cite{muto2016}.

Итоговая информация, получаемая в результате TDM (output), обычно представляет
собой обобщённые данные и отчёты, не включающие фрагменты исходных материалов,
что позволяет избежать нарушения прав \cite{triaille2014tdm}. Проблемы могут
возникать в случае, если сохраняются и передаются копии исходных произведений
третьим лицам. В ЕС это может, вероятно, привести к нарушению права на
публичное сообщение (the right of communication to the public)
\cite{geiger2020tdm}, в России - к нарушению права на распространение /
доведение до всеобщего сведения в зависимости от того, каким образом
предоставляется доступ.

\subsection*{Авторское право баз данных}

Когда объектом TDM является база данных, охраняемая авторским правом,
копирование её структуры и подборки материалов на этапе подготовки может
рассматриваться как воспроизведение. Дополнительные операции, такие как
удаление нерелевантных данных или перевод на другой язык, могут затрагивать
права на модификацию и перевод соответственно \cite{triaille2014tdm}.

Тем не менее на завершающем этапе TDM итоговые данные редко сохраняют исходную
структуру или содержат идентифицируемые элементы базы. Обычно результат анализа
представляет собой обобщённую информацию, не позволяющую восстановить
оригинальную базу или её части. Поэтому итоговый отчёт (output) маловероятно
нарушает права на базу данных \cite{muto2016, triaille2014tdm}.

\subsection*{Смежное право баз данных}

При обработке баз данных, охраняемых по режиму sui generis или смежным правом,
технологии TDM могут затрагивать исключительное право на извлечение и
использование существенных частей этих баз. Под извлечением понимается перенос
данных на другой носитель, что подтверждено в практике Суда ЕС
\cite{britishhorseracing2004} и закреплено в российском праве (ГК РФ абз. 2 п.
1 ст. 1334).

Создателям баз данных предоставляется право ограничивать извлечение
существенного объёма данных, чтобы обеспечить компенсацию за инвестиции в их
создание \cite{directive96}. Однако, как отмечают некоторые исследователи, при
использовании технологий анализа больших данных, включая TDM, не происходит
замещения отдельных баз — напротив, формируется новое информационное качество
за счёт объединения данных из множества источников \cite{zieger2020}.

Рассматривая эти риски, Европейский союз в 2019 году принял положения (ст. 3 и
4 Директивы 2019/790/ЕС), легализующие свободное использование материалов для
целей TDM. Эти нормы дополняют ранее действовавшие исключения, существовавшие,
в частности, в законодательстве Великобритании, Франции и Германии. Так, ещё в
2014 году в Великобритании было введено исключение для некоммерческого научного
использования произведений \cite{ukcopyright1988}. В 2016 и 2017 годах
аналогичные положения появились во Франции и Германии соответственно
\cite{frcode, gerurhg}.

Эти исключения направлены на содействие исследовательской деятельности, не
нарушая при этом разумные интересы правообладателей. Согласно немецкому
законодательству, TDM в научных целях допускается, включая извлечение данных и
их временное хранение, при условии, что деятельность носит некоммерческий
характер и копии уничтожаются после завершения исследований.

\subsection*{Свободное использование при TDM: нормы ЕС}

Ряд норм ЕС предусматривает случаи, при которых использование охраняемых
произведений и баз данных допускается без согласия правообладателя — в том
числе в целях TDM. Важнейшие положения содержатся в:
\begin{itemize}
    \item п. 1 и 3a ст. 5 Директивы Европейского парламента и Совета ЕС от
    22.06.2001 N 2001/29/ЕС "О гармонизации некоторых аспектов авторских и
    смежных прав в информационном обществе" (далее - Директива об
    информационном обществе) \cite{directive2001};
    \item п. 1 ст. 6 и п. 1 ст. 8 Директивы Европейского парламента и Совета ЕС
    от 11.03.1996 N 96/9/ЕС "О правовой охране баз данных" (далее -
    Директива о базах данных) \cite{directive1996}.
\end{itemize}

Согласно ст. 5(1) Директивы об информационном обществе, временное техническое
копирование допускается, если оно:
\begin{itemize}
  \item краткосрочно и связано с технологическим процессом;
  \item необходимо для правомерного использования;
  \item не имеет самостоятельного экономического значения.
\end{itemize}

ЕС расценивает, что такие копии не нарушают авторские права, если создаются,
например, в оперативной памяти и удаляются автоматически \cite{muto2016,
triaille2014tdm}. Однако применение этого исключения к TDM оценивается
по-разному: возникают сомнения, являются ли создаваемые копии временными и
некоммерческими \cite{geiger2020tdm}.

В Директиве о базах данных указано, что правомерный пользователь может
копировать базу, если это необходимо для доступа к её содержанию (ст. 6(1)), а
также использовать несущественные части базы (ст. 8(1)). Германия, например,
уже признала, что TDM может быть частью «нормального использования»
\cite{triaille2014tdm}.

