\section{Авторские и смежные права при применении TDM в РФ}

В Послании Президента Российской Федерации от 15 января 2020 года подчёркнута
важность обеспечения условий для научной и инновационной деятельности, а также
необходимость развития правового регулирования оборота больших данных в
условиях цифровой экономики.

Анализ массивов данных с применением технологий TDM зачастую затрагивает
объекты авторского права, особенно когда речь идёт о произведениях науки,
литературы или искусства. Такие действия могут включать техническое
копирование, трансформацию и анализ, что подпадает под режим исключительных
прав.

В связи с этим в научных и экспертных кругах высказываются предложения о
корректировке законодательства в целях установления специальных оснований для
правомерного использования защищённых данных в рамках TDM. Особый акцент
делается на некоммерческое исследовательское использование, при котором права
правообладателей в минимальной степени затрагиваются.

В этом контексте актуальной задачей становится анализ уже существующих
ограничений авторских и смежных прав в российском законодательстве. При
выявлении пробелов возможным шагом является разработка новых исключений,
позволяющих проводить интеллектуальный анализ данных без необходимости
получения разрешения в ограниченных и чётко определённых случаях.

\subsection*{Авторское право}

Согласно подп. 1 п. 2 ст. 1270 ГК РФ \cite{gk_rf}, воспроизведением
произведения признаётся изготовление его экземпляра (или части) в любой
материальной форме, включая запись в память ЭВМ. Это положение охватывает
действия по цифровому копированию, в том числе при TDM.

Если копируется лишь часть произведения, её использование будет нарушением
исключительного права, только если она представляет собой самостоятельный
результат творческого труда (п. 7 ст. 1259 ГК РФ). Исключением признаётся
краткосрочная запись, неотъемлемая часть технологического процесса, не имеющая
самостоятельного экономического значения, если она необходима для правомерного
использования или технической передачи произведения (подп. 1 п. 2 ст. 1270 ГК
РФ). Эта норма была введена в 2006 году и уточнена в 2014 году, сблизившись по
содержанию с аналогичным исключением из Директивы об информационном обществе.

Применение этого исключения возможно только при соблюдении всех условий:
кратковременность записи, её побочный характер, связь с правомерным
использованием и отсутствие самостоятельной коммерческой ценности. На практике,
соблюдение этих условий при TDM не всегда очевидно, особенно при создании
долговременных копий или последующем анализе базы данных.

Отдельный вопрос касается применения этой нормы к базам данных. Если база
охраняется как составное произведение (п. 2 ст. 1260 ГК РФ), исключение
применимо. Однако для баз, охраняемых смежным правом (гл. 71 ГК РФ),
аналогичная норма отсутствует. Некоторые авторы предлагают применять положения
ст. 1270 ГК РФ по аналогии, но расширительное толкование ограничений
исключительных прав представляется недопустимым \cite{sergeev2018}.

Применительно к базам данных действует также п. 1 ст. 1280 ГК РФ. Он допускает
техническое копирование, исправление ошибок и создание архивной копии, но
исключительно в целях обеспечения функционирования базы. Поскольку TDM
предполагает извлечение и анализ, а не эксплуатацию базы в прямом смысле,
положения ст. 1280 ГК РФ ограничены по применимости.

Таким образом, российское законодательство не содержит специального исключения
для TDM, в том числе в научных целях. Это отличает его от права ЕС, где
подобные исключения предусмотрены напрямую. Отсутствие таких норм затрудняет
правовую квалификацию TDM и может вести к рискам нарушения исключительного
права при обработке защищённых данных.

\subsection*{Смежное право}

Согласно абз. 2 п. 1 ст. 1334 ГК РФ, никто не вправе извлекать из базы данных
материалы и использовать их без разрешения правообладателя, за исключением
случаев, установленных законом. Извлечение — это перенос всей базы или её
существенной части на другой носитель любыми средствами. Перенос незначительной
части базы не подпадает под понятие извлечения и не входит в содержание
исключительного права её изготовителя.

Таким образом, исключительное право распространяется на всю базу или её
существенные части. Использование несущественной части не нарушает права, кроме
случаев, указанных в п. 3 ст. 1335.1 ГК РФ.

Извлечение означает заимствование информации, а не создание копии базы данных,
включая временный перенос данных. Понятие «последующее использование» не
раскрыто в законе, но, по мнению исследователей, оно означает доведение
информации до третьих лиц (например, публикация), в отличие от простого
хранения.

В ЕС под последующим использованием понимается любое публичное предоставление
содержания базы (ст. 7 Директивы). Российский подход строже: нарушение возможно
лишь при совокупности извлечения и использования (абз. 2 п. 1 ст. 1334 и п. 3
ст. 1335.1 ГК РФ). Извлечение без использования не образует нарушение.

П. 1 ст. 1335.1 ГК РФ предусматривает случаи свободного использования баз
данных: по назначению, в личных, научных, образовательных целях, либо при
использовании несущественной части. Эти исключения не распространяются на иные
объекты в базе (например, авторские произведения).

«Правомерное использование» предполагает, что база получена на законных
основаниях (договор, онлайн-доступ, и т.п.).

Первое исключение — использование по целевому назначению, если договором не
предусмотрено иное (абз. 2 п. 1 ст. 1335.1). Договор может ограничивать способы
использования.

Второе — использование в личных, научных, образовательных целях (абз. 3 п. 1
ст. 1335.1), объём определяется поставленной целью. Иногда допустимо
использование всей базы, если это необходимо.

Третье — использование несущественной части (абз. 4 п. 1 ст. 1335.1). Закон не
даёт определения несущественной части; она устанавливается по
контексту. Не допускается договорное ограничение этого исключения,
кроме случаев, указанных в п. 3 ст. 1335.1.

Согласно абз. 5 п. 1 ст. 1335.1 ГК РФ, при публичном доступе к извлечённым
материалам должна указываться база данных. Нарушение этого требования не влечёт
нарушения исключительного права, так как оно направлено на обеспечение
прозрачности источников. В случае невозможности соблюдения
(например, при Big Data) целесообразно ввести исключение \cite{geiger2020tdm}.

П. 3 ст. 1335.1 ГК РФ запрещает неоднократное извлечение несущественных частей,
если оно нарушает нормальное использование и ущемляет интересы изготовителя.
Эти критерии аналогичны трёхступенчатому тесту (п. 5 ст. 1229 ГК РФ, Бернская
конвенция, ст. 13 ТРИПС)~\cite{hug}.

Суд определяет, нарушены ли права изготовителя с учётом объёма, целей
использования, убытков и прочих факторов.

Обработка Big Data может привести к нарушению прав по п. 3 ст. 1335.1 ГК РФ,
если в результате извлечения правообладатель теряет возможность получения
дохода. Однако извлечение без последующего использования (например, если
результат анализа не содержит существенных частей базы) не считается нарушением
по абз. 2 п. 1 ст. 1334 ГК РФ.
