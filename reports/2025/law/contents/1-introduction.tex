\introduction

В современном цифровом обществе, на фоне быстрого прогресса в сфере
информационных технологий, ежедневно генерируются и обрабатываются огромные
объёмы цифровой информации \cite{geiger2020tdm}. Для того чтобы эффективно
справляться с такими объёмами данных, активно применяются передовые методы
анализа, в том числе технология интеллектуального анализа текстов и данных ---
\textit{text and data mining} (ТДМ). Эта методика позволяет автоматически
выявлять полезную информацию из больших массивов данных, что делает её особенно
востребованной в таких отраслях, как менеджмент, научные исследования,
здравоохранение, маркетинг и другие.

Методы ТДМ могут использоваться не только для анализа открытых данных, но и в
отношении объектов, охраняемых авторским правом, включая научные труды,
произведения литературы и искусства, а также базы данных. Это обстоятельство
вызывает ряд юридических вопросов, в частности, необходимо ли получение
разрешения от правообладателя для проведения такого анализа, либо в некоторых
ситуациях возможно свободное использование материалов без согласия их
владельцев.

Юридическое регулирование обработки информации с помощью автоматизированных
систем охватывает множество правовых направлений: право интеллектуальной
собственности, договорное право, конкурентное право, законодательство о
персональных данных, законодательство об открытых данных, коммерческой тайне и
др. Эта работа сосредоточена, прежде всего, на аспектах, связанных с правами
интеллектуальной собственности.

В рамках изучаемой темы особое внимание уделяется следующим ключевым вопросам:
\begin{enumerate}
    \item При каких условиях набор данных или база данных может быть признана
          объектом авторского права и (или) смежных прав?
    \item Подпадает ли использование базы данных или её элементов в процессе
          машинного обучения под правовой режим использования объектов
          интеллектуальной собственности?
    \item В каких случаях допустимо свободное использование таких объектов, а
          когда необходимо получение согласия правообладателя?
    \item Какие рекомендации по изменению законодательства РФ можно дать?
\end{enumerate}

Основная цель данной работы заключается в анализе указанных правовых вопросов в
контексте применения технологий обработки больших данных, включая ТДМ, с учётом
правовых норм и подходов, сформированных в Европейском союзе и на территории
Российской Федерации.
