\structure{ОСНОВНАЯ~ЧАСТЬ}

\section*{Основные этапы TDM}

Технология интеллектуального анализа текстов и данных (TDM, ТДМ), как правило,
реализуется через несколько ключевых этапов, каждый из которых играет важную
роль в достижении конечного результата \cite{triaille2014tdm}. Ниже приведена
общая структура этого процесса:

\begin{enumerate}
    \item На первом этапе определяется массив информации, подлежащий анализу.
    Это могут быть как самостоятельно собранные данные, так и уже
    структурированные массивы в составе существующих баз. Информация может
    поступать из самых различных источников и иметь разнообразные форматы ---
    от текстовых документов до веб-страниц и других цифровых ресурсов.

    \item Далее производится предварительная подготовка выбранных материалов.
    На этом шаге данные преобразуются в машиночитаемый вид, пригодный для
    анализа с использованием выбранной технологии. Сюда входит фильтрация
    нерелевантной информации, выявление и устранение ошибок, анализ пропущенных
    значений, а также оптимизация объёма данных. Эти действия направлены на
    повышение эффективности и точности последующего анализа.

    \item После подготовки данные размещаются на выбранной вычислительной
    платформе. Это может быть как локальное устройство, так и
    специализированное хранилище или облачная среда, в зависимости от методов и
    объёма анализа. Стоит отметить, что данный шаг не всегда обязателен: при
    определённых условиях возможно выполнение операций ТДМ без создания
    отдельной базы данных, используя, например, программные интерфейсы (API).

    \item На следующем этапе осуществляется непосредственно анализ данных
    (mining), в ходе которого применяются различные аналитические алгоритмы,
    включая методы классификации, регрессии, кластеризации и обобщения.
    Основная цель этого шага --- извлечение скрытых взаимосвязей и полезных
    паттернов из подготовленного массива данных.

    \item Заключительная стадия включает интерпретацию полученных результатов.
    Здесь производится анализ выявленных закономерностей, формулируются выводы,
    составляется итоговый отчёт. Как правило, такой документ не содержит
    исходных или преобразованных данных и не представляет собой краткое
    изложение проанализированных материалов.
\end{enumerate}

При необходимости возможно возвращение к предыдущим этапам для корректировки
процесса --- например, исключения нерелевантных данных или повторной настройки
алгоритма.

