\conclusion

Правовое регулирование интеллектуальной собственности в контексте
автоматизированной обработки данных, включая технологии машинного обучения и
интеллектуального анализа текста и данных (TDM), нуждается в системной
модернизации. Применение TDM может затрагивать различные объекты авторского и
смежного права — от отдельных произведений до сложных структурированных баз
данных. В ряде случаев такие действия квалифицируются как воспроизведение,
переработка или извлечение, что требует наличия правовых оснований.

Российское законодательство предоставляет отдельные исключения для
использования объектов интеллектуальной собственности, однако они, как правило,
сформулированы узко и фрагментарно. В отличие от европейской модели, в
национальном праве отсутствует комплексная правовая конструкция, специально
направленная на урегулирование вопросов, связанных с TDM и использованием
данных в целях машинного обучения.

Проведённый анализ показывает, что существующие исключения — например,
воспроизведение в личных, научных и образовательных целях, а также
использование несущественных частей базы данных (п. 1 ст. 1335.1 ГК РФ) — могут
применяться в отдельных случаях. Однако они не обеспечивают правовую
определённость и предсказуемость для участников цифрового оборота, включая
научные учреждения, разработчиков ИИ и владельцев данных.

В целях содействия развитию технологий искусственного интеллекта, цифровой
экономики и исследовательской деятельности представляется обоснованным
адаптировать российское законодательство к современным условиям. Наиболее
эффективным подходом может стать внедрение дифференцированной модели,
аналогичной практике Европейского союза, где предусматривается свободное
использование обнародованных произведений научными учреждениями при сохранении
возможности правообладателей ограничивать такое использование.

Рекомендованные изменения включают:

\begin{itemize}
  \item Дополнить ст. 1274 ГК РФ правом научных организаций и учреждений
  культуры без согласия правообладателя и без выплаты вознаграждения
  использовать правомерно обнародованные произведения для автоматизированной
  обработки с помощью ЭВМ при проведении исследований без цели извлечения
  прибыли (данное положение также будет распространяться на объекты смежных
  прав в соответствии со статьей 1306 ГК РФ);
  \item Дополнить главу 70 ГК РФ новой статьей «Использование правомерно
  обнародованных произведений при автоматизированной обработке с помощью ЭВМ»
  следующего содержания: «Использование правомерно обнародованных произведений
  при автоматизированной обработке с помощью ЭВМ, если такое использование не
  противоречит обычному использованию произведений и не ущемляет
  необоснованным образом законные интересы правообладателей, допускается без
  согласия правообладателя и без выплаты вознаграждения, кроме случаев, когда
  правообладатель установил запрет или ограничения на такое использование (в
  том числе посредством технических средств защиты авторских прав)».
  Аналогичную статью следует предусмотреть в отношении объектов смежных прав
  в главе 71 ГК РФ «Права, смежные с авторскими»;
  \item В целях защиты интересов авторов (правообладателей) оригинальных
  произведений, используемых в машинном обучении, в новую статью об
  использовании произведений при автоматизированной обработке с помощью ЭВМ
  целесообразно добавить следующее положение: «Создание производных
  произведений в результате автоматизированной обработки оригинальных
  произведений, не связанное исключительно с технической переработкой
  оригинального произведения в машиночитаемый формат, допускается с соблюдением
  прав авторов оригинальных произведений»
  \item Для внесения определенности в вопросы правовой охраны баз данных как
  объектов смежных прав следует внести в статью 1334 ГК РФ уточнение о
  характере финансовых, материальных, организационных или иных затрат,
  понесенных на изготовление баз данных. Рекомендуется учитывать существенные
  затраты не только непосредственно на создание базы данных, но и на
  деятельность, способствующую созданию базы данных. Такой широкий подход к
  распространению правовой охраны на базы данных будет компенсирован
  исключениями, предусматривающими более свободное использование объектов
  интеллектуальной собственности в машинном обучении;
  \item Для внесения определенности в вопрос квалификации правомерно
  обнародованного произведения и базы данных, используемых в целях машинного
  обучения, целесообразно добавить уточнение в соответствующие статьи ГК РФ
  (ст. 1274 и другие), что под правомерно обнародованным произведением (базой
  данных) понимается произведение (база данных), доступ к которому возможен
  без нарушения закона (включая доступ неограниченного круга лиц в сети
  Интернет).
\end{itemize}

Принятие таких норм обеспечит баланс интересов исследователей и
правообладателей, будет способствовать формированию правовой среды,
стимулирующей как научную, так и предпринимательскую активность в сфере больших
данных и машинного обучения.
