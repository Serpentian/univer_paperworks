\newglossaryentry{id0}{
    name={\\ База данных},
    description={
        представленная в объективной форме совокупность самостоятельных
        материалов (статей, расчетов, нормативных актов, судебных решений и
        иных подобных материалов), систематизированных таким образом, чтобы эти
        материалы могли быть найдены и обработаны с помощью электронной
        вычислительной машины (ЭВМ).
    }
}

\newglossaryentry{id1}{
    name={\\ Большие данные (Big Data)},
    description={
        массив данных, созданный с высокой скоростью с помощью значительного
        количества различных источников. Данные могут быть созданы людьми или
        сгенерированы машиной, например датчиками, собирающими климатическую
        информацию, спутниками, системами GPS и т.д.
    }
}

\newglossaryentry{id2}{
    name={\\ Машинное обучение},
    description={
        процесс, реализующий вычислительные методы, которые предоставляют
        системам возможность обучаться на данных или на основе опыта.
    }
}

\newglossaryentry{id3}{
    name={\\ Майнинг данных},
    description={
        совокупность методов исследования (включая методы машинного обучения),
        связанных со сбором и последующей обработкой большого количества данных
        с помощью автоматизированных программных инструментов с целью
        обнаружения в данных новых знаний, необходимых для принятия решений в
        различных сферах человеческой деятельности.
    }
}

\newglossaryentry{id4}{
    name={\\ TDM (text and data mining)},
    description={
        майнинг текста и данных (в терминологии Директивы ЕС об авторском праве
        в цифровой среде).
    }
}
