\section{Система контроля исполнения политик безопасности рабочих станций}

Система контроля исполнения политик безопасности рабочих станций (Endpoint
Security Policy Enforcement) — это инструмент, обеспечивающий мониторинг и
контроль соблюдения установленных политик безопасности на рабочих станциях
пользователей, включая обновления ПО, антивирусную защиту, настройки ОС и
конфигурации безопасности.

Данная система является ключевой частью инфраструктуры информационной
безопасности организации, поскольку предотвращает нарушения политик, снижает
риски компрометации конечных устройств и обеспечивает соответствие
внутренним и внешним нормативным требованиям.

\subsection{Границы и контекст предметной области}

Границы системы определяют функции и компоненты, которые контролируются
непосредственно системой, а контекст показывает её взаимодействие с
пользователями, ИТ-персоналом и внешними сервисами.

\subsubsection{Заинтересованные стороны}

\begin{table}
    \caption{Заинтересованные стороны системы контроля политик безопасности}
    \begin{tabularx}{\textwidth}{|X|X|X|}\hline
    Заинтересованная сторона & Потребности & Цели \\ \hline
    Руководство &
    Отчёты о соблюдении политик, метрики рисков &
    Снижение угроз, принятие управленческих решений \\ \hhline{---}

    ИТ / Системные администраторы &
    Удалённое управление политиками, отчёты об отклонениях &
    Поддержка рабочих станций в безопасном состоянии \\ \hhline{---}

    Служба информационной безопасности &
    Контроль выполнения политик, выявление нарушений &
    Снижение рисков утечек и инцидентов \\ \hhline{---}

    SOC / Операционные аналитики &
    События безопасности с рабочих станций &
    Реагирование на инциденты \\ \hhline{---}

    Пользователи &
    Сообщения о нарушениях политик, рекомендации &
    Обеспечение безопасной работы рабочих станций \\ \hhline{---}

    Аудит / Комплаенс &
    История выполнения политик, отчёты &
    Соответствие стандартам и внутренним регламентам \\ \hline
    \end{tabularx}
    \label{tab:stakeholders-ep}
\end{table}

\subsubsection{Границы системы}

Внутренние компоненты системы:

\begin{itemize}
    \item Агент контроля на рабочей станции.
    \item Централизованная консоль управления политиками.
    \item Модуль анализа соответствия политик.
    \item Подсистема уведомлений и отчётности.
    \item Интерфейсы для администраторов и аудиторов.
    \item Подсистема хранения логов и конфигураций.
\end{itemize}

Внешние элементы:

\begin{itemize}
    \item Рабочие станции и пользователи.
    \item Сервера обновлений и сторонние сервисы безопасности.
    \item Регуляторы и внутренние аудиторы.
\end{itemize}

\subsubsection{Внешние интерфейсы}

\begin{table}[h!]
    \caption{Основные внешние интерфейсы системы контроля политик}
    \begin{tabularx}{\textwidth}{|X|X|X|}\hline
    Внешний объект / система & Тип интерфейса & Назначение \\ \hline
    Рабочие станции & Агент, API, SNMP & Контроль и отчёт о соблюдении политик \\ \hhline{---}
    Сервера обновлений & API, пакетные менеджеры & Проверка и установка обновлений \\ \hhline{---}
    Антивирус и защитное ПО & API, интеграционные коннекторы & Проверка статуса защиты \\ \hhline{---}
    ИТ-администраторы & Веб-консоль, отчёты & Управление политиками и инцидентами \\ \hhline{---}
    Аудиторы / регуляторы & Экспорт отчётов, CSV/PDF & Проверка соблюдения требований \\ \hline
    \end{tabularx}
    \label{tab:external-interfaces-ep}
\end{table}

\subsection{Терминологический и объектный анализ предметной области}

\begin{description}
    \item[Рабочая станция] Компьютер пользователя, на котором должны соблюдаться
    политики безопасности.
    \item[Политика безопасности] Правила и требования к конфигурации системы,
    ПО и поведению пользователя.
    \item[Агент контроля] Программный компонент, обеспечивающий мониторинг
    и исполнение политик на рабочей станции.
    \item[Нарушение политики] Событие, когда рабочая станция не соответствует
    установленной политике.
    \item[Отчёт] Документ о состоянии соответствия политик, генерируемый системой.
    \item[Администратор] Пользователь, управляющий политиками и контролирующий
    соблюдение.
\end{description}

\subsubsection{Атрибуты сущностей}

\begin{itemize}
    \item \textbf{Рабочая станция:} идентификатор, статус соответствия,
    установленное ПО, конфигурация безопасности.
    \item \textbf{Политика безопасности:} уникальный идентификатор, тип политики,
    требования, приоритет.
    \item \textbf{Агент контроля:} версия, состояние, частота проверок,
    результаты сканирования.
    \item \textbf{Нарушение политики:} тип, уровень критичности, дата и время,
    идентификатор рабочей станции.
    \item \textbf{Отчёт:} период, включаемые нарушения, ответственный администратор,
    формат.
\end{itemize}

\subsubsection{Бизнес-правила}

\begin{enumerate}
    \item Каждая рабочая станция должна иметь установленного агента контроля.
    \item Агент периодически проверяет соблюдение всех политик.
    \item Нарушения фиксируются и передаются в централизованную консоль.
    \item Администратор обязан реагировать на критические нарушения.
    \item Отчёты генерируются автоматически в установленные сроки.
    \item История нарушений хранится в течение периода, соответствующего требованиям.
\end{enumerate}

\subsection{Структурная модель предметной области}

\subsubsection{UML-диаграмма классов}

Ниже приведена UML-диаграмма классов, отражающая основные сущности предметной
области системы контроля исполнения политик безопасности рабочих станций,
связи между ними, кардинальности и ключевые ограничения. Диаграмма предназначена
для согласования модели данных и границ ответственности компонентов системы.

\begin{figure}
  \centering
  \includegraphics[scale=0.5]{inc/uml-5.png}
  \caption{UML-диаграмма классов системы контроля исполнения политик безопасности
рабочих станций}
\end{figure}

\subsubsection{Пояснение к элементам диаграммы}

\begin{itemize}
    \item \textbf{Workstation (Рабочая станция)} — конечное устройство, на котором
    должны соблюдаться политики безопасности; каждая рабочая станция имеет одного
    агента контроля.
    \item \textbf{Agent (Агент контроля)} — устанавливается на рабочей станции,
    выполняет проверки соответствия политикам и фиксирует нарушения.
    \item \textbf{Policy (Политика безопасности)} — правила и требования к
    конфигурации, ПО и поведению пользователя; нарушения фиксируются как
    \texttt{Violation}.
    \item \textbf{Violation (Нарушение политики)} — событие, фиксирующее
    несоответствие политики; может быть обнаружено агентом и включено в отчёт.
    \item \textbf{Admin (Администратор)} — управляет политиками, реагирует
    на критические нарушения и формирует отчёты.
    \item \textbf{Report (Отчёт)} — документ о состоянии соблюдения политик;
    содержит информацию о выявленных нарушениях и состоянии рабочих станций.
\end{itemize}

\subsubsection{Основные ограничения и бизнес-правила}

\begin{enumerate}
    \item \textbf{Обязательная установка агента:} каждая рабочая станция должна
    иметь установленного агента контроля.
    \item \textbf{Регулярная проверка политик:} агент периодически сканирует
    рабочую станцию на соответствие всем политикам.
    \item \textbf{Фиксация нарушений:} все несоответствия фиксируются как
    \texttt{Violation} и передаются в централизованную консоль.
    \item \textbf{Реакция на критические нарушения:} администратор обязан
    реагировать на нарушения с высоким уровнем критичности.
    \item \textbf{Генерация отчётов:} отчёты создаются автоматически или по запросу
    администратора, включают сведения о выявленных нарушениях.
    \item \textbf{Хранение истории:} данные о нарушениях сохраняются в течение
    установленного периода для аудита и комплаенса.
\end{enumerate}

\subsection{Модель поведения системы}

Модель поведения системы отражает динамическое взаимодействие сущностей и
пользователей системы контроля исполнения политик безопасности рабочих станций.
Она позволяет формализовать сценарии взаимодействия, понять бизнес-процессы
и определить жизненные циклы ключевых объектов.

\subsubsection{Диаграмма прецедентов (Use Cases)}

Диаграмма прецедентов показывает всех акторов системы и основные сценарии
взаимодействия с системой контроля политик. Акторы включают: пользователей
(администраторов, аудиторов), рабочие станции, агенты контроля, внешние сервисы
обновлений и регуляторов.

\begin{figure}
  \centering
  \includegraphics[scale=0.38]{inc/uml-6.png}
\caption{Диаграмма прецедентов системы контроля исполнения политик безопасности рабочих станций}
\end{figure}

\subsubsection{Диаграмма деятельности (Activity Diagram)}

Пример ключевого бизнес-процесса: проверка соблюдения политики и обработка
нарушения на рабочей станции.

\begin{figure}
  \centering
  \includegraphics[scale=0.6]{inc/uml-7.png}
\caption{Диаграмма деятельности системы контроля исполнения политик безопасности рабочих станций}
\end{figure}

\subsubsection{Диаграмма состояний (State Diagram)}

Пример состояния для сущности \texttt{Violation}, которая имеет жизненный цикл:
обнаружение, реакция, подтверждение и закрытие.

\begin{figure}
  \centering
  \includegraphics[scale=0.5]{inc/uml-8.png}
\caption{Диаграмма состояний сущности \texttt{Violation} в системе контроля исполнения политик безопасности}
\end{figure}

\subsubsection{Пояснения}

\begin{itemize}
    \item \textbf{Диаграмма прецедентов} показывает акторов системы и функции,
    которые они используют, включая автоматизированные проверки и уведомления.
    \item \textbf{Диаграмма деятельности} иллюстрирует процесс проверки политики,
    фиксации нарушений и уведомления администратора.
    \item \textbf{Диаграмма состояний} описывает жизненный цикл нарушения
    (\texttt{Violation}) с переходами: \texttt{Detected} → \texttt{Notified} →
    \texttt{InProgress} → \texttt{Resolved} → \texttt{Closed}.
\end{itemize}

\subsection{Верификация модели предметной области}

Верификация модели предметной области необходима для обеспечения её
корректности, полноты и соответствия требованиям заинтересованных сторон.
Процесс верификации позволяет выявить противоречия, ошибки терминологии и
несоответствия стандартам проектирования. В контексте системы контроля исполнения
политик безопасности рабочих станций проверка проводится по следующим критериям.

\subsubsection{Критерии верификации}

\begin{enumerate}
    \item \textbf{Непротиворечивость}

    Все элементы модели должны быть логически согласованы:
    \begin{itemize}
        \item отсутствуют противоречивые определения сущностей и атрибутов;
        \item кардинальности и связи между классами не конфликтуют;
        \item бизнес-правила не содержат взаимно исключающих условий.
    \end{itemize}

    \item \textbf{Полнота}

    Модель должна охватывать всю предметную область:
    \begin{itemize}
        \item учтены все ключевые сущности, процессы и взаимодействия;
        \item все прецеденты использования и активности включены;
        \item жизненные циклы основных объектов описаны.
    \end{itemize}

    \item \textbf{Соответствие требованиям заинтересованных сторон}

    Проверка выполняется на основе анализа таблицы \ref{tab:stakeholders-ep}:
    \begin{itemize}
        \item все потребности и цели акторов реализованы в модели;
        \item каждый прецедент и процесс обеспечивает удовлетворение требований;
        \item предусмотрены интерфейсы для внешних систем, регуляторов и аудиторов.
    \end{itemize}

    \item \textbf{Корректность терминологического аппарата (ISO 704)}

    Проверяется единообразие и точность терминов:
    \begin{itemize}
        \item определения сущностей соответствуют правилам ISO 704 (родовое
        понятие + отличительные признаки);
        \item нет дублирования терминов или неоднозначных формулировок;
        \item атрибуты и свойства корректно связаны с сущностями.
    \end{itemize}

    \item \textbf{Соответствие процессам ISO/IEC/IEEE 42010}

    Модель проверяется на архитектурное соответствие:
    \begin{itemize}
        \item архитектурные виды (viewpoints) и представления (views) соответствуют
        требованиям стандартов;
        \item диаграммы классов, прецедентов, деятельности и состояний отражают
        архитектурные решения;
        \item соблюдены принципы трассируемости между требованиями, бизнес-правилами
        и элементами модели.
    \end{itemize}
\end{enumerate}

\subsubsection{Выводы по верификации}

Проведённая проверка модели предметной области показала:

\begin{itemize}
    \item Модель системы контроля исполнения политик безопасности рабочих станций
    является непротиворечивой и логически согласованной.
    \item Все ключевые сущности, процессы и сценарии использования включены,
    обеспечивая полноту модели.
    \item Потребности заинтересованных сторон реализованы через прецеденты
    использования и функциональные компоненты системы.
    \item Терминология согласована с ISO 704, определения и атрибуты корректны.
    \item Диаграммы и архитектурные представления соответствуют стандартам
    ISO/IEC/IEEE 42010.
\end{itemize}

Таким образом, модель предметной области прошла верификацию и может быть
использована для дальнейшего проектирования системы контроля исполнения политик
безопасности рабочих станций и её архитектуры.

\subsection{Заключение}

Настоящий отчёт содержит результаты проектирования предметной области
системы контроля исполнения политик безопасности рабочих станций. Проектирование
выполнено с учётом требований ГОСТ Р 7.0.97–2016, ГОСТ Р 59793–2021,
ISO/IEC/IEEE 15288, ISO/IEC 12207 и ISO/IEC/IEEE 42010.

Архитектурный документ по ISO/IEC/IEEE 42010 включает следующие компоненты:

\begin{itemize}
    \item \textbf{Stakeholders:} руководство организации, ИТ-администраторы,
    служба информационной безопасности, SOC, пользователи, аудиторы и внешние
    сервисы.
    \item \textbf{Concerns:} соблюдение политик безопасности на рабочих станциях,
    своевременное выявление нарушений, защита конечных устройств,
    соответствие требованиям регуляторов, доступность и надёжность системы.
    \item \textbf{Viewpoints:} структурная (UML class), поведенческая (activity,
    use case), информационная, эксплуатационная.
    \item \textbf{Views:} UML-диаграммы классов, диаграммы прецедентов,
    диаграммы деятельности, диаграммы состояний, словарь сущностей,
    описание бизнес-процессов.
    \item \textbf{Correspondence rules:} соответствие бизнес-правил моделям,
    непротиворечивость связей, корректность терминологии.
    \item \textbf{Rationale:} архитектура обеспечивает контроль исполнения политик,
    прозрачность процессов, своевременное реагирование на нарушения,
    накопление доказательной базы и автоматизацию формирования отчётов.
\end{itemize}

В ходе проектирования были выполнены:

\begin{enumerate}
    \item Анализ заинтересованных сторон и их потребностей, определение границ
    и контекста системы.
    \item Терминологический и объектный анализ предметной области, выделение
    сущностей, атрибутов и бизнес-правил.
    \item Построение структурной модели системы с использованием UML-диаграммы
    классов, отражающей сущности, их связи, кардинальности и ограничения.
    \item Моделирование поведения системы: диаграммы прецедентов, деятельности
    и состояний.
    \item Верификация модели на непротиворечивость, полноту, соответствие
    требованиям заинтересованных сторон, корректность терминологии и соответствие
    стандартам ISO/IEC/IEEE 42010.
\end{enumerate}

Проверка модели показала её непротиворечивость, полноту и соответствие
архитектурным требованиям. Терминологический аппарат соответствует ISO 704,
все бизнес-правила корректно отражены в модели, а диаграммы обеспечивают
трассируемость между требованиями, процессами и объектами системы.

Таким образом, созданная модель предметной области является надежной основой
для дальнейшего проектирования системы контроля исполнения политик безопасности
рабочих станций, её архитектуры, бизнес-процессов и реализации компонентов.
Архитектура обеспечивает прозрачность процессов, эффективность реагирования
на нарушения и соблюдение нормативных требований организации.
