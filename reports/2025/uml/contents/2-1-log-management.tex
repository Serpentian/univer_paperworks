\structure{ОСНОВНАЯ ЧАСТЬ}

\section{Система мониторинга журналов безопасности (лог-менеджмент)}

Система мониторинга журналов безопасности (лог-менеджмент) — это инструмент,
который централизованно собирает, хранит и анализирует журналы событий из
разных источников, помогает выявлять инциденты, отслеживать активность,
обеспечивать соответствие требованиям и ускорять расследование угроз.

Данная система является важной частью инфраструктуры информационной
безопасности организации, поскольку обеспечивает прозрачность процессов,
выявление аномалий, фиксацию действий пользователей и служб, а также служит
основой для оперативного и ретроспективного анализа происходящих событий.
Контекст предметной области включает управление информационными рисками,
обеспечение непрерывности работы, поддержание соответствия нормативным
требованиям и повышение уровня защищённости.

\subsection{Границы и контекст предметной области}

Границы системы определяют, какие функции и компоненты входят в лог-менеджмент
и за что система отвечает непосредственно. Контекст предметной области
показывает окружение, в рамках которого эта система функционирует, а также
описывает связи с внешними элементами — источниками данных, пользователями,
службами реагирования и регуляторами.

\subsubsection{Заинтересованные стороны}

В таблице \ref{tab:stakeholders} представлены ключевые заинтересованные
стороны, которые взаимодействуют с системой, а также их основные потребности и
цели.

\begin{table}
    \caption{Заинтересованные стороны системы лог-менеджмента}
    \begin{tabularx}{\textwidth}{|X|X|X|}\hline
    Заинтересованная сторона & Потребности & Цели \\ \hline
    Руководство &
    Краткие отчёты, метрики рисков, видимость инцидентов &
    Управленческие решения, снижение рисков \\ \hhline{---}

    ИТ / Системные администраторы &
    Централизованный сбор логов, простая интеграция, доступность данных &
    Стабильная работа инфраструктуры, быстрая диагностика \\ \hhline{---}

    Служба информационной безопасности &
    Полные журналы, корреляция событий, алерты &
    Раннее выявление угроз, расследование инцидентов \\ \hhline{---}

    SOC / Операционные аналитики &
    Данные в реальном времени, приоритизация, удобные панели &
    Сокращение времени реакции на инциденты \\ \hhline{---}

    Разработчики &
    Доступ к логам приложений, структурированные данные &
    Быстрое устранение ошибок, повышение качества ПО \\ \hhline{---}

    Аудит / Комплаенс &
    Контроль целостности логов, долгосрочное хранение, отчётность &
    Соответствие стандартам и требованиям регуляторов \\ \hline
    \end{tabularx}
    \label{tab:stakeholders}
\end{table}

\subsubsection{Границы системы}

Границы определяют, какие функциональные возможности предоставляет система, а
какие задачи выполняются внешними элементами или другими системами. Внутренние
компоненты лог-менеджмента включают:

\begin{itemize}
    \item Модуль сбора логов (агенты, коннекторы).
    \item Централизованное хранилище журналов.
    \item Механизм нормализации и корреляции событий.
    \item Подсистема оповещений и отчётности.
    \item Интерфейсы визуализации (дашборды, поиск по логам).
    \item Механизмы аутентификации и авторизации пользователей.
    \item Подсистема резервного копирования и архивирования логов.
\end{itemize}

За пределами системы остаются те элементы, которые взаимодействуют с
лог-менеджментом, но не управляются им напрямую:

\begin{itemize}
    \item Источники логов (серверы, приложения, сетевое оборудование, СЗИ).
    \item Внешние системы реагирования и тикетирования.
    \item Пользователи и персонал (администраторы, SOC, аудиторы).
    \item Внешние регуляторы и требования стандартизации.
\end{itemize}

Таким образом формируются функциональные и организационные границы, в рамках
которых система обеспечивает свою работу.

\subsubsection{Внешние интерфейсы}

Внешние интерфейсы определяют способы взаимодействия лог-менеджмента с
окружением, обеспечивая передачу данных, получение уведомлений, формирование
отчётов и интеграцию с другими сервисами. Они приведены в таблице
\ref{tab:external-interfaces}.

\begin{table}[h!]
    \caption{Основные внешние интерфейсы системы лог-менеджмента}
    \begin{tabularx}{\textwidth}{|X|X|X|}\hline
    Внешний объект / система & Тип интерфейса & Назначение \\ \hline

    Источники логов (серверы, сетевые устройства, приложения) &
    Syslog, API, агенты, файловые коллекторы &
    Передача событий и журналов в систему лог-менеджмента \\ \hhline{---}

    Системы ИБ (IDS/IPS, WAF, антивирус, DLP) &
    Syslog, REST API, интеграционные коннекторы &
    Получение событий безопасности для анализа и корреляции \\ \hhline{---}

    Система управления инцидентами (ITSM, SOAR) &
    REST API, вебхуки &
    Экспорт алертов и автоматизация реакции \\ \hhline{---}

    Пользователи (SOC, администраторы, аудиторы) &
    Веб-интерфейс, ролевой доступ, отчёты &
    Просмотр логов, расследование, анализ, управление настройками \\ \hhline{---}

    Внешние регуляторы (при проверках) &
    Экспорт отчётов, выгрузка архивов &
    Предоставление доказательной базы и журналов \\ \hline

    \end{tabularx}
    \label{tab:external-interfaces}
\end{table}

\subsection{Терминологический и объектный анализ предметной области}

Терминологический и объектный анализ позволяет формализовать ключевые понятия
предметной области лог-менеджмента, определить сущности, их характеристики и
взаимосвязи. Это обеспечивает единое понятийное пространство для дальнейшего
проектирования системы, уменьшает неоднозначность интерпретаций и служит основой
для построения модели данных и бизнес-процессов.

\subsubsection{Выделение сущностей и их определения}

В соответствии с требованиями стандарта ISO 704 определения приводятся через
ближайшее родовое понятие с указанием отличительных признаков. Ниже
сформирован перечень сущностей, используемых в системе мониторинга журналов
безопасности.

\begin{description}
    \item[Журнал события (лог)] Документированная запись о состоянии, действии
    или событии, зафиксированная информационной системой в определённый момент
    времени.

    \item[Событие безопасности] Факт, отражающий действие или изменение
    состояния информационной системы, имеющее значение для оценки её защищённости.

    \item[Источник логов] Система или компонент, генерирующий журналы событий
    и передающий их в лог-менеджмент.

    \item[Агент сбора логов] Программный компонент, обеспечивающий получение,
    преобразование и передачу журналов от источника логов в центральную систему.

    \item[Хранилище логов] Централизованный компонент, обеспечивающий долговременное
    хранение, поиск и доступ к журналам событий.

    \item[Нормализация событий] Процесс приведения записей логов к единому
    структурированному формату для унификации анализа.

    \item[Корреляция событий] Процесс установления взаимосвязей между
    разрозненными событиями с целью выявления аномалий или инцидентов.

    \item[Пользователь системы] Лицо или роль, имеющая доступ к функциям
    системы лог-менеджмента.

    \item[Инцидент информационной безопасности] Событие или совокупность
    событий, нарушающих или потенциально нарушающих конфиденциальность,
    целостность или доступность информации.

    \item[Оповещение (алерт)] Автоматически сформированное уведомление о
    событиях, требующих внимания.

    \item[Отчёт] Формализованный документ, описывающий результаты анализа,
    статистику и выводы на основе журналов событий.
\end{description}

\subsubsection{Атрибуты сущностей}

Для каждой сущности определены ключевые атрибуты, необходимые для её
представления в системе.

\begin{itemize}
    \item \textbf{Журнал события (лог):}
    \begin{itemize}
        \item время события;
        \item идентификатор источника;
        \item тип события;
        \item уровень важности;
        \item текст сообщения;
        \item структурированные поля (IP, пользователь, процесс и др.).
    \end{itemize}

    \item \textbf{Событие безопасности:}
    \begin{itemize}
        \item категория;
        \item критичность;
        \item контекст (объект, субъект);
        \item статус (нормальное, подозрительное, инцидент).
    \end{itemize}

    \item \textbf{Источник логов:}
    \begin{itemize}
        \item тип (сервер, приложение, устройство);
        \item уникальный идентификатор;
        \item протокол передачи;
        \item частота генерации событий.
    \end{itemize}

    \item \textbf{Агент сбора логов:}
    \begin{itemize}
        \item версия;
        \item поддерживаемые форматы;
        \item метод передачи;
        \item состояние (активен/недоступен).
    \end{itemize}

    \item \textbf{Хранилище логов:}
    \begin{itemize}
        \item объём хранения;
        \item срок ретенции;
        \item механизмы шифрования;
        \item политика доступа.
    \end{itemize}

    \item \textbf{Нормализованное событие:}
    \begin{itemize}
        \item формат записи;
        \item значения нормализованных полей;
        \item признак полноты.
    \end{itemize}

    \item \textbf{Коррелированное событие:}
    \begin{itemize}
        \item список исходных событий;
        \item правило корреляции;
        \item выявленный сценарий;
        \item приоритизация.
    \end{itemize}

    \item \textbf{Пользователь системы:}
    \begin{itemize}
        \item роль;
        \item уровень доступа;
        \item статус учётной записи.
    \end{itemize}

    \item \textbf{Алерт:}
    \begin{itemize}
        \item тип триггера;
        \item серьёзность;
        \item метаданные (время, источник);
        \item статус обработки.
    \end{itemize}

    \item \textbf{Отчёт:}
    \begin{itemize}
        \item тип отчёта;
        \item период анализа;
        \item ответственный;
        \item формат (PDF, HTML, CSV).
    \end{itemize}
\end{itemize}

\subsubsection{Бизнес-правила, регулирующие поведение сущностей}

Данный раздел определяет нормативные ограничения и логику работы системы.

\begin{enumerate}
    \item Каждый источник логов должен иметь уникальный идентификатор и
    передавать журналы согласно установленному регламенту.

    \item Все журналы событий должны содержать корректное время,
    синхронизированное по NTP.

    \item Передача логов от агентов должна выполняться в защищённом виде
    (шифрование, аутентификация).

    \item Все входящие события проходят нормализацию перед помещением в
    хранилище.

    \item Корреляционные правила должны быть документированы, версионированы
    и регулярно пересматриваться.

    \item Каждое коррелированное событие должно содержать ссылки на свои
    исходные события.

    \item Алерты формируются только при выполнении условий, определённых
    правилами корреляции или детекции.

    \item Пользователь выполняет действия только в пределах своей роли и
    прав доступа.

    \item Хранилище логов обеспечивает неизменность данных в течение срока
    ретенции, соответствующего требованиям регуляторов.

    \item Отчёты формируются только на основе проверенных и нормализованных
    данных.

    \item Любые изменения конфигурации системы фиксируются в журнале
    административных действий.
\end{enumerate}

\subsection{Структурная модель предметной области}

\subsubsection{UML-диаграмма классов}

Ниже приведена UML-диаграмма классов, отражающая основные сущности предметной
области лог-менеджмента, связи между ними, кардинальности и ключевые ограничения.
Диаграмма предназначена для использования как средство согласования модели
данных и границ ответственности компонентов системы.

\begin{figure}
  \centering
  \includegraphics[scale=0.33]{inc/uml-1.png}
  \caption{UML-диаграмма классов системы лог-менеджмента}
\end{figure}

\subsubsection{Пояснение к элементам диаграммы}

\begin{itemize}
    \item \textbf{Source (Источник логов)} — генерирует \texttt{LogEntry}; для
    каждого источника задан уникальный \texttt{sourceId}. Один источник может
    иметь ноль, одну или несколько установленных сущностей \texttt{Agent}
    (в зависимости от архитектуры — агент на хосте или централизованный коллектор).
    \item \textbf{Agent (Агент сбора)} — собирает записи событий и передаёт
    их в систему; один агент может собирать множество \texttt{LogEntry}.
    \item \textbf{LogEntry (Журнал события)} — исходная запись; согласно
    бизнес-правилу все входящие записи должны иметь корректный NTP-таймстемп
    и проходить нормализацию.
    \item \textbf{NormalizedEvent (Нормализованное событие)} — результат
    приведения \texttt{LogEntry} к единому формату; связь 1..1 с \texttt{LogEntry}
    (обязательно) в модели, т.к. нормализация равна требованию.
    \item \textbf{CorrelatedEvent (Коррелированное событие)} — объединяет
    несколько нормализованных событий по правилу корреляции; одно коррелированное
    событие агрегирует 1..* \texttt{NormalizedEvent}.
    \item \textbf{Alert (Алерт)} — создаётся на основе коррелированных событий;
    алерты могут приводить к созданию \texttt{Incident}.
    \item \textbf{Incident (Инцидент)} — следствие подтверждённых алертов,
    управляется процессом реагирования.
    \item \textbf{LogStorage (Хранилище)} — хранит исходные и/или нормализованные
    события, обеспечивает ретеншн, шифрование и неизменность в рамках
    регламентированных сроков.
    \item \textbf{User (Пользователь)} и \textbf{Report (Отчёт)} — пользователи
    создают отчёты и обрабатывают алерты; права операций ограничены ролями.
\end{itemize}

\subsubsection{Основные ограничения и бизнес-правила}

\begin{enumerate}
    \item \textbf{Нормализация обязательна:} каждое поступившее событие должно
    иметь соответствующий \texttt{NormalizedEvent}. (см. диаграмму: \texttt{LogEntry 1..1 NormalizedEvent})
    \item \textbf{Синхронизация времени:} все записи обязаны иметь корректный
    временной штамп, синхронизированный по NTP.
    \item \textbf{Аутентификация и шифрование:} передача от агентов в систему
    должна выполняться в защищённом виде.
    \item \textbf{Трассируемость:} каждое \texttt{CorrelatedEvent} содержит
    ссылки на все исходные \texttt{NormalizedEvent}.
    \item \textbf{Неизменность:} данные в \texttt{LogStorage} являются
    неизменяемыми в период ретенции.
    \item \textbf{Ограничения доступа:} операции над алертами, инцидентами и
    отчётами доступны в соответствии с ролью и уровнем доступа пользователя.
\end{enumerate}

\subsection{Модель поведения системы}

Модель поведения системы отражает динамическое взаимодействие сущностей и
пользователей системы лог-менеджмента. Она позволяет формализовать сценарии
взаимодействия, понять бизнес-процессы и определить жизненные циклы ключевых
объектов.

\subsubsection{Диаграмма прецедентов (Use Cases)}

Диаграмма прецедентов показывает всех акторов системы и основные сценарии их
взаимодействия с системой лог-менеджмента. Акторы включают: пользователей
(SOC, администраторов, аудиторов), внешние источники логов, системы ИБ, системы
управления инцидентами и регуляторов.

\begin{figure}
  \centering
  \includegraphics[scale=0.35]{inc/uml-2.png}
  \caption{Диаграмма прецедентов системы лог-менеджмента}
\end{figure}

\subsubsection{Диаграмма деятельности (Activity Diagram)}

Пример ключевого бизнес-процесса: обработка инцидента на основе
коррелированного события.

\begin{figure}
  \centering
  \includegraphics[scale=0.5]{inc/uml-3.png}
  \caption{Диаграмма деятельности системы лог-менеджмента}
\end{figure}

\subsubsection{Диаграмма состояний (State Diagram)}

Пример состояния для сущности \texttt{Incident}, которая имеет сложный жизненный
цикл: открытие, обработка, закрытие.

\begin{figure}
  \centering
  \includegraphics[scale=0.5]{inc/uml-4.png}
  \caption{Диаграмма состояний системы лог-менеджмента}
\end{figure}

\subsubsection{Пояснения}

\begin{itemize}
    \item \textbf{Диаграмма прецедентов} показывает, кто и какие функции системы
    использует, включая автоматизированные взаимодействия с внешними источниками
    и системами.
    \item \textbf{Диаграмма деятельности} иллюстрирует поток обработки события
    от нормализованного события до создания инцидента и реакции SOC.
    \item \textbf{Диаграмма состояний} описывает жизненный цикл инцидента, включая
    переходы между состояниями: \texttt{New} → \texttt{Assigned} → \texttt{InProgress} →
    \texttt{Resolved} → \texttt{Closed}, с возможностью повторного открытия.
\end{itemize}

\subsection{Верификация модели предметной области}

Верификация модели предметной области необходима для обеспечения её
корректности, полноты и соответствия требованиям заинтересованных сторон.
Процесс верификации позволяет выявить противоречия, ошибки терминологии и
несоответствия стандартам проектирования. В контексте системы мониторинга
журналов безопасности проверка проводится по следующим критериям.

\subsubsection{Критерии верификации}

\begin{enumerate}
    \item \textbf{Непротиворечивость}

    Все элементы модели должны быть логически согласованы:
    \begin{itemize}
        \item отсутствуют противоречивые определения сущностей и атрибутов;
        \item кардинальности и связи между классами не конфликтуют;
        \item бизнес-правила не содержат взаимно исключающих условий.
    \end{itemize}

    \item \textbf{Полнота}

    Модель должна охватывать всю предметную область:
    \begin{itemize}
        \item учтены все ключевые сущности, процессы и взаимодействия;
        \item все прецеденты использования и активности включены;
        \item жизненные циклы основных объектов описаны.
    \end{itemize}

    \item \textbf{Соответствие требованиям заинтересованных сторон}

    Проверка выполняется на основе анализа таблицы \ref{tab:stakeholders}:
    \begin{itemize}
        \item все потребности и цели акторов реализованы в модели;
        \item каждый прецедент и процесс обеспечивает удовлетворение требований;
        \item предусмотрены интерфейсы для внешних систем и регуляторов.
    \end{itemize}

    \item \textbf{Корректность терминологического аппарата (ISO 704)}

    Проверяется единообразие и точность терминов:
    \begin{itemize}
        \item определения сущностей соответствуют правилам ISO 704 (родовое
        понятие + отличительные признаки);
        \item нет дублирования терминов или неоднозначных формулировок;
        \item атрибуты и свойства корректно связаны с сущностями.
    \end{itemize}

    \item \textbf{Соответствие процессам ISO/IEC/IEEE 42010}

    Модель проверяется на архитектурное соответствие:
    \begin{itemize}
        \item архитектурные виды (viewpoints) и представления (views) соответствуют
        требованиям стандартов;
        \item диаграммы классов, прецедентов, деятельности и состояний отражают
        архитектурные решения;
        \item соблюдены принципы трассируемости между требованиями, бизнес-правилами
        и элементами модели.
    \end{itemize}
\end{enumerate}

\subsubsection{Выводы по верификации}

Проведённая проверка модели предметной области показала:

\begin{itemize}
    \item Модель лог-менеджмента является непротиворечивой и логически согласованной.
    \item Все ключевые сущности, процессы и сценарии использования включены,
    обеспечивая полноту модели.
    \item Потребности заинтересованных сторон реализованы через прецеденты
    использования и функциональные компоненты системы.
    \item Терминология согласована с ISO 704, определения и атрибуты корректны.
    \item Диаграммы и архитектурные представления соответствуют стандартам
    ISO/IEC/IEEE 42010.
\end{itemize}

Таким образом, модель предметной области прошла верификацию и может быть
использована для дальнейшего проектирования системы лог-менеджмента и её
архитектуры.

\subsection{Заключение}

Настоящий отчёт содержит результаты проектирования предметной области
системы мониторинга журналов безопасности (лог-менеджмент). Проектирование
выполнено с учётом требований ГОСТ Р 7.0.97–2016, ГОСТ Р 59793–2021,
ISO/IEC/IEEE 15288, ISO/IEC 12207 и ISO/IEC/IEEE 42010.

Архитектурный документ по ISO/IEC/IEEE 42010 включает следующие компоненты:

\begin{itemize}
    \item \textbf{Stakeholders:} руководство организации, ИТ-администраторы,
    служба информационной безопасности, SOC, разработчики, аудиторы и внешние
    системы.
    \item \textbf{Concerns:} корректность данных, своевременное выявление инцидентов,
    безопасность и целостность информации, соответствие требованиям регуляторов,
    доступность и надёжность системы.
    \item \textbf{Viewpoints:} структурная (UML class), поведенческая (activity,
    use case), информационная, эксплуатационная.
    \item \textbf{Views:} UML-диаграммы классов, диаграммы прецедентов,
    диаграммы деятельности, диаграммы состояний, словарь сущностей, описание бизнес-процессов.
    \item \textbf{Correspondence rules:} соответствие бизнес-правил моделям,
    непротиворечивость связей, корректность терминологии.
    \item \textbf{Rationale:} архитектура обеспечивает прозрачность процессов,
    своевременное реагирование на инциденты, целостность и сохранность логов,
    автоматизацию обработки событий и формирование отчётов.
\end{itemize}

В ходе проектирования были выполнены:

\begin{enumerate}
    \item Анализ заинтересованных сторон и их потребностей, определение границ
    и контекста системы.
    \item Терминологический и объектный анализ предметной области, выделение
    сущностей, атрибутов и бизнес-правил.
    \item Построение структурной модели системы с использованием UML-диаграммы
    классов, отражающей сущности, их связи, кардинальности и ограничения.
    \item Моделирование поведения системы: диаграммы прецедентов, деятельности
    и состояний.
    \item Верификация модели на непротиворечивость, полноту, соответствие
    требованиям заинтересованных сторон, корректность терминологии и соответствие
    стандартам ISO/IEC/IEEE 42010.
\end{enumerate}

Проверка модели показала её непротиворечивость, полноту и соответствие
архитектурным требованиям. Терминологический аппарат соответствует ISO 704,
все бизнес-правила корректно отражены в модели, а диаграммы обеспечивают
трассируемость между требованиями, процессами и объектами системы.

Таким образом, созданная модель предметной области является надежной основой
для дальнейшего проектирования системы лог-менеджмента, её архитектуры,
бизнес-процессов и реализации компонентов. Архитектура обеспечивает
транспарентность процессов, эффективность реагирования на инциденты и
соблюдение нормативных требований организации.
