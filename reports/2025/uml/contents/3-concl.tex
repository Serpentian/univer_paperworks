\conclusion

В рамках рубежного контроля были спроектированы и верифицированы модели
предметной области двух ключевых систем информационной безопасности
организации:

\begin{enumerate}
    \item \textbf{Система мониторинга журналов безопасности (лог-менеджмент)}, обеспечивающая централизованный сбор, хранение, нормализацию, корреляцию и анализ событий, формирование алертов и инцидентов, а также подготовку отчётов для заинтересованных сторон.
    \item \textbf{Система контроля исполнения политик безопасности рабочих станций}, обеспечивающая мониторинг и контроль соблюдения установленных политик безопасности на конечных устройствах, фиксацию нарушений, уведомления и отчётность.
\end{enumerate}

Проектирование выполнялось с учётом требований ГОСТ Р 7.0.97–2016, ГОСТ Р
59793–2021, ISO/IEC/IEEE 15288, ISO/IEC/IEEE 12207 и ISO/IEC/IEEE 42010.

В ходе работы были выполнены следующие ключевые этапы:

\begin{itemize}
    \item Определение заинтересованных сторон и их требований, формирование контекста и границ систем.
    \item Терминологический и объектный анализ, выделение сущностей, их атрибутов и бизнес-правил.
    \item Построение структурных моделей с использованием UML-диаграмм классов, отражающих связи, кардинальности и ограничения.
    \item Моделирование поведения систем через диаграммы прецедентов, деятельности и состояний.
    \item Верификация моделей на непротиворечивость, полноту, соответствие требованиям заинтересованных сторон, корректность терминологии и соответствие стандартам ISO/IEC/IEEE 42010.
\end{itemize}

Проверка моделей показала их непротиворечивость, полноту и соответствие
архитектурным требованиям. Терминологический аппарат соответствует ISO 704,
бизнес-правила корректно отражены в моделях, а диаграммы обеспечивают
трассируемость между требованиями, процессами и объектами систем.

Таким образом, созданные модели предметной области обеспечивают надёжную основу
для дальнейшего проектирования, внедрения и эксплуатации систем рубежного
контроля. Архитектуры систем гарантируют прозрачность процессов, своевременное
выявление инцидентов и нарушений, а также соблюдение нормативных требований
организации.
