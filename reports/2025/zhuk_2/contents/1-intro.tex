\structure{ВВЕДЕНИЕ}

В современном цифровом мире информация является одним из ключевых
стратегических активов любой организации. Ее защита от широкого спектра угроз —
от кибератак до внутренних инцидентов — перестала быть второстепенной задачей и
превратилась в необходимое условие устойчивого развития и
конкурентоспособности. Эффективное управление этой защитой невозможно без
системного подхода, который реализуется в рамках Системы Управления
Информационной Безопасностью (СУИБ). СУИБ представляет собой не разрозненный
набор технических средств, а целостную структуру политик, процессов, процедур и
организационных мер, построенную на международных стандартах, таких как ISO/IEC
27001.

Особую значимость в жизненном цикле СУИБ приобретает начальный этап —
планирование. Именно на этой стадии закладывается фундамент всей будущей
системы безопасности, определяются ее цели, границы и основные механизмы
функционирования. Этап планирования не сводится лишь к выбору технологических
решений; это комплексный процесс, включающий анализ контекста организации,
оценку рисков, постановку целей, определение ресурсов и разработку программы.
Успех или неудача всей СУИБ во многом предопределены тщательностью и глубиной
работ, выполненных на данной стадии.

Целью данного реферата является всестороннее исследование этапа планирования
СУИБ. В работе будут последовательно рассмотрены основные действия и процессы,
выполняемые на этой ключевой стадии: от определения области применения системы
и оценки рисков до формулирования политики безопасности и разработки плана
обработки рисков. Анализ этих элементов позволит сформировать целостное
представление о том, как грамотное планирование обеспечивает создание
адекватной, экономически обоснованной и интегрированной в бизнес-процессы
системы защиты информации.
