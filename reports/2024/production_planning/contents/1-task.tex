\structure{УСЛОВИЕ~ДОМАШНЕГО~ЗАДАНИЯ}

Предприятие планирует организовать производство нового изделия, используя
собственные и заемные средства. Проведены исследования рынка, что позволило
ориентироваться на определенную величину проектной цены изделия $\text{Ц}_{\text{пр.и}}$
и дать прогноз ожидаемого проектного объема продаж $q_{\text{пр}}$. Предполагается
проводить определенную ценовую политику при производстве и реализации продукции,
влияя тем самым на ожидаемый объем продаж в каждом году производства (установлены
значения коэффициента эластичности спроса $k_{\text{э}}$, при этом ожидаемый объем
продаж реагирует на изменение цены в интервале $\pm \Delta$ от величины
$\text{Ц}_{\text{пр.и}}$.

\subsection*{При выполнении задания необходимо:}

\begin{enumerate}
\item Рассчитать:
    \begin{enumerate}
        \item продолжительность периода освоения производства нового изделия -- $t_{\text{осв}}$;
        \item по каждому $j$-ому году производства изделия:
        \begin{enumerate}
            \item максимально возможный годовой выпуск продукции $N_{\text{max.год.j}}$;
            \item среднюю трудоёмкость единицы продукции $T_{\text{ср.j}}$.
        \end{enumerate}
    \end{enumerate}

\item Используя заданные значения $k$, $\Delta$ обосновать для каждого года
    производства плановую цену $C_{\text{пл}}$ и ожидаемый плановый объём продаж
    $q_{\text{пл.j}}$. Для планируемого варианта освоения производства:
    \begin{enumerate}
        \item рассчитать по каждому $j$-ому году производства:
        \begin{enumerate}
            \item среднегодовую себестоимость единицы продукции $S_{\text{ср.j}}$.
            \item себестоимость годового объёма продукции $S_{\text{год.j}}$;
            \item выручку от реализации продукции $W_{\text{год.j}}$;
            \item прибыль от производства и реализации продукции $P_{\text{год.j}}$;
            \item среднегодовую численность основных рабочих $C_{\text{ср.j}}$;
            \item фонд оплаты труда основных рабочих $\text{Ф}_{\text{опл.j}}$;
        \end{enumerate}
        \item обосновать тактику возврата заёмных средств.
    \end{enumerate}

\item Дать оценку экономической целесообразности освоения производства нового изделия.
    Предложить возможные направления использования получаемой в каждом году прибыли.
    Выполнить сводную таблицу основных показателей, отражающую планируемый вариант
    освоения производства нового изделия.

\item Использовать графическое представление рассчитываемых показателей в виде
    диаграмм, графиков.
\end{enumerate}

\subsection*{Общие для всех вариантов задания:}

\begin{enumerate}
    \item Новое изделие предполагается выпускать в течение 5 лет ($t_{\text{n}} = 5$ лет);
    \item Проектная трудоемкость изготовления освоенного изделия $T_{\text{осв}} = 120$ нормо-час;
    \item Среднемесячный выпуск установленного производства (проектный выпуск) $N_{\text{мес.осв}} = 60$ изд/мес.;
    \item Капитальные затраты для обеспечения проектного выпуска (проектные капзатраты) $K_{\text{пр}} = 20$ млн. руб.;
    \item Интенсивность снижения трудоемкости в период освоения (показатель степени «b») зависит от коэффициента готовности
        $k_r$ и рассчитывается по формуле: $b = 0,6 - 5k_r$;
    \item Данные, используемые при укрупненном калькулировании себестоимости продукции изделия:
    \begin{itemize}
        \item затраты на основные материалы и комплектующие $M = 8965$ руб/шт;
        \item средняя часовая ставка оплаты труда основных рабочих $l_{\text{час}} = 112$ руб/час;
        \item дополнительная оплата основных рабочих $\alpha = 15\%$;
        \item страховые взносы $\beta = 30\%$;
        \item цеховые косвенные расходы $k_{\text{ц}} = 150\%$;
        \item общепроизводственные расходы $k_{\text{оп}} = 25\%$;
        \item внереализационные расходы $k_{\text{вп}} = 5\%$.
    \end{itemize}
\end{enumerate}

\subsection*{Задаваемые по вариантам:}

\begin{enumerate}
    \item Собственные капитальные вложения предприятия к началу производства $K_c$, млн.руб.;
    \item Возможный банковский кредит на освоение производства изделия $K_b$, млн.руб.;
    \item Срок возврата кредита $t_{\text{кр}}$, лет;
    \item Процентная ставка за кредит $p$, \%/год;
    \item Коэффициент ежегодного увеличения процентной ставки при превышении срока возврата кредита $k_{\text{у}}$;
    \item Ожидаемое проектное количество продаж по годам производства изделия $q_{\text{пр}}$, шт./год.;
    \item Трудоемкость изготовления первого изделия (начальная трудоемкость) $T_{\text{н}}$, нормо-час.;
    \item Среднемесячный выпуск изделий на период освоения $N_{\text{мес}}$, шт./мес.;
    \item Рост себестоимости изделия на каждый процент недоиспользованных мощностей $k_{p}$, \%;
    \item Коэффициент эластичности спроса $k_{\text{э}}$, \%/%;
    \item Интервал изменения цены $\Delta$, \%;
    \item Проектная цена изделия $\text{Ц}_{\text{пр.и}}$, тыс.руб.
\end{enumerate}

\begin{table}
\caption{Начало значений по вариантам}
\begin{tabular}{|c|c|c|c|c|}
\hline
Вариант & $k_p$ & $k_{\text{э}}$ & $\Delta$ & $\text{Ц}_{\text{пр.и}}$ \\
\hline
Г & 0.4 & 1.5 & 30 & 94 \\
\hline
\end{tabular}
\end{table}

\begin{table}
\caption{Продолжение значений по вариантам}
\begin{tabular}{|c|c|c|c|c|c|c|c|c|c|c|c|c|}
\hline
\multirow{2}{*}{№ вар.} & \multirow{2}{*}{$K_c$} & \multirow{2}{*}{$K_b$} & \multirow{2}{*}{$t_{\text{кр}}$} & \multirow{2}{*}{$p$} & \multirow{2}{*}{$k_y$} & \multicolumn{5}{c|}{$q_{\text{пр}}$ по годам выпуска} & \multirow{2}{*}{$T_{\text{н}}$} & \multirow{2}{*}{$N_{\text{мес}}$} \\ \cline{7-11}
 &  &  &  &  &  & 1 & 2 & 3 & 4 & 5 &  &  \\ \hline
7 & 12,0 & 3.5 & 4 & 8 & 1.3 & 350 & 580 & 600 & 500 & 450 & 540 & 27 \\ \hline
\end{tabular}
\end{table}

